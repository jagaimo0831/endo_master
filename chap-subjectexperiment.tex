\chapter[主観評価実験を踏まえた位置による身体像拡張の差異を評価する被験者実験]%
        {主観評価実験を踏まえた\\位置による身体像拡張の差異を評価する\\被験者実験}

\section{はじめに}
        本章では,主観評価実験を踏まえた位置による身体像拡張の差異を評価する被験者実験を行う.筋電計測位置と羽ばたく感覚の提示位置を変化させた場合の,羽ばたいて飛ぶ感覚の感じ方の違いについて検証を行う.

\section{実験の目的}
        第4章より,筋電計測位置と触覚提示位置によって羽ばたいて飛ぶ感覚の感じ方に違いが生じることが分かった.そこで,筋電計測位置と触覚提示位置による感じ方の違い(身体像拡張の度合い)について,被験者実験より検証を行う.
        
\section{人研究倫理審査について}
        % 倫理審査,コロナ関係は1段落で簡潔に触れる程度とする
        本実験は「東京農工大学 人を対象とする研究に関する倫理審査委員会の倫理審査」を通過しており,実験は被験者の同意を得て行う(「翼で飛ぶ感覚を提示するVR(仮想現実)システムに関する研究」,倫理審査委員会承認番号 210908-0343).
        被験者の募集は学内メーリングリスト, 掲示, アルバイト募集用WEBサイトなどを利用して行う. 被験者の選定方針に関しては,胴体部(胸・腹・背中)に電極を貼る都合上,男性に限定する.ただし, 未成年の場合には保護者の承諾を取ることとする. 

        また,被験者に生じるリスクとしては,実験中に発生するVR酔いや新型コロナウイルス感染症への感染がある.これらのリスクは,感染症予防対策を十分に行い,被験者が体調に違和感を感じたらすぐに対応することで対策をする.

        被験者が回答したアンケートは,研究実施者以外アクセスできないようにし,保管期間(5年間)が過ぎたらシュレッダーに掛けて解読不能にして廃棄する.
        
        % 被験者実験の様子(予定)の図があるともっとわかりやすいかも

\section{実験環境}
        % システムと装置の説明

        \fig{subjectexp-system.png}{width=1\hsize}{(後で差し替え)Subjective expetiment system}
        % Arduinoの存在をPC/Unityの項目に組み込む

        \fig{subjectexp-env.png}{width=1\hsize}{(写真だとわかりずらいので抽象的な図とか)Subjectexperiment environment}

        \fig{hmd-vive_pro_eye.jpg}{width=1\hsize}{HTC VIVE PRO EYE\cite{htc-vive}}
        \fig{VirtualWingsV2.pdf}{width=1\hsize}{Virtual Wings model version 2}

        \fig{subjectexp-haptic_dev_motor.png}{width=1\hsize}{(回路図か何かを併記すると見やすい)Haptics display(vibration)}
        \fig{subjectexp-haptic_dev_ems.png}{width=1\hsize}{(回路図か何かを併記すると見やすい)Haptics display(EMS)}
        

        被験者実験を行う実験環境について述べる.被験者実験では,第4章の主観評価実験と同様に\figref{subjectexp-system.png}のような環境のシステムで実験を行う.

        筋電計測装置にはMyoWare(\figref{MyoWare.pdf})を使用し,Arduinoを用いて筋電位の値を取得する.
        取得した筋電位の値をPC上のソフトウェア(Unity)へ送る.そして,Unityから視覚提示装置と触覚提示装置を制御する.
        視覚提示装置にはVIVE PRO EYE(HTC社)\figref{hmd-vive_pro_eye.jpg}を用いる.VIVE PRO EYEを用いて,仮想翼(\figref{VirtualWingsV2.pdf})が生えている様子を,1人称視点で提示する.
        触覚提示装置は,\figref{subjectexp-haptic_dev_motor.png}と\figref{subjectexp-haptic_dev_ems.png}を用いる.\figref{subjectexp-haptic_dev_motor.png}は,偏心モータをArduinoからPWMで制御し振動を与える装置である.\figref{subjectexp-haptic_dev_motor.png}は,低周波治療器(Omron HV-F127\cite{Omron-HV-F127})をArduinoで制御し,電気刺激を与える装置である.
        % (修正)触覚提示装置に関してもう少し詳しく書いた方が良い(Ex. 通信方式(シリアル),データの処理...)

        % 操縦方法について(物理環境→操作方法→提示)
        実験の物理環境について述べる.
        本実験でのVR空間では,飛行を行いやすくするために重力加速度$g^{\prime}$を月と同等(\equref{gravity})に設定する.
        \begin{eqnarray}
                \equlabel{gravity}
                g^{\prime}=1.62\;[m/s^{2}]
        \end{eqnarray}

        また,飛行の際に\equref{air_resistance}のような,速度に比例した空気抵抗$R\;[N]$を与える.
        \begin{eqnarray}
                \equlabel{air_resistance}
                \bm{R}=k\bm{v} \;[N]
        \end{eqnarray}
        $k$は比例定数($k=500$).
        % 進行方向,上昇方向どちらも比例定数は等しい

        VR空間内での具体的な飛行方法について述べる.
        まず,翼の操作方法は,操作位置(筋電計測位置)を力ませている間は仮想翼が内側に羽ばたき,弛緩させると仮想翼を広げる用に設計する.仮想翼が操作位置を力ませ仮想翼を羽ばたかせることにより,進行・上昇方向へ力が発生し飛行することができる.この際発生する力$F\;[N]$は\equref{force}に従う.
        % この式についてのrefをしっかりする(空機抵抗の終端速度の式を参考にしたが,もっと理屈的に説明できるように)
        \begin{eqnarray}
                \equlabel{force}
                \bm{F}=\frac{a}{\bm{l}}  \tanh\left(\,\frac{x}{a}\,\right) \;[N]
        \end{eqnarray}
        \begin{eqnarray}
                \equlabel{force_prop}
                \bm{l}=\begin{pmatrix}400 \\ 500 \end{pmatrix}
        \end{eqnarray}
        $l$はそれぞれ1行目が進行方向,2行目が上昇方向の比例定数.$a\;[N]$と$x\;[N]$はそれぞれ,筋電位の最大値と計測された筋電位である.

        また,仮想翼を内側に羽ばたかせた際に,触覚提示を行う.この時の触覚提示の強さは計測された筋電位に応じて変化するようにする.ウェーバー・フェヒナーの法則\cite{10011340169}より,人間の感覚の大きさは受ける刺激の強さの対数に比例することが知られている.
        \begin{eqnarray}
                \equlabel{weber_fechner}
                P = k \log{e}\frac{I}{I_{0}}
        \end{eqnarray}
        $P:\;$感覚の強さ(Perception),$I:\;$刺激の強さ(Intensity of stimulation),$I_{0}:\;$感覚の強さが0になる刺激の強さ,$k:\;$刺激固有の定数.

        従って,触覚提示の強さを対数に比例するように\equref{haptics}に従うようにする.
        
        \begin{eqnarray}
                \equlabel{haptics}
                hoge=hoge
        \end{eqnarray}
        

        



        

        システム構成,ハードウェア,ソフトウェア,操作方法

\section{実験方法}
        実験の順番,アンケートの内容(リッカート尺度,t検定)

\section{結果考察}
        結果(グラフ),考察,課題

\section{おわりに}

\section{未}

        本節では,関連動向調査分析と問題点・課題の提示を行う.
        (※追加で身体像拡張について?)



        被験者はHMD,筋電計測装置,羽ばたく感覚の提示装置を装着し,仮想翼を操縦する.
この際,被験者の筋電のデータを記録する.
筋電計測装置に関してはMyoWare(Advancer Technologies)を使用し体に直接貼り付けて計測を行う.
筋電取得位置は関節動作を伴わない静的な筋収縮が容易な部位である胸肩部・腹部・臀部を検討している.
羽ばたく感覚の提示に関しては,ハプティックスーツによる振動・押す力,またはEMS 機器の筋収縮作用による疑似的な力覚提示によって行う.
羽ばたく感覚提示の位置に関しては仮想翼が存在する背中から体の側面を検討している.
その後,操縦中の没入感に関して被験者の回答を得る.
被験者へは筋電取得箇所と力覚提示提示位置ごと没入感の違いについての回答を得る.具体的には以下のような内容を検討している.
• 筋電計測位置別の没入感
• 力覚提示位置別の没入感
• ハプティックスーツとEMS 機器の没入感
• 没入感において筋電計測位置の羽ばたく感覚の提示位置重要性の比較
• どの組み合わせが一番没入感が高かったか
本実験は「東京農工大学人を対象とする研究に関する倫理審査委員会の倫理審査」を通過しており,実験は被験者の同意を得て行う.また,被験者に生じるリスクとしては,実験中に発生するVR 酔いや新型コロナウイルス感染症への感染がある.これらのリスクは,感染症予防対策を十分に行い,被験者が体調に違和感を感じたらすぐに対応することで対策をする.

※筋電以外の計測法案...腹囲の変化,肩の上下動の計測




        % 被験者実験の資料の内容をちゃんと反映する()

        MyoとUnity通信\\    
        myowareとUnity通信\\

        \fig{subjectexp-human_model.png}{width=1\hsize}{Human model}
        \fig{Movement_of_VirtualWingsV2.pdf}{width=1\hsize}{Virtual presentation of Virtual wings}


        浮遊感や飛ぶ感覚を与える研究は多く行われてきた.視覚刺激をによって発生する落下感覚に関しての研究\cite{奥川夏輝2017VR空間における視覚刺激によって発生する落下感覚の分析}や身体幇助メカニズムを用いた飛行体験装置の提案\cite{鈴木拓馬2014hmd}等がある.また,飛行しているドローンを上半身のジェスチャーで制御し,ドローンからの映像をヘッドマウントディスプレイ(以下HMD)によって与えることで飛ぶ感覚を提示する研究\cite{rognon2018flyjacket}もある.

        浮遊感と飛ぶ感覚の研究に対して,鳥のように羽ばたいて飛ぶ感覚を与える研究はまだ少ない.羽ばたいて飛ぶ感覚を与える研究の例としては,飛行中の鳥の体験をすることができる装置であるBirdly\cite{rheiner2014birdly}やHypersuit\cite{hypersuit}がある.操縦装置にうつ伏せで搭乗し手と腕を用いて翼を動かしながら,鳥視点での景色の映像を提示することで,飛行中の鳥のような体験できる装置である.


        \section{被験者実験}
        \fig{position_of_mesurement.png}{width=1\hsize}{position of mesurement}
        
        % 実験の目的→方法の順番で書くこと
        被験者実験では,筋電計測位置と羽ばたく感覚の提示位置を変化させた場合の没入感の違いについて検証する.
        
        被験者はHMD,筋電計測装置,羽ばたく感覚の力覚提示装置を装着し,仮想翼を操縦する.この際,被験者の筋電のデータを記録する.
        筋電計測装置に関してはMyoWareを使用し体に直接貼り付けて計測を行う.筋電取得位置は関節動作を伴わない静的な筋収縮が容易な部位である胸肩部・腹部・臀部を検討している.
        羽ばたく感覚の力覚提示に関しては,ハプティックスーツ等による振動・押し力,またはEMS機器の筋収縮作用による疑似的な力覚提示によって行う.羽ばたく感覚の提示位置に関しては仮想翼が存在する背中から体の側面を検討している.
        
        その後,操縦中の没入感に関してアンケートを行う.被験者へは筋電取得箇所と力覚提示提示位置ごと没入感の違いについての回答を得る.具体的には以下のような内容を検討している.
        \begin{itemize}
        \item 筋電計測位置別の没入感
        \item 力覚提示位置別の没入感
        \item 振動・押し力提示機器とEMS機器の没入感
        \item 筋電計測位置と力覚提示位置の重要性比較
        \item 一番没入感の高い組み合わせ
        \end{itemize}
    
        
       
        
    \section{被験者アンケート}
    リッカート尺度,t検定
    
    

\section{hoge}        
        ・岩垂先輩のも参考文献に入れる(+早稲田の磐田研?):第3,第4の腕シリーズ(腕を増やすシリーズ) 

