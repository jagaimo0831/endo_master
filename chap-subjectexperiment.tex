\chapter[主観評価実験を踏まえた位置による身体像拡張の差異を評価する被験者実験]%
        {主観評価実験を踏まえた\\位置による身体像拡張の差異を評価する\\被験者実験}

\section{はじめに}
        本節では,関連動向調査分析と問題点・課題の提示を行う.
        (※追加で身体像拡張について?)

        MyoとUnity通信\\    
        myowareとUnity通信\\
        
        \fig{VirtualWingsV2.pdf}{width=1\hsize}{Virtual Wings model version 2}
    \fig{Movement_of_VirtualWingsV2.pdf}{width=1\hsize}{Virtual presentation of Virtual wings}

\section{実験環境}
        浮遊感や飛ぶ感覚を与える研究は多く行われてきた.視覚刺激をによって発生する落下感覚に関しての研究\cite{奥川夏輝2017VR空間における視覚刺激によって発生する落下感覚の分析}や身体幇助メカニズムを用いた飛行体験装置の提案\cite{鈴木拓馬2014hmd}等がある.また,飛行しているドローンを上半身のジェスチャーで制御し,ドローンからの映像をヘッドマウントディスプレイ(以下HMD)によって与えることで飛ぶ感覚を提示する研究\cite{rognon2018flyjacket}もある.

        浮遊感と飛ぶ感覚の研究に対して,鳥のように羽ばたいて飛ぶ感覚を与える研究はまだ少ない.羽ばたいて飛ぶ感覚を与える研究の例としては,飛行中の鳥の体験をすることができる装置であるBirdly\cite{rheiner2014birdly}やHypersuit\cite{hypersuit}がある.操縦装置にうつ伏せで搭乗し手と腕を用いて翼を動かしながら,鳥視点での景色の映像を提示することで,飛行中の鳥のような体験できる装置である.


        \section{被験者実験}
        \fig{position_of_mesurement.png}{width=1\hsize}{position of mesurement}
        
        % 実験の目的→方法の順番で書くこと
        被験者実験では,筋電計測位置と羽ばたく感覚の提示位置を変化させた場合の没入感の違いについて検証する.
        
        被験者はHMD,筋電計測装置,羽ばたく感覚の力覚提示装置を装着し,仮想翼を操縦する.この際,被験者の筋電のデータを記録する.
        筋電計測装置に関してはMyoWareを使用し体に直接貼り付けて計測を行う.筋電取得位置は関節動作を伴わない静的な筋収縮が容易な部位である胸肩部・腹部・臀部を検討している.
        羽ばたく感覚の力覚提示に関しては,ハプティックスーツ等による振動・押し力,またはEMS機器の筋収縮作用による疑似的な力覚提示によって行う.羽ばたく感覚の提示位置に関しては仮想翼が存在する背中から体の側面を検討している.
        
        その後,操縦中の没入感に関してアンケートを行う.被験者へは筋電取得箇所と力覚提示提示位置ごと没入感の違いについての回答を得る.具体的には以下のような内容を検討している.
        \begin{itemize}
        \item 筋電計測位置別の没入感
        \item 力覚提示位置別の没入感
        \item 振動・押し力提示機器とEMS機器の没入感
        \item 筋電計測位置と力覚提示位置の重要性比較
        \item 一番没入感の高い組み合わせ
        \end{itemize}
    
        
        % --------以下倫理審査の書類を参考にして作成--------------
        % 倫理審査,コロナ関係は1段落で簡潔に触れる程度とする
        本実験は「東京農工大学 人を対象とする研究に関する倫理審査委員会の倫理審査」を通過しており,実験は被験者の同意を得て行う.
        % 被験者の募集は学内メーリングリスト, 掲示, アルバイト募集用WEBサイトなどを利用して行う. 被験者の選定方針に関しては特に定めない.ただし, 未成年の場合には保護者の承諾を取ることとする. 
        また,被験者に生じるリスクとしては,実験中に発生するVR酔いや新型コロナウイルス感染症への感染がある.これらのリスクは,感染症予防対策を十分に行い,被験者が体調に違和感を感じたらすぐに対応することで対策をする.
        
        % 被験者実験の様子(予定)の図があるともっとわかりやすいかも
        
    \section{被験者アンケート}
    リッカート尺度,t検定
    
    

\section{hoge}        
        ・岩垂先輩のも参考文献に入れる(+早稲田の磐田研?):第3,第4の腕シリーズ(腕を増やすシリーズ) 

\section{おわりに}