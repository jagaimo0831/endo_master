\include{my_layout_grad}


\usepackage{ikuo}%%便利コマンド集.

\usepackage[dvipdfmx]{hyperref}  % 目次や参考文献をリンクにする。
\usepackage{pxjahyper} %% これを入れるとしおりが文字化けしない。out2uniが不要になる。
%% \hypersetup{bookmarksnumbered=true}
\hypersetup{colorlinks=true}
\hypersetup{linkcolor=black}
%% \hypersetup{linkbordercolor=black}
\hypersetup{urlcolor=black}
%% \hypersetup{urlbordercolor=black}
\hypersetup{citecolor=black}
%% \hypersetup{citebordercolor=black}

\usepackage{url} % \url のために必要。パッケージが無い人は探して入れる。
%% \url{http://nile.ulis.ac.jp/~yuka/}のようにして使う。

\newcommand{\FIGDIR}{./fig}        %図を置くディレクトリを指定する


\date{令和1年度卒業論文}
\title{モータ1自由度直列弾性活用連続跳躍ロボットにおける脚の回転運動による着地姿勢制御の研究}
\author{指導教員 水内 郁夫 准教授 \\
\ \\
東京農工大学 \\
工学部 機械システム工学科 \\
\ \\
平成30年度入学\\
18255517\\
{\bf 横山 颯太}}

\begin{document}
\setlength{\baselineskip}{20pt}
\maketitle
\tableofcontents

%%各章は別ファイルにして以下にinculudeすると良い.
% \include{introduction}
\chapter[序論]%
        {序論}

        \fig{WingMan.png}{width=1\hsize}{Flying with flapping virtual wings independent of the limbs}

        ヒトは古くから空を飛ぶことに憧れを抱いている.
        これまで私たちは,飛行機やハンググライダーといった乗り物を用いることで飛行体験をしてきた.
        また,個人飛行装置\footnote{Portable Parsonal Airmobility System...ジェットパック,動力式ウイングスーツ,動力式パラフォイル(風により展開される柔軟構造を持つ翼.Ex.パラグライダーの翼)}
        のような,ウェアラブルな装置で空を飛ぶ研究も行われている\cite{gravityindustries}.
        % 実際に飛ぶことにはリスクやコストが伴うが,VR装置を使用することで簡単に飛行体験が可能である.
        しかし,実際に空を飛ぶことは墜落などのリスクや燃料といったコスト,機器を操縦するための技術が必要となる.
        VR(Virtual Reality: 仮想現実)システムを使用することで,それらリスクやコストを回避し,乗り物・ウェアラブルな装置を問わず簡単に飛行体験が可能となる.

        \figref{WingMan.png}は,四肢から独立した翼で羽ばたいて飛ぶ様子を示した図である.
        本研究では,VRシステムを用いて\figref{WingMan.png}のように,ヒトの背中から翼が生えた生物になり羽ばたいて飛ぶ感覚の提示手法を提案する.

\section{本研究での羽ばたいて飛ぶ感覚の定義}
        本論文では「浮遊感」,「飛ぶ感覚」,「羽ばたいて飛ぶ感覚」を\figref{intro-Classification_of_floating_feeling.pdf}のように位置付ける.
        \fig{intro-Classification_of_floating_feeling.pdf}{width=1\hsize}{Classification of floating feeling}

        \begin{itemize}
                \item 「浮遊感」\\
                ...空中に浮いて漂っている感覚.
                \item 「飛ぶ感覚」\\
                ...「浮遊感」に,空中を移動する感覚を追加した感覚.
                \item 「羽ばたいて飛ぶ感覚」\\
                ...「飛ぶ感覚」に,翼を羽ばたかせる感覚を追加した感覚.
        \end{itemize}

\section{研究の背景と目的}
% VRの飛ぶ研究についてもっと充実させる(分類して一気にciteする感じ)
% 身体像拡張は次章で引用

        VR装置を用いた「浮遊感」や「飛ぶ感覚」を与える研究は多く行われてきた.視覚刺激によって発生する落下感覚に関しての研究\cite{奥川夏輝2017VR空間における視覚刺激によって発生する落下感覚の分析}や身体幇助メカニズムを用いた飛行体験装置の提案\cite{鈴木拓馬2014hmd}等がある.また,飛行しているドローンを上半身のジェスチャーで制御し,ドローンからの映像をHMD(Head Mounted Display: ヘッドマウントディスプレイ)によって与えることで飛ぶ感覚を提示する研究\cite{rognon2018flyjacket}もある.

        \fig{Birdly.jpg}{width=0.7\hsize}{System of presenting the sensation of flying with flapping wings\cite{rheiner2014birdly}}

        「羽ばたいて飛ぶ感覚」を与える研究について,\figref{Birdly.jpg}のような操縦装置に搭乗し,飛行中の鳥の体験をすることができる装置の研究が行われている\cite{rheiner2014birdly}\cite{hypersuit}.
        上記装置は,操縦装置にうつ伏せで搭乗し手と腕を用いて翼を動かしながら,鳥視点での景色の映像を提示することで,飛行中の鳥のような体験できる装置である.この方法の場合,大がかりな装置が必要であることや,手足の動きが制限されるといったデメリットが存在する.
% 四肢の動きを用いないことで,VR飛行体験の質(※)が向上する.(※質とは?)その根拠となる論文は?
% 大きく体を動かすことによる疲労感が生まれ,VR体験に影響を与える.(※疲労感がVR体験に影響を与えるかどうかのソースがあると説得力が上がる)
% 体の動きが制限されることによるデメリット等が言えると更に良い.
        また,羽ばたいて飛ぶ感覚を与える研究はまだ知見が少なく,鳥になり飛ぶ感覚を与える研究が大半であり,トビトカゲのような四肢から独立した翼を持つ生物になり,飛ぶ感覚を与える研究は未だ着目されていない.

        本研究では,四肢の動きを用いずに背中から生えた翼を操作し羽ばたく感覚を提示する手法を提案する.四肢の動きを用いないことで,VR飛行体験中に手足を用いた動作,例えば飛びながら物を投げるといった行為,が可能となりVR飛行体験の幅が広がることが期待できる.

\section{本論文の構成}
        
        本論文は全6章で構成される.以下に各章の概要を述べる.

        \begin{itemize}
                \item 第1章「序論」では,羽ばたいて飛ぶ感覚の定義,本研究の背景と目的について述べた.
                
                \item 第2章「身体像の拡張」では,身体像について説明し,身体像の拡張の仕組みと方法について示し,本研究での身体像拡張のアプローチについて述べる.
                
                \item 第3章「四肢から独立した翼の提示方法」では,四肢から独立した翼の提示方法について述べる.
                
                \item 第4章「提案手法を用いた身体像拡張の主観評価実験」では,提案した身体像拡張の手法を用いて主観評価実験を行う.操作・提示方法の検討,操作・提示位置の検討を行い,それぞれの組み合わせの評価を下す.主観評価実験の結果を踏まえ,被験者実験で比較する対象について述べる.
                
                \item 第5章「主観評価実験を踏まえた位置による身体像拡張の差異を評価する被験者実験」では,主観評価実験を踏まえた位置による身体像拡張の差異を評価する被験者実験を行う.筋電計測位置と羽ばたく感覚の提示位置を変化させた場合の,羽ばたいて飛ぶ感覚の感じ方の違い,触覚提示として振動と電気刺激を用いた装置を比較し検証を行う.
                
                \item 第6章「結論および今後の展望」では,本研究の結論と今後の展望について述べる.
        \end{itemize}
        
        


        

\chapter[設計]%
        {設計}
\section{はじめに}
\section{hoge}
\section{fuga}
\section{おわりに}
\chapter[直列弾性要素を有するモータ1自由度跳躍ロボットの動力学シミュレータの開発]%
        {直列弾性要素を有するモータ1自由度跳躍ロボットの動力学シミュレータの開発}

    \section{はじめに}
    \section{ラグランジュ法を用いた運動方程式の導出}
    \section{ルンゲクッタ法による数値積分}
    \section{おわりに}
\chapter[最急降下法による着地動作のためのモータ入力の最適化]%
        {最急降下法による着地動作のためのモータ入力の最適化}
    
    \section{はじめに}
    \section{着地姿勢の評価関数}
    \section{アルゴリズム}
    \section{おわりに}
\chapter[直列弾性要素を有するモータ1自由度跳躍ロボットの動作実験]%
        {直列弾性要素を有するモータ1自由度跳躍ロボットの動作実験}
    \section{はじめに}
    \section{跳躍動作実験}
    \section{着地姿勢制御実験}
    \section{おわりに}
\chapter[結論および今後の展望]%
        {結論および今後の展望}

\section{結論}
    本稿では,翼を動かして飛ぶ感覚を与える研究に注目し,四肢を用いず翼を操作している感覚の提示方法と,VR空間で翼に作用する力をヒトに伝達する手法を提案した.
    実験装置のシステムを作成し,振動とEMS装置による力覚提示についての有用性についての実験を行った.実験より主観ではあるが,力覚提示として振動とEMS装置を用いることの有用性を確認し,ヒトに本来備わっていない部位である翼の存在を感じ,それを操作している感覚を得た.
    %そして,被験者実験のために倫理審査を行い実験の方法と環境について検討をした.
  
    今後の展望として,ヒトによる没入感を調査するために被験者実験を行う.そして,筋電計測位置・力覚提示位置を変化させた場合の没入感の違いについて検証する.その結果から得られる最も評価が高い筋電計測位置・力覚提示位置より,没入感を向上させる.また,デバイスからヒトへの提示情報として前庭電気刺激による加速度感覚\cite{maeda2005shaking}\cite{青山一真2014前庭電気刺激における逆方向不感電流を用いた加速度感覚の増強}の追加し,さらに飛行体験の没入感を高めることも検討している.
  
\section{今後の展望}
% \include{fig_tab_equ}
\addcontentsline{toc}{chapter}{謝辞}
\markboth{謝辞}{謝辞}
% \chapter*{謝辞}
 修士論文を執筆するに当たり, 指導教官のた東京農工大学工学部機械システム工学科 水内郁夫 教授からは多大なるご指導・ご鞭撻を賜りました.
 研究室に配属して間もない頃,研究は疎かパソコン等の研究に使用するツールに関しても何もわからない自分に水内先生が仰った「ユーザになるな」という言葉は今でも印象に残っています.
 他にも物事を論理的に考え説明することなど,
 数え切れなほどの多くのことを,この3年間で学ばせて頂きました.
 深く感謝の意を申し上げると共に深くお礼を申し上げます.

 また,研究室の先輩・後輩方には
 研究活動に疎い執筆者に多くの助言をくださりました.
 厚くお礼を申し上げ,感謝する次第です.
 そして,研究活動中の相談や雑談をしてくれた同輩である
 西くん,横山くんに感謝します.
 
 
 
 
% \begin{verbatim}
% %% \addcontentsline{toc}{chapter}{謝辞}
% %% \markboth{謝辞}{謝辞}
% %% \chapter*{謝辞}
 修士論文を執筆するに当たり, 指導教官のた東京農工大学工学部機械システム工学科 水内郁夫 教授からは多大なるご指導・ご鞭撻を賜りました.
 研究室に配属して間もない頃,研究は疎かパソコン等の研究に使用するツールに関しても何もわからない自分に水内先生が仰った「ユーザになるな」という言葉は今でも印象に残っています.
 他にも物事を論理的に考え説明することなど,
 数え切れなほどの多くのことを,この3年間で学ばせて頂きました.
 深く感謝の意を申し上げると共に深くお礼を申し上げます.

 また,研究室の先輩・後輩方には
 研究活動に疎い執筆者に多くの助言をくださりました.
 厚くお礼を申し上げ,感謝する次第です.
 そして,研究活動中の相談や雑談をしてくれた同輩である
 西くん,横山くんに感謝します.
 
 
 
 
% \begin{verbatim}
% %% \addcontentsline{toc}{chapter}{謝辞}
% %% \markboth{謝辞}{謝辞}
% %% \chapter*{謝辞}
 修士論文を執筆するに当たり, 指導教官のた東京農工大学工学部機械システム工学科 水内郁夫 教授からは多大なるご指導・ご鞭撻を賜りました.
 研究室に配属して間もない頃,研究は疎かパソコン等の研究に使用するツールに関しても何もわからない自分に水内先生が仰った「ユーザになるな」という言葉は今でも印象に残っています.
 他にも物事を論理的に考え説明することなど,
 数え切れなほどの多くのことを,この3年間で学ばせて頂きました.
 深く感謝の意を申し上げると共に深くお礼を申し上げます.

 また,研究室の先輩・後輩方には
 研究活動に疎い執筆者に多くの助言をくださりました.
 厚くお礼を申し上げ,感謝する次第です.
 そして,研究活動中の相談や雑談をしてくれた同輩である
 西くん,横山くんに感謝します.
 
 
 
 
% \begin{verbatim}
% %% \addcontentsline{toc}{chapter}{謝辞}
% %% \markboth{謝辞}{謝辞}
% %% \include{thanks}
% \end{verbatim}
 
% emacs の人は、M-x comment-region ですね。
% コメント解除は、C-u M-x comment-region ですね。
% \end{verbatim}
 
% emacs の人は、M-x comment-region ですね。
% コメント解除は、C-u M-x comment-region ですね。
% \end{verbatim}
 
% emacs の人は、M-x comment-region ですね。
% コメント解除は、C-u M-x comment-region ですね。

\addcontentsline{toc}{chapter}{参考文献}
\markboth{参考文献}{参考文献}
\bibliographystyle{junsrt}
\bibliography{reference}

\end{document}
