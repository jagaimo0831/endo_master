\include{my_layout_grad}

\usepackage{ikuo}%%便利コマンド集.
\usepackage{jtygm}

\usepackage[dvipdfmx]{hyperref}  % 目次や参考文献をリンクにする。
\usepackage{pxjahyper} %% これを入れるとしおりが文字化けしない。out2uniが不要になる。
%% \hypersetup{bookmarksnumbered=true}
\hypersetup{colorlinks=true}
\hypersetup{linkcolor=black}
%% \hypersetup{linkbordercolor=black}
\hypersetup{urlcolor=black}
%% \hypersetup{urlbordercolor=black}
\hypersetup{citecolor=black}
%% \hypersetup{citebordercolor=black}

\usepackage{url} % \url のために必要。パッケージが無い人は探して入れる。
%% \url{http://nile.ulis.ac.jp/~yuka/}のようにして使う。

% ハイパーリンク関係のセットアップ
\usepackage[dvipdfmx]{hyperref}
\hypersetup{
    colorlinks=true,
    linkcolor=blue,
    filecolor=magenta,
    urlcolor=cyan,
    pdftitle={Sharelatex Example},
    % bookmarks=true,
    pdfpagemode=FullScreen,
    }
\urlstyle{same}
\usepackage{csquotes} 

\newcommand{\FIGDIR}{./fig}        %図を置くディレクトリを指定する


\date{令和3年度修士論文}
% \title{四肢から独立した翼で羽ばたいて飛ぶ感覚の提示 Presenting the sensation of flying with flapping virtual wings independent of the limbs}
\title{
    \vspace{30mm}
    \huge{} \ 四肢から独立した翼で羽ばたいて飛ぶ感覚の提示\\
     \\
    %\LARGE{}(英文)\\
        \LARGE{} \ Presenting the sensation of flying with flapping virtual wings independent of the limbs
}
\author{指導教員 水内 郁夫 教授 \\
\ \\
東京農工大学 \\
工学部 機械システム工学科 \\
\ \\
令和2年度入学\\
20643010\\
{\bf 遠藤 健}}

\begin{document}
\setlength{\baselineskip}{20pt}
\maketitle
\tableofcontents

%%各章は別ファイルにして以下にinculudeすると良い.
% \chapter[ロボメック2021の内容]%
        {ロボメック2021の内容}
    
\section{緒言}

\fig{WingMan.png}{width=1\hsize}{Flying with flapping virtual wings independent of the limbs}

\fig{How2present_the_feeling_of_flapping_eng.pdf}{width=1\hsize}{Research concept}

ヒトは古くから空を飛ぶことに憧れを抱いている.実際に飛ぶことにはリスクやコストが伴うが,VR装置を使用することで簡単に飛行体験が可能である.
\figref{WingMan.png}は,四肢から独立した翼で羽ばたいて飛ぶ様子を示した物である.
本研究では,
VR装置を用いて\figref{WingMan.png}のように羽ばたいて飛ぶ感覚の提示手法を提案する.

    \subsection{浮遊感・飛ぶ感覚・羽ばたいて飛ぶ感覚の定義}
    本研究では「羽ばたいて飛ぶ感覚」を「浮遊感」,「飛ぶ感覚」を用いて次のように位置づける.
    「浮遊感」とは,空中に浮いて漂っている感覚を指す.「飛ぶ感覚」は浮遊感に空中を移動する感覚を追加したものとする.本稿で注目する「羽ばたいて飛ぶ感覚」は,飛ぶ感覚に加え,翼を羽ばたかせる感覚を追加したものである.

    \subsection{関連研究}    
    浮遊感や飛ぶ感覚を与える研究は多く行われてきた.視覚刺激をによって発生する落下感覚に関しての研究\cite{奥川夏輝2017VR空間における視覚刺激によって発生する落下感覚の分析}や身体幇助メカニズムを用いた飛行体験装置の提案\cite{鈴木拓馬2014hmd}等がある.また,飛行しているドローンを上半身のジェスチャーで制御し,ドローンからの映像をヘッドマウントディスプレイ(以下HMD)によって与えることで飛ぶ感覚を提示する研究\cite{rognon2018flyjacket}もある.

    浮遊感と飛ぶ感覚の研究に対して,鳥のように羽ばたいて飛ぶ感覚を与える研究はまだ少ない.羽ばたいて飛ぶ感覚を与える研究の例としては,飛行中の鳥の体験をすることができる装置であるBirdly\cite{rheiner2014birdly}やHypersuit\cite{hypersuit}がある.操縦装置にうつ伏せで搭乗し手と腕を用いて翼を動かしながら,鳥視点での景色の映像を提示することで,飛行中の鳥のような体験できる装置である.

    \subsection{研究目的}
    従来の羽ばたいて飛ぶ感覚を与える研究は,ヒトが鳥のように腕を動かすことによって羽ばたいて飛ぶ感覚を提示していた.しかし,大がかりな装置が必要であることや,手足の動きが制限されるといったデメリットが存在する.
    また,羽ばたいて飛ぶ感覚を与える研究において,\figref{WingMan.png}のような背中から翼が生えた生物になる感覚を提示する研究はまだ着目されていない.

    そこで四肢から独立した翼を用いて,羽ばたいて飛ぶ感覚を提示する方法を提案する.
    本研究では,
    % 大がかりな装置を使用しないこと,(核ではないので排除)
    四肢を用いずに翼を操作している感覚の提示手法とVR空間で翼に作用する力をヒトに提示する手法を考案する.
    これにより,羽ばたいて飛んでいる状態での投擲や射撃が可能となる.これらのように飛行体験中に手足を使用する新しいアプリケーションが期待できる.

    本研究では,ヒトに翼が生えている感覚を与えるために,身体像の拡張について注目をする.

    
    \subsection{身体像の拡張}
    ヒトは身体像と呼ばれる,自身の身体形状を知覚する能力を有している.それにより自己とそれ以外を区別することができる.しかし,自己以外の部分に身体像が拡張する場合がある.身体像の拡張に関する代表的な研究として,
    % 偽物の手である
    ラバーハンドをあたかも自分の手のように感じるラバーハンド錯覚についての研究がある\cite{botvinick1998rubber}.視界から隠れた本物の手と目の前にあるラバーハンドに絵筆等で2分から20分程度同期した触覚刺激を与え続けると,ラバーハンド上に触覚刺激を知覚するという錯覚現象である.
    このように提示される視覚情報と,触覚情報の位置が一致または近しければ身体像を拡張することが可能となる.
    ラバーハンド錯覚ではヒトは情報を受けとるだけであったが,ヒトから情報を送信し,それに対する返信を受け取ることで身体像の拡張をより円滑にすることができると考える.
    
    身体像の拡張には情報の双方向性が重要であることを踏まえ,本研究では\figref{How2present_the_feeling_of_flapping_eng.pdf}のような形で身体像の拡張を行う.
    ヒトから仮想翼へは,翼を動かす指令を与える.仮想翼からヒトへは,翼が生えている様子,翼を動かして飛んでいる様子,翼へ作用する空気抵抗の感覚を伝える.上記より,四肢から独立した翼で羽ばたいて飛ぶ感覚を提示する.
    
    \subsection{ヒトから仮想翼}
    まず,ヒトから仮想翼へ翼を動かす指令を与える方法について述べる.

    ヒトから仮想翼を操作する方法として,コントローラやジェスチャによる操作や,生体信号を用いることが挙げられる.
    本研究では,四肢以外で動かすことが目的なので,主に手を用いるコントローラや,手足の動きが必要となるジェスチャではなく,生体信号を用いる.また,生体信号の中でも数値の取得が容易な筋電位によって翼を操作する.

    \fig{How2present_force_applied2wings_eng.pdf}{width=1\hsize}{How to present force applied virtual wings}




   
        



\chapter[序論]%
        {序論}

        \fig{WingMan.png}{width=1\hsize}{Flying with flapping virtual wings independent of the limbs}

        ヒトは古くから空を飛ぶことに憧れを抱いている.
        これまで私たちは,飛行機やハンググライダーといった乗り物を用いることで飛行体験をしてきた.
        また,個人飛行装置\footnote{Portable Parsonal Airmobility System...ジェットパック,動力式ウイングスーツ,動力式パラフォイル(風により展開される柔軟構造を持つ翼.Ex.パラグライダーの翼)}
        のような,ウェアラブルな装置で空を飛ぶ研究も行われている\cite{gravityindustries}.
        % 実際に飛ぶことにはリスクやコストが伴うが,VR装置を使用することで簡単に飛行体験が可能である.
        しかし,実際に空を飛ぶことは墜落などのリスクや燃料といったコスト,機器を操縦するための技術が必要となる.
        VR(Virtual Reality: 仮想現実)システムを使用することで,それらリスクやコストを回避し,乗り物・ウェアラブルな装置を問わず簡単に飛行体験が可能となる.

        \figref{WingMan.png}は,四肢から独立した翼で羽ばたいて飛ぶ様子を示した図である.
        本研究では,VRシステムを用いて\figref{WingMan.png}のように,ヒトの背中から翼が生えた生物になり羽ばたいて飛ぶ感覚の提示手法を提案する.

\section{本研究での羽ばたいて飛ぶ感覚の定義}
        本論文では「浮遊感」,「飛ぶ感覚」,「羽ばたいて飛ぶ感覚」を\figref{intro-Classification_of_floating_feeling.pdf}のように位置付ける.
        \fig{intro-Classification_of_floating_feeling.pdf}{width=1\hsize}{Classification of floating feeling}

        \begin{itemize}
                \item 「浮遊感」\\
                ...空中に浮いて漂っている感覚.
                \item 「飛ぶ感覚」\\
                ...「浮遊感」に,空中を移動する感覚を追加した感覚.
                \item 「羽ばたいて飛ぶ感覚」\\
                ...「飛ぶ感覚」に,翼を羽ばたかせる感覚を追加した感覚.
        \end{itemize}

\section{研究の背景と目的}
% VRの飛ぶ研究についてもっと充実させる(分類して一気にciteする感じ)
% 身体像拡張は次章で引用

        VR装置を用いた「浮遊感」や「飛ぶ感覚」を与える研究は多く行われてきた.視覚刺激によって発生する落下感覚に関しての研究\cite{奥川夏輝2017VR空間における視覚刺激によって発生する落下感覚の分析}や身体幇助メカニズムを用いた飛行体験装置の提案\cite{鈴木拓馬2014hmd}等がある.また,飛行しているドローンを上半身のジェスチャーで制御し,ドローンからの映像をHMD(Head Mounted Display: ヘッドマウントディスプレイ)によって与えることで飛ぶ感覚を提示する研究\cite{rognon2018flyjacket}もある.

        \fig{Birdly.jpg}{width=0.7\hsize}{System of presenting the sensation of flying with flapping wings\cite{rheiner2014birdly}}

        「羽ばたいて飛ぶ感覚」を与える研究について,\figref{Birdly.jpg}のような操縦装置に搭乗し,飛行中の鳥の体験をすることができる装置の研究が行われている\cite{rheiner2014birdly}\cite{hypersuit}.
        上記装置は,操縦装置にうつ伏せで搭乗し手と腕を用いて翼を動かしながら,鳥視点での景色の映像を提示することで,飛行中の鳥のような体験できる装置である.この方法の場合,大がかりな装置が必要であることや,手足の動きが制限されるといったデメリットが存在する.
% 四肢の動きを用いないことで,VR飛行体験の質(※)が向上する.(※質とは?)その根拠となる論文は?
% 大きく体を動かすことによる疲労感が生まれ,VR体験に影響を与える.(※疲労感がVR体験に影響を与えるかどうかのソースがあると説得力が上がる)
% 体の動きが制限されることによるデメリット等が言えると更に良い.
        また,羽ばたいて飛ぶ感覚を与える研究はまだ知見が少なく,鳥になり飛ぶ感覚を与える研究が大半であり,トビトカゲのような四肢から独立した翼を持つ生物になり,飛ぶ感覚を与える研究は未だ着目されていない.

        本研究では,四肢の動きを用いずに背中から生えた翼を操作し羽ばたく感覚を提示する手法を提案する.四肢の動きを用いないことで,VR飛行体験中に手足を用いた動作,例えば飛びながら物を投げるといった行為,が可能となりVR飛行体験の幅が広がることが期待できる.

\section{本論文の構成}
        
        本論文は全6章で構成される.以下に各章の概要を述べる.

        \begin{itemize}
                \item 第1章「序論」では,羽ばたいて飛ぶ感覚の定義,本研究の背景と目的について述べた.
                
                \item 第2章「身体像の拡張」では,身体像について説明し,身体像の拡張の仕組みと方法について示し,本研究での身体像拡張のアプローチについて述べる.
                
                \item 第3章「四肢から独立した翼の提示方法」では,四肢から独立した翼の提示方法について述べる.
                
                \item 第4章「提案手法を用いた身体像拡張の主観評価実験」では,提案した身体像拡張の手法を用いて主観評価実験を行う.操作・提示方法の検討,操作・提示位置の検討を行い,それぞれの組み合わせの評価を下す.主観評価実験の結果を踏まえ,被験者実験で比較する対象について述べる.
                
                \item 第5章「主観評価実験を踏まえた位置による身体像拡張の差異を評価する被験者実験」では,主観評価実験を踏まえた位置による身体像拡張の差異を評価する被験者実験を行う.筋電計測位置と羽ばたく感覚の提示位置を変化させた場合の,羽ばたいて飛ぶ感覚の感じ方の違い,触覚提示として振動と電気刺激を用いた装置を比較し検証を行う.
                
                \item 第6章「結論および今後の展望」では,本研究の結論と今後の展望について述べる.
        \end{itemize}
        
        


        

\chapter[関連研究]%
        {関連研究}

% この章いるかどうか微妙,だが折角沢山調べたので載せたい?(参考文献を消費したい...)

\section{はじめに}
        本節では,関連動向調査分析と問題点・課題の提示を行う.
        (※追加で身体像拡張について?)

\section{関連研究}
        浮遊感や飛ぶ感覚を与える研究は多く行われてきた.視覚刺激をによって発生する落下感覚に関しての研究\cite{奥川夏輝2017VR空間における視覚刺激によって発生する落下感覚の分析}や身体幇助メカニズムを用いた飛行体験装置の提案\cite{鈴木拓馬2014hmd}等がある.また,飛行しているドローンを上半身のジェスチャーで制御し,ドローンからの映像をヘッドマウントディスプレイ(以下HMD)によって与えることで飛ぶ感覚を提示する研究\cite{rognon2018flyjacket}もある.

        浮遊感と飛ぶ感覚の研究に対して,鳥のように羽ばたいて飛ぶ感覚を与える研究はまだ少ない.羽ばたいて飛ぶ感覚を与える研究の例としては,飛行中の鳥の体験をすることができる装置であるBirdly\cite{rheiner2014birdly}やHypersuit\cite{hypersuit}がある.操縦装置にうつ伏せで搭乗し手と腕を用いて翼を動かしながら,鳥視点での景色の映像を提示することで,飛行中の鳥のような体験できる装置である.


        ・飛ぶ研究
        ・身体像拡張
        上記2種類の研究について記載


\section{hoge}        
        ・岩垂先輩のも参考文献に入れる(+早稲田の磐田研?):第3,第4の腕シリーズ(腕を増やすシリーズ) 

\section{おわりに}
\chapter[身体像の拡張]%
        {身体像の拡張}

\section{はじめに}
    本研究において以下の要素が重要となる.
    \begin{itemize}
        \item ヒトに本来存在しない「翼」を感じさせる(存在)
        \item その翼で「羽ばたいて飛ぶ感覚」を提示する(運動)
    \end{itemize}
    上記の感覚を与えるために,身体像の拡張について注目する.本章では,身体像について説明し,身体像の拡張の仕組みと方法について述べる.

\section{身体像}
    ヒトは身体像(Body image)\cite{head1911sensory}と呼ばれる,自分自身の身体に関するイメージを持っている.
    自身の身体形状を知覚する能力を有している.それにより自己とそれ以外を区別することができる.
    身体像の基盤となる概念に身体図式(Body schema)がある.身体像が意識された身体の形状情報に対し,身体図式は習慣としての身体の表像,つまり無意識下に身体運動を調整している主体であり顕在的な知識があるとは限らない.身体像は身体図式を基盤として構成される顕在的な自己身体に関する知識を指す\cite{nishida-bodyimage}.
    % コトバンクに詳しく書かれている
    身体像や身体図式は,幻肢痛\cite{Ramachandran}の観察により生じた概念である.幻肢痛とは,事故で存在しないはずの失った手や脚に痛みを感じる症状である.幻肢痛が発症する仕組みのとして,脳における各機能の分布(脳地図\cite{池谷裕二2007進化しすぎた脳})が書き変わり,幻肢を自分の意思で動かせないことが原因として挙げられる.
    % 脳地図(ブロードマン脳地図)

\section{身体像拡張}
    自己以外の部分に身体像がダイナミックに変化することがある.このことを身体像の拡張(Body image expansion)と呼ぶ.身体像拡張の例として,手に持った道具(テニスラケットや野球バット)を,その形状を意識せず自分の体の一部であるかのように球を打ち返すといった事が挙げられる\cite{渡辺貴文2005仮想道具による身体像拡張の評価手法に関する研究}.
    身体像の拡張は,言い換えると感覚の情報処理を神経系空拡張する能力の事である.

    身体像拡張は,大きく分類して2種類存在し,1つはラバーハンド錯覚\cite{botvinick1998rubber}のような感覚のリマッピング,もう1つは先に挙げた道具使用時に身体像がダイナミックに拡張することである.

    ラバーハンド錯覚とは, ラバーハンドをあたかも自分の手のように感じる錯覚である..視界から隠れた本物の手と目の前にあるラバーハンドに絵筆等で2分から20分程度同期した触覚刺激を与え続けると,ラバーハンド上に触覚刺激を知覚するという錯覚現象である.

    道具への身体像拡張のとして,二ホンザルを用いた道具への身体像拡張を神経生理学的に示した研究がある\cite{iriki1996coding}.この研究では,道具使用時の二ホンザルの登頂連合野における手の体性感覚受容やと手近傍の視覚受容野を持つバイモーダルニューロンの活動を観測することにより,サルの身体像が道具先端まで拡張している事を示した.

    ラバーハンド錯覚と同様な身体像拡張の研究に関しては,視触覚を同期することで,遠隔にあるロボットやアバターへ乗り移ったような感覚が生成可能ということが知られてる\cite{舘201533_215}\cite{iwadare2017thirdarm}.
    道具への身体像拡張に関しては,手先から道具への身体像において,余剰筋力を用いて第3の腕となるロボットアームを操作する研究\cite{iwadare2017thirdarm}\cite{岩垂真哉2016余剰筋力を用いた第三の腕ロボットの操縦}や顔面ベクトルを用いて第3の腕を操作する研究\cite{iwasaki2017research},両足を用いて第3・第4の腕を操作する研究がある\cite{sasaki2017metalimbs}.
    
    このように,ヒトと自由度やダイナミクスが類似した遠隔ロボットやアバタを,身体動作と完全に同期させることで,乗り移ったような感覚が生成可能ということが知られてる.しかし,身体像拡張の研究において身体像をヒトと異なる構造の対象に投射することや,四肢以外から身体像を拡張させる知見はまだ少ないのが現状である.


    本研究では,身体像拡張の中でも道具への身体像拡張に注目する.ラバーハンド錯覚と道具への身体像拡張のどちらにおいても,視覚情報と触覚情報の位置が一致または近しければ身体像を拡張することが可能と考える.また,ヒトから情報を送信し,それに対する返信を受け取ることで身体像の拡張をより円滑にすることができると考える.
    このように,身体像の拡張には情報の双方向性が重要であることを踏まえ,本研究では\figref{How2present_the_feeling_of_flapping_eng.pdf}のような形で身体像の拡張を行う.
    ヒトから仮想翼へは,翼を動かす指令を与える.仮想翼からヒトへは,翼が生えている様子,翼を動かして飛んでいる様子,翼へ作用する空気抵抗の感覚を伝える.上記より,四肢から独立した翼で羽ばたいて飛ぶ感覚を提示する.\\


\section{おわりに}
    本章では,身体像の概要と身体像拡張の例と方法について述べた.そして身体像の拡張に着目しヒトに本来無い「翼」を感じさせる方法,その翼で羽ばたいて飛ぶ感覚の提示方法について述べた.
    

\chapter[四肢から独立した翼の提示方法]%
        {四肢から独立した翼の提示方法}

\section{はじめに}
    本節では,四肢から独立した翼の提示方法について述べる.



\section{身体像拡張を行う方法}
    まず,ヒトから仮想翼へ翼を動かす指令を与える方法について述べる.

    ヒトから仮想翼を操作する方法として,コントローラやジェスチャによる操作や,生体信号を用いることが挙げられる.
    本研究では,四肢以外で動かすことが目的なので,主に手を用いるコントローラや,手足の動きが必要となるジェスチャではなく,生体信号を用いる.また,生体信号の中でも数値の取得が容易な筋電位によって翼を操作する.

    \fig{How2present_force_applied2wings_eng.pdf}{width=1\hsize}{How to present force applied virtual wings}

    \fig{hmd_vection.png}{width=1\hsize}{Virtual presentation by HMD}

    次に,仮想翼からヒトへ情報を与える方法について述べる.

    ヒトへ働きかける感覚として主に五感が挙げられる.
    ヒトへ働きかける情報として,五感の中でも力覚(触覚)と視覚,聴覚が重要と考えた.
    聴覚に関しては空間的定位,ここでは翼のある場所を認識する場合において,一般的に視覚よりも情報としての重要度が低い\cite{岡嶋克典20182}ので今回は不採用とする.
    以上を踏まえて本研究では,五感の中でも力覚(触覚)と視覚を用いて仮想翼からの情報を提示する.

    \subsubsection{力覚を用いた仮想翼からヒトへの情報提示}

        力覚を用いた提示は,
        \figref{How2present_force_applied2wings_eng.pdf}
        のように羽ばたく際に翼の場所ごとに作用する力を,ヒトの体に対応させることで,翼が連動的にしなっている様子を伝える.

        力覚提示の種類として2種類について比較した.
        1つ目は,モータによる振動または押す力を活用し力覚を提示する.
        2つ目は,EMS(神経筋電気刺激療法)という筋肉や運動神経へ電気刺激を与えることで筋収縮を促し,筋肉の増強や萎縮の予防等をする治療法を用いたものである.EMSにより筋肉を収縮させることで,疑似的に重量を知覚させる研究がある\cite{小川剛史2017電気的筋肉刺激が重量知覚に及ぼす影響の分析}.本研究では,EMS機器により筋収縮を起こすことで疑似的に力覚を提示する.
    
    \subsubsection{視覚を用いた仮想翼からヒトへの情報提示}
 
        視覚を用いた提示は,\figref{hmd_vection.png}のようにUnityで作成した映像をHMDに出力することで行う.HMDに出力される映像は,空中を移動している様子と背中から翼が生えている様子である.

        空中を移動している様子の提示について述べる.
        ベクションと呼ばれる,視野の大部分に一様な運動刺激を提示すると刺激の運動方向と反対の方向に体が動いているように感じる錯覚がある\cite{妹尾武治2014ベクションとその周辺の近年の動向}.例として,停車中の電車から動き出す他の電車の視覚情報を受け取ると,観測者側の電車が動いているように感じる現象が挙げられる.
        浮遊感に関する研究で,ベクションによる落下感覚を分析した研究がある\cite{奥川夏輝2017VR空間における視覚刺激によって発生する落下感覚の分析}.
        空中を移動している様子の提示はベクションを用いる.

        背中から翼が生えている様子は,使用者の背中から翼が生えている可のような映像を出力することで再現する.  


\section{筋電計測について}
    生体信号,筋電計測
    ジェスチャ,コントローラ,外骨格
    ・乾式,湿式
    MYO(使い方等?構成,ソフトウェア)

    自作筋電計測装置
    https://invbrain.neuroinf.jp/static/moth/EMG-tool.pdf
    (12/15)





\section{触覚提示について}
    振動と電気にした
    (電気の仕組み)
    デバイスの選択

    電気系(もっと調べる)(場所はどこでも対応可)
EMSの内医療認可を受けたものを低周波治療器と呼ぶ(周波数関係なしに低周波治療器という)
携帯型(例えばこんなの)は周波数?パターンが既にプログラムされているので制御はダルそう?
→安定化電源的なやつ(これとか,これ)
ヒトの抵抗値(1000~3000Ω),30mAで死ねる.→なのになぜ1000V, 100mA出せる?
制御できるのだろうか
周波数について(リンク)
低周波:筋肉運動しやすい
高周波:皮膚抵抗が減る→インナーマッスルまで届く
装置候補
論文:電気的筋肉刺激が重量近くに及ぼす影響の分析
電気刺激装置:Digitimer社の医療用電気刺激装置マルチパスD185
トリガ制御:Arudiono MEGA(シリアル通信でPCから刺激タイミング,刺激時間,周波数の調整)
電極:日本光電社のPALS Electrodes(MODEL 895220)
装置に関しては「電気刺激装置」と検索すると良い(not EMS)


    トランジスタon/off

    HV-F122,125,127



    


\section{視覚提示について}
    ・翼の3Dオブジェクトの準備(選定)

    ・HMDの選定
        - HMDの紹介(psvr, oculus, htc) (PC用,独立型,スマホ用)(情報まとめ)-> 選定理由

    ・環境の選定?(Unity, UnrealEngine, Blender),OS
        - 候補と各ソフトウェアの説明,選定理由

    ・terrain の説明



\section{流体シミュレータについて} 
%流体シミュレータは結局使わなかったけど折角調べてあったのでとりあえず書いとく...
    空気から受ける力をシミュレーションし,その力をヒトへ与えることで翼で羽ばたいて飛ぶ感覚を提示する.空気から受ける力をシミュレートするのに流体シミュレータを用いる.使用する流体シミュレータの候補として以下のソフトウェアが挙げられる.

    
        \begin{itemize}
        \item \href{http://flowsquare.com/jp/}{Flowsquare}
            \begin{itemize}
            \item 開発: Nora Scientific(2009年)
            \item 特徴: 2次元非定常,非反応/反応性,完全圧縮性/非圧縮性流体のシミュレーションソフト 
            \item 対応OS: Windows
            \item 料金: 無料
            % \item 無料(典型的な流体シミュレーションソフトは1ライセンスあたり数10万くらい(参考:\href{https://icfd.co.jp/product/price.html}{株式会社流体力学研究所})
            % \item 専門知識(プログラミング・CAD・メッシュ生成・前処理(初期場ほ生成)・後処理etc)を必要としない.\\
            % -\textgreater ペイントソフトを用いて解析対象の絵を書く,解析したい条件(流体速度)をテキストファイルに入力
            \end{itemize}
            
        \item \href{https://fsp.norasci.com/}{Flowsquare+}
            \begin{itemize}
            \item 開発: Nora Scientific
            \item 特徴:
                \begin{itemize}
                \item Flowsqureの新バージョン.
                \item 3次元の解析に対応
                \item CFD(Computational Fluid Dynamics:数値流体力学)搭載
                \end{itemize}
            \item 対応OS: Windows
            \item 料金: 無料
            % \textless\textless 通常100万以上のコスト
            % \item 以前と同様に専門知識不要
            \end{itemize}
            
        \item \href{https://fastar.chofu.jaxa.jp/}{FaSTAR}
            \begin{itemize}
            \item 開発: JAXA (宇宙航空研究開発機構)
            \item 特徴: 
                \begin{itemize}
                \item Fast Unstructuired CFD Code
                \item 高速非構造格子(任意の形状のメッシュ)に対応した圧縮性流体解析ソルバー
                \item 航空機や宇宙器などの空力解析に適する
                \end{itemize}
            \item 料金: 授業等の教育目的に限り無償で提供
            \end{itemize}
            
        \item \href{https://altairhyperworks.jp/product/ultrafluidx}{ultraFluidX}
            \begin{itemize}
            \item GPUが必要 (というかサーバーが1基必要...)
            \end{itemize}
        
        \item \href{https://www.openfoam.com/}{OpenFOAM}
        
        \item \href{http://www.ciss.iis.u-tokyo.ac.jp/dl/}{FrontFlow/blue}
            \begin{itemize}
            \item 国産
            \item blue: 乱流音場用,  red: 乱流燃焼用
            \end{itemize}
        
        \item \href{http://www.cenav.org/kdb/?page_id=328}{FrontFlow/violet Cartesian}
            \begin{itemize}
            \item 直交格子を用いた実用複雑系流体解析プログラム
            \end{itemize}
        
        \item \href{http://www.cenav.org/kdb/?page_id=334}{FrontWorkBench}
            \begin{itemize}
            \item 流体・構造・音響錬成解析の自動設定
            \end{itemize}
        
        \item \href{https://www.blender.org/download/}{Blender}
            \begin{itemize}
            \item コンピュータグラフィックスソフトで有名
            \item Unityでも流体解析はできる
            \end{itemize}

        \item \href{https://fenicsproject.org/}{FEniCS}
            \begin{itemize}
            \item pythonやC++で開発可能
            \item 英語
            \end{itemize}
        
        \end{itemize}

        手持ちのノートPCのスペックで使用可能(コロナで在宅な為),無償,3次元シミュレーションが出来る,という観点から今回はFlowSqure+を使用する.((美術)解剖学的には人間の形を保ったまま,背中から生えた翼でバランスよく飛翔することは困難であるので,現実的にはあまり意味はない解析である(\href{https://genkosha.pictures/illustration/18103116710}{小田隆 PICTURES 美しい美術解剖図 第2回 人体に翼を生やすことは可能か?キューピッドを美術解剖図で考察する})).
        
\section{おわりに}
\chapter[直列弾性要素を有するモータ1自由度跳躍ロボットの動作実験]%
        {直列弾性要素を有するモータ1自由度跳躍ロボットの動作実験}
    \section{はじめに}
    \section{跳躍動作実験}
    \section{着地姿勢制御実験}
    \section{おわりに}
\chapter[主観評価実験を踏まえた位置による身体像拡張の差異を評価する被験者実験]%
        {主観評価実験を踏まえた\\位置による身体像拡張の差異を評価する\\被験者実験}

\section{はじめに}
        本章では,主観評価実験を踏まえた位置による身体像拡張の差異を評価する被験者実験より,筋電計測位置と羽ばたく感覚の提示位置を変化させた場合の,羽ばたいて飛ぶ感覚の感じ方の違いについて検証を行う.

\section{実験の目的}
        第4章より,筋電計測位置と触覚提示位置によって羽ばたいて飛ぶ感覚の感じ方に違いが生じることが分かった.そこで,筋電計測位置と触覚提示位置による感じ方の違い(身体像拡張の度合)について,被験者実験より検証を行う.
        
\section{人研究倫理審査}
        % 倫理審査,コロナ関係は1段落で簡潔に触れる程度とする
        本実験は「東京農工大学 人を対象とする研究に関する倫理審査委員会の倫理審査」を通過しており,実験は被験者の同意を得て行う(「翼で飛ぶ感覚を提示するVR(仮想現実)システムに関する研究」,倫理審査委員会承認番号 210908-0343).
        被験者の募集は学内メーリングリスト, 掲示, アルバイト募集用WEBサイトなどを利用して行う. 被験者の選定方針に関しては,胴体部(胸・腹・背中)に電極を貼る都合上,男性に限定する.ただし, 未成年の場合には保護者の承諾を取ることとする. 

        また,被験者に生じるリスクとしては,実験中に発生するVR酔いや新型コロナウイルス感染症への感染がある.これらのリスクは,感染症予防対策を十分に行い,被験者が体調に違和感を感じたらすぐに対応することで対策をする.

        被験者が回答したアンケートは,研究実施者以外アクセスできないようにし,保管期間(5年間)が過ぎたらシュレッダーに掛けて解読不能にして廃棄する.
        
        % 被験者実験の様子(予定)の図があるともっとわかりやすいかも

\section{被験者実験を行う実験環境}
        % システムと装置の説明

        \fig{subjectexp-Experiment_equipment_system.pdf}{width=1\hsize}{Experimental environment system}

        % \fig{subjectexp-env.png}{width=1\hsize}{(写真だとわかりずらいので抽象的な図とか)Subjectexperiment environment}
% 実験装置の具体的な接続に関しての図が欲しい

% 実験の様子を表した図や写真も欲しい

        \fig{hmd-vive_pro_eye.jpg}{width=0.7\hsize}{HTC VIVE PRO EYE}
        \fig{VirtualWingsV2.png}{width=0.7\hsize}{Virtual Wings}

       

        被験者実験を行う実験環境について述べる.被験者実験では,第4章の主観評価実験と同様に\figref{subjectexp-Experiment_equipment_system.pdf}のような環境のシステムで実験を行う.

        筋電計測装置にはMyoWare(\figref{MyoWare.pdf})を使用し,Arduinoを用いて筋電位の値を取得する.
        取得した筋電位の値を端末上のソフトウェア(Unity)へ送る.そして,Unityから視覚提示装置と触覚提示装置を制御する.

        視覚提示装置にはVIVE PRO EYE(HTC社)\figref{hmd-vive_pro_eye.jpg}\cite{htc-vive}を用いる.VIVE PRO EYEを用いて,仮想翼(\figref{VirtualWingsV2.png})が生えている様子を,1人称視点で提示する.


        \fig{subjectexp-haptic_dev_motor.png}{width=0.6\hsize}{(写真だけだと何もわからないので回路図か何かを併記するべき)Haptics display(Vibration)}

        \fig{subjectexp-haptic_dev_ems.png}{width=0.6\hsize}{(写真だけだと何もわからないので回路図か何かを併記するべき)Haptics display(EMS)}

        触覚提示装置は,\figref{subjectexp-haptic_dev_motor.png}と\figref{subjectexp-haptic_dev_ems.png}を用いる.\figref{subjectexp-haptic_dev_motor.png}は,偏心モータをArduinoから
        % PWMで
        制御し振動を与える装置である.\figref{subjectexp-haptic_dev_motor.png}は,低周波治療器(Omron HV-F127\cite{Omron-HV-F127})をArduinoで制御し,電気刺激を与える装置である.
        % (修正)触覚提示装置に関してもう少し詳しく書いた方が良い(Ex. 通信方式(シリアル),データの処理...)

        % 操縦方法について(物理環境→操作方法→提示)
        実験の物理環境について述べる.
        本実験でのVR空間では,飛行を行いやすくするために重力加速度$g^{\prime}$を月と同等(\equref{gravity})に設定する.
        \begin{eqnarray}
                \equlabel{gravity}
                g^{\prime}=1.62\;[m/s^{2}]
        \end{eqnarray}

        また,飛行の際に\equref{air_resistance}のような,速度に比例した空気抵抗$R\;[N]$を与える.
        \begin{eqnarray}
                \equlabel{air_resistance}
                \bm{R}=k\bm{v} \;[N]
        \end{eqnarray}
        $k$は比例定数($k=5.0$).
        % 進行方向,上昇方向どちらも比例定数は等しい

        \fig{Movement_of_VirtualWingsV2.pdf}{width=0.7\hsize}{Movement of Virtual Wings}

        VR空間内での具体的な飛行方法について述べる.
        まず,翼の操作方法は,操作位置(筋電計測位置)を力ませている間は仮想翼が内側に羽ばたき,弛緩させると仮想翼を広げる用に設計する(\figref{Movement_of_VirtualWingsV2.pdf}).
        % 仮想翼が操作位置を力ませ
        仮想翼を羽ばたかせることにより,進行・上昇方向へ力が発生し飛行する(波状飛行\cite{bird-flying})ことができる.この際発生する力$F\;[N]$は\equref{force}に従う.
        % この式についてのrefをしっかりする(空機抵抗の終端速度の式を参考にしたが,もっと理屈的に説明できるように)
        \begin{eqnarray}
                \equlabel{force}
                \bm{F}=\frac{a}{\bm{l}}  \tanh\left(\,\frac{x}{a}\,\right) \;[N]
        \end{eqnarray}
        \begin{eqnarray}
                \equlabel{force_prop}
                \bm{l}=\begin{pmatrix}400 \\ 500 \end{pmatrix}
        \end{eqnarray}
        $l\;$は比例定数(1行目:進行方向,2行目:上昇方向).$a\;$[N]は計測可能な筋電位の最大値($a=1024$),$x\;$[N]は計測された筋電位.

        また,仮想翼を内側に羽ばたかせている際に,触覚提示を行う.この時の触覚提示の強さ$P\;[N]$は計測された筋電位に応じて\equref{haptics}の変化するようにする.
        % ウェーバー・フェヒナーの法則\equref{weber_fechner}\cite{10011340169}より,人間の感覚の大きさは受ける刺激の強さの対数に比例することが知られている.
        % \begin{eqnarray}
        %         \equlabel{weber_fechner}
        %         P = k \log{e}\frac{I}{I_{0}}
        % \end{eqnarray}
        % $P:\;$感覚の強さ(Perception),$I:\;$刺激の強さ(Intensity of stimulation),$I_{0}:\;$感覚の強さが0になる刺激の強さ,$k:\;$刺激固有の定数.
        % 従って,触覚提示の強さを対数に比例するように\equref{haptics}に従うようにする.
        
        \begin{eqnarray}
                \equlabel{haptics}
                P=a \tanh\left(\,\frac{x}{a}\,\right) \;[N]
        \end{eqnarray}
        
\section{実験方法}
        筋電計測位置3種と触覚提示位置4種と触覚提示装置2種の比較を行う.以下に各項目について示す.

        \begin{itemize}
        \item 筋電計測位置
                \begin{itemize}
                \item 上腕二頭筋(動的収縮)
                \item 大胸筋(静的収縮)
                \item 僧帽筋(静的収縮)
                \end{itemize}
        \item 触覚提示位置
                \begin{itemize}
                \item 胸
                \item 腹
                \item 腰
                \item 背中
                \end{itemize}
        \item 触覚提示種類
                \begin{itemize}
                \item 振動
                % (\figref{subjectexp-haptic_dev_motor.png})
                \item 電気刺激
                % (\figref{subjectexp-haptic_dev_ems.png})
                \end{itemize}
        \end{itemize}

        実験の手順を示す.
        \begin{itemize}
        \item まず,触覚提示として振動を用いた装置を使用する.筋電計測位置を上腕二頭筋(一番力み動作が容易な部位)に固定し,触覚提示位置を変化させ比較を行う.
        \item 触覚提示位置を,先ほどの一番評価が高い部位に固定し,筋電計測位置を変化させ比較を行う.
        \item 次に,触覚提示として電気刺激を用いた装置を使用する.筋電計測位置を一番評価が高かった部位に固定し,触覚提示位置を変化させ比較を行う.
        \item 触覚提示位置を一番評価が高かった部位に固定し,筋電計測位置を変化させ比較を行う.
        \item 最後に,アンケートに回答してもらう.
        \end{itemize}


        アンケートには9段階のリッカート尺度\cite{lickert1932method}を用いて回答してもらう.以下にアンケートの内容を示す.
        \begin{itemize}
        \item 触覚提示(胸・腹・腰・背中)の中で,羽ばたいて飛んでいる感覚が強かった順番とそれぞれの評価を記述してください.
        \item 筋電計測位置(腕・胸・肩)の中で,羽ばたいて飛んでいる感覚が強かった順番とそれぞれの評価を記述してください.\\
        (上記2つの質問を触覚提示として振動と電気刺激を用いた場合の2回行う.)
        \item  触覚提示として振動と電気刺激どちらの方が羽ばたいて飛んでいる感覚が強かったか.
        % またそれぞれの評価を記述してください.
        \item 筋電計測位置と触覚提示位置のどちらの方が,羽ばたいて飛ぶ感覚の提示において重要だと感じたか.
        \item 背中から生えた翼で,羽ばたいて飛ぶ感覚を感じることが出来たか.
        \end{itemize}


\section{実験結果と考察}
        % 視覚的な結果
        \fig{subjectexp-result_haptics_pos.png}{width=0.8\hsize}{Comparison of haptics display positions}

        \fig{subjectexp-result_emg_pos.png}{width=0.8\hsize}{Comparison of EMG mesuring position}

        \fig{subjectexp-result_vib_or_ems.png}{width=0.8\hsize}{Comparison of Vibration and EMS}

        \fig{subjectexp-result_mesure_or_presen.png}{width=0.8\hsize}{Comparison of mesuring position and haptcs presentation position}

        % 数値的な結果
        % \begin{table}[tb]
        %         \tablabel{subexp-bodyimage_expansion}
        %         \begin{center}
        %           \caption{Results of subject experiment}
        %           \begin{tabular}{l|c|c|c|c}
        %                 \hline
        %                 Questionnaire & Evaluated value \\
        %                 \hline
        %                 Haptic presentation position(Vibration) & Chest & Abs & Waist & Back\\


        %                 Average of Evaluated Value [-] & 7.4\\
        %                 \hline
        %                 \\
        %                 \hline
        %           \end{tabular}
        %         \end{center}
        %       \end{table}

        \figref{subjectexp-result_haptics_pos.png},\figref{subjectexp-result_emg_pos.png},\figref{subjectexp-result_vib_or_ems.png},\figref{subjectexp-result_mesure_or_presen.png}に被験者実験のアンケート結果を示す.被験者は15人で,20から24歳の男性である.

        \figref{subjectexp-result_haptics_pos.png}は,筋電計測位置を固定し触覚提示位置のを比較した際の評価の平均値を表した図である.
        図より,提示機器が振動・電気刺激に関わらず背中,腰,胸,腹の順番に評価が高いことが分かる.これより触覚の提示位置は,視覚提示で与えた仮想翼の位置からの絶対的な距離よりも,胴体の前面・後面が身体像拡張
        % (道具の身体化)
        の評価に影響することがわかる.

        \figref{subjectexp-result_emg_pos.png}は,触覚提示位置を固定し筋電計測位置の比較した際の評価の平均値である.
        図より,提示機器が振動・電気刺激に関わらず,肩による仮想翼の操作の評価が一番高いことが確認できる.また,筋肉の位置が近い腕(動的筋収縮)と胸(静的筋収縮)の評価には大きな違いが見られない.これより身体像拡張において,ジェスチャ(動的筋収縮)による操作と関節動作を伴わない力み(静的筋収縮)による操作によって惹起される感覚に大きな違いが生じないと考えられる.そして,腕が直感的動作であるなジェスチャであるのに対し,胸の力みは普段行わない動作且つ行いにくい動作であるのにも関わらず評価の差異が少ない.これより,
        % 胸による
        力みによる操作の訓練を行うことで,力みによる操作がジェスチャによる操作よりも仮想翼の身体像拡張において有効になることが期待できる.

        振動と電気刺激の触覚提示の比較を\figref{subjectexp-result_vib_or_ems.png}に示す.7人の被験者が触覚提示として振動を用いた方が良いと感じ,残りの8人の被験者が電気刺激を用いた方が良いと回答した.これより,触覚提示として一般的な振動に加え,電気刺激を用いた提示方法も有用であることが分かる.また,\figref{subjectexp-result_emg_pos.png},\figref{subjectexp-result_haptics_pos.png}より,全体的に電気刺激を用いた提示の方が評価が高いことが分かる.これは,振動を用いた触覚提示が皮膚表面(表在感覚)に対しての刺激であるのに対し,電気刺激を用いた触覚提示は皮膚表面に加えて筋肉(深部感覚)へも刺激が伝わり易く,提示される感覚のモードが増えたことが要因と考えられる.

        \figref{subjectexp-result_mesure_or_presen.png}は,四肢から独立した翼で羽ばたいて飛ぶ感覚の提示について筋電計測位置と触覚提示位置のどちらが重要であるかという質問の結果である.図より,振動・電気刺激どちらの場合でも筋電計測位置が重要であると答える被験者が多かった.
        触覚提示位置が翼からの反応であることに対し,筋電計測位置はヒトから働きかける情報を取得する部分である.
        これより,被験者が自分で翼を動かしている感覚,自己主体感を重視していると捉えることができる.
        第2章では,RHIの研究例\cite{armel2003projecting}で,身体像拡張において触覚提示位置の空間的一致は柔軟であると述べた.しかし,実験結果より筋電計測位置(力みによる操作位置)に関しては,身体像拡張において空間的一致が重要であると考えられる.

        % 身体像拡張において,計測位置と触覚提示位置のどちらが重要であるかと感じたか,という質問では振動・電気刺激問わず計測位置を選択した被験者が多いことが分かる(\figref{subjectexp-result_mesure_or_presen.png}).慣れない力みでの仮想翼の操作にも関わらず測定位置の方が重視されていることより,力み動作を訓練することでより強く仮想翼の身体像拡張が行えることが期待できる.

        最後に,背中から生えた翼で羽ばたいて飛ぶ感覚を感じることが出来たかという項目には,平均で7.11の評価を得ることができた.従って,本研究で提案した手法での仮想翼の身体像拡張が可能であることが分かった.


\section{おわりに}
        本章では,主観評価実験を踏まえた位置による身体像拡張の差異を評価する被験者実験を行った.筋電計測位置と羽ばたく感覚の提示位置を変化させた場合の,羽ばたいて飛ぶ感覚の感じ方の違いについて検証を行った.そして,触覚提示位置が,絶対的な距離よりも胴体の前面・後面の要素が重要となることが分かった.また,筋電計測位置について,ジェスチャに加え力みによる操作も有効であること,訓練次第でジェスチャよりも力みによる操作の方が評価が高くなる可能性があることを示した.そして,触覚として電気刺激を用いた提示の方が全体的に評価が高くなることが分かった.そして,本研究で提案した手法での仮想翼の身体像拡張が可能であることが分かった.


\chapter[結論および今後の展望]%
        {結論および今後の展望}

\section{結論}
    本稿では,翼を動かして飛ぶ感覚を与える研究に注目し,四肢を用いず翼を操作している感覚の提示方法と,VR空間で翼に作用する力をヒトに伝達する手法を提案した.
    実験装置のシステムを作成し,振動とEMS装置による力覚提示についての有用性についての実験を行った.実験より主観ではあるが,力覚提示として振動とEMS装置を用いることの有用性を確認し,ヒトに本来備わっていない部位である翼の存在を感じ,それを操作している感覚を得た.
    %そして,被験者実験のために倫理審査を行い実験の方法と環境について検討をした.
  
    今後の展望として,ヒトによる没入感を調査するために被験者実験を行う.そして,筋電計測位置・力覚提示位置を変化させた場合の没入感の違いについて検証する.その結果から得られる最も評価が高い筋電計測位置・力覚提示位置より,没入感を向上させる.また,デバイスからヒトへの提示情報として前庭電気刺激による加速度感覚\cite{maeda2005shaking}\cite{青山一真2014前庭電気刺激における逆方向不感電流を用いた加速度感覚の増強}の追加し,さらに飛行体験の没入感を高めることも検討している.
  
\section{今後の展望}
\chapter[その他]%
    {その他}

\section{鳥の飛ぶ仕組み}
    \subsection{羽の仕組み}
        \fig{bird-how_to_fly.png}{width=1\hsize}{Wing Quill Mechanism}
        \fig{bird-wing_types.png}{width=1\hsize}{Wing Types}
        
        鳥が空を飛べるのは,翼に「風切羽」という飛ぶための羽が付いているからである.風切羽は翼を持ち上げるときには空気を通すように縦になり(疎になる),下すときには空気を通さないように横に倒れる(密になる)(\figref{bird-how_to_fly.png}).これにより,翼を下すときにのみ力が発生し空を飛ぶことが出来る.

        鳥の羽には種類(\figref{bird-wing_types.png})があり,それぞれ役割が違う.以下に代表的な鳥の羽の種類と役割を示す.

        \begin{itemize}
        \item 初列風切羽\\
            \quad ...進むための羽.羽軸が進行方向にカーブし,左右の幅が違う.これにより羽ばたいた際に,進行方向逆側へ風が生まれ前に進むことができる.

        \item 次列風切羽\\
            \quad 浮かぶための羽.初列風切よりも短く,太く,羽軸がカーブしており左右の幅はほぼ等しい.羽ばたくと下へ向かって風が生まれ,上昇することができる.

        \item 三列風切羽\\
            \quad ...翼と体の間を埋める羽.他の風切羽よりも短い. 翼をたたむと風切羽は重なり合い,小さく折りたたまれる.
            
        \item 尾羽\\
            \quad ...空中でのブレーキや方向転換を行うための羽.次列風切羽と似た形状で,比較的羽軸が真っすぐな羽が多い.
        \end{itemize}

        他にも飛ぶための仕組みとして,発達した胸筋(鳩胸),軽量化のため骨が空洞,短い腸(食事は直ぐ消化し水分と一緒に分とつぃて放出),顎が無いといったことが挙げられる.


        % 参考文献
        % \href{https://global.canon/ja/environment/bird-branch/bird-column/kids2/}{Canon Global 鳥はなぜ飛べるの}
        % \href{https://k-tac.jp/about_feather/}{Kamatac 羽の専門店}

    \subsection{飛び方}
        \begin{itemize}
        \item 直線飛行
        \item 波状飛行
        \item ホバリング
        \item 滑空
        \item はん翔...翼を広げた滑空の姿勢のまま,上昇気流に乗って飛び上がる方法.
        \end{itemize}

        本研究では上記のうち,hogehogeを対象とした飛び方を行っている.


        \section{流体シミュレータについて} 
        %流体シミュレータは結局使わなかったけど折角調べてあったのでとりあえず書いとく...
            空気から受ける力をシミュレーションし,その力をヒトへ与えることで翼で羽ばたいて飛ぶ感覚を提示する.空気から受ける力をシミュレートするのに流体シミュレータを用いる.使用する流体シミュレータの候補として以下のソフトウェアが挙げられる.
        
            
                \begin{itemize}
                \item \href{http://flowsquare.com/jp/}{Flowsquare}
                    \begin{itemize}
                    \item 開発: Nora Scientific(2009年)
                    \item 特徴: 2次元非定常,非反応/反応性,完全圧縮性/非圧縮性流体のシミュレーションソフト 
                    \item 対応OS: Windows
                    \item 料金: 無料
                    % \item 無料(典型的な流体シミュレーションソフトは1ライセンスあたり数10万くらい(参考:\href{https://icfd.co.jp/product/price.html}{株式会社流体力学研究所})
                    % \item 専門知識(プログラミング・CAD・メッシュ生成・前処理(初期場ほ生成)・後処理etc)を必要としない.\\
                    % -\textgreater ペイントソフトを用いて解析対象の絵を書く,解析したい条件(流体速度)をテキストファイルに入力
                    \end{itemize}
                    
                \item \href{https://fsp.norasci.com/}{Flowsquare+}
                    \begin{itemize}
                    \item 開発: Nora Scientific
                    \item 特徴:
                        \begin{itemize}
                        \item Flowsqureの新バージョン.
                        \item 3次元の解析に対応
                        \item CFD(Computational Fluid Dynamics:数値流体力学)搭載
                        \end{itemize}
                    \item 対応OS: Windows
                    \item 料金: 無料
                    % \textless\textless 通常100万以上のコスト
                    % \item 以前と同様に専門知識不要
                    \end{itemize}
                    
                \item \href{https://fastar.chofu.jaxa.jp/}{FaSTAR}
                    \begin{itemize}
                    \item 開発: JAXA (宇宙航空研究開発機構)
                    \item 特徴: 
                        \begin{itemize}
                        \item Fast Unstructuired CFD Code
                        \item 高速非構造格子(任意の形状のメッシュ)に対応した圧縮性流体解析ソルバー
                        \item 航空機や宇宙器などの空力解析に適する
                        \end{itemize}
                    \item 料金: 授業等の教育目的に限り無償で提供
                    \end{itemize}
                    
                \item \href{https://altairhyperworks.jp/product/ultrafluidx}{ultraFluidX}
                    \begin{itemize}
                    \item GPUが必要 (というかサーバーが1基必要...)
                    \end{itemize}
                
                \item \href{https://www.openfoam.com/}{OpenFOAM}
                
                \item \href{http://www.ciss.iis.u-tokyo.ac.jp/dl/}{FrontFlow/blue}
                    \begin{itemize}
                    \item 国産
                    \item blue: 乱流音場用,  red: 乱流燃焼用
                    \end{itemize}
                
                \item \href{http://www.cenav.org/kdb/?page_id=328}{FrontFlow/violet Cartesian}
                    \begin{itemize}
                    \item 直交格子を用いた実用複雑系流体解析プログラム
                    \end{itemize}
                
                \item \href{http://www.cenav.org/kdb/?page_id=334}{FrontWorkBench}
                    \begin{itemize}
                    \item 流体・構造・音響錬成解析の自動設定
                    \end{itemize}
                
                \item \href{https://www.blender.org/download/}{Blender}
                    \begin{itemize}
                    \item コンピュータグラフィックスソフトで有名
                    \item Unityでも流体解析はできる
                    \end{itemize}
        
                \item \href{https://fenicsproject.org/}{FEniCS}
                    \begin{itemize}
                    \item pythonやC++で開発可能
                    \item 英語
                    \end{itemize}
                
                \end{itemize}
        
                手持ちのノートPCのスペックで使用可能(コロナで在宅な為),無償,3次元シミュレーションが出来る,という観点から今回はFlowSqure+を使用する.((美術)解剖学的には人間の形を保ったまま,背中から生えた翼でバランスよく飛翔することは困難であるので,現実的にはあまり意味はない解析である(\href{https://genkosha.pictures/illustration/18103116710}{小田隆 PICTURES 美しい美術解剖図 第2回 人体に翼を生やすことは可能か?キューピッドを美術解剖図で考察する})).

% \include{fig_tab_equ}
\addcontentsline{toc}{chapter}{謝辞}
\markboth{謝辞}{謝辞}
% \chapter*{謝辞}
 修士論文を執筆するに当たり, 指導教官のた東京農工大学工学部機械システム工学科 水内郁夫 教授からは多大なるご指導・ご鞭撻を賜りました.
 研究室に配属して間もない頃,研究は疎かパソコン等の研究に使用するツールに関しても何もわからない自分に水内先生が仰った「ユーザになるな」という言葉は今でも印象に残っています.
 他にも物事を論理的に考え説明することなど,
 数え切れなほどの多くのことを,この3年間で学ばせて頂きました.
 深く感謝の意を申し上げると共に深くお礼を申し上げます.

 また,研究室の先輩・後輩方には
 研究活動に疎い執筆者に多くの助言をくださりました.
 厚くお礼を申し上げ,感謝する次第です.
 そして,研究活動中の相談や雑談をしてくれた同輩である
 西くん,横山くんに感謝します.
 
 
 
 
% \begin{verbatim}
% %% \addcontentsline{toc}{chapter}{謝辞}
% %% \markboth{謝辞}{謝辞}
% %% \chapter*{謝辞}
 修士論文を執筆するに当たり, 指導教官のた東京農工大学工学部機械システム工学科 水内郁夫 教授からは多大なるご指導・ご鞭撻を賜りました.
 研究室に配属して間もない頃,研究は疎かパソコン等の研究に使用するツールに関しても何もわからない自分に水内先生が仰った「ユーザになるな」という言葉は今でも印象に残っています.
 他にも物事を論理的に考え説明することなど,
 数え切れなほどの多くのことを,この3年間で学ばせて頂きました.
 深く感謝の意を申し上げると共に深くお礼を申し上げます.

 また,研究室の先輩・後輩方には
 研究活動に疎い執筆者に多くの助言をくださりました.
 厚くお礼を申し上げ,感謝する次第です.
 そして,研究活動中の相談や雑談をしてくれた同輩である
 西くん,横山くんに感謝します.
 
 
 
 
% \begin{verbatim}
% %% \addcontentsline{toc}{chapter}{謝辞}
% %% \markboth{謝辞}{謝辞}
% %% \chapter*{謝辞}
 修士論文を執筆するに当たり, 指導教官のた東京農工大学工学部機械システム工学科 水内郁夫 教授からは多大なるご指導・ご鞭撻を賜りました.
 研究室に配属して間もない頃,研究は疎かパソコン等の研究に使用するツールに関しても何もわからない自分に水内先生が仰った「ユーザになるな」という言葉は今でも印象に残っています.
 他にも物事を論理的に考え説明することなど,
 数え切れなほどの多くのことを,この3年間で学ばせて頂きました.
 深く感謝の意を申し上げると共に深くお礼を申し上げます.

 また,研究室の先輩・後輩方には
 研究活動に疎い執筆者に多くの助言をくださりました.
 厚くお礼を申し上げ,感謝する次第です.
 そして,研究活動中の相談や雑談をしてくれた同輩である
 西くん,横山くんに感謝します.
 
 
 
 
% \begin{verbatim}
% %% \addcontentsline{toc}{chapter}{謝辞}
% %% \markboth{謝辞}{謝辞}
% %% \include{thanks}
% \end{verbatim}
 
% emacs の人は、M-x comment-region ですね。
% コメント解除は、C-u M-x comment-region ですね。
% \end{verbatim}
 
% emacs の人は、M-x comment-region ですね。
% コメント解除は、C-u M-x comment-region ですね。
% \end{verbatim}
 
% emacs の人は、M-x comment-region ですね。
% コメント解除は、C-u M-x comment-region ですね。

\addcontentsline{toc}{chapter}{参考文献}
\markboth{参考文献}{参考文献}
\bibliographystyle{junsrt}
\bibliography{reference}

\end{document}


