\chapter[提案手法を用いた身体像拡張の主観評価実験]%
        {提案手法を用いた\\身体像拡張の主観評価実験}

\section{はじめに}
    本章では,前章に提案した身体像拡張の手法を用いて主観評価実験を行う.操作/提示方法の検討,操作/提示位置の検討を行い,それぞれの組み合わせの評価を下す.主観評価実験の結果を踏まえ,被験者実験(客観評価実験)で比較する対象について述べる.

\section{主観評価実験を行う実験環境}
    \fig{Experiment_equipment_system_eng.pdf}{width=1\hsize}{Experiment equipment system}
    % --図の修正--
    % Virtual Wingsではなく,Visual display
    % Force presentation dev -> Haptic display

    主観評価実験を行う実験環境について述べる.本章では,\figref{Experiment_equipment_system_eng.pdf}のような,筋電計測装置で計測した値を,端末上のソフトウェア(Unity\footnote{Unity Technology Inc.が開発したゲームエンジン及びゲームの統合開発環境.2005年に配信され,ポケモンGO(任天堂)やFall Guys(Mediatonic)といった様々なゲームの開発に用いられている.})へ送信し,Unity上から視覚提示装置と触覚提示装置を動作させるシステムを用いる.

    \fig{myo_armband.pdf}{width=1\hsize}{Myo(Thalmic Labs社)}
    \fig{MyoWare.pdf}{width=1\hsize}{MyoWare(Advancer Technologies LLC)}
    まず,筋電計測装置として,ジェスチャと力みによる操作を比較するためにMyo(\figref{myo_armband.pdf})\cite{thalmiclabs}とMyoWare(\figref{MyoWare.pdf})\cite{advancertechnologies}の2つを用意する.MyoはThalmic Labsが開発した,筋電センサを搭載したマルチジェスチャバンドであり,上腕部の筋肉位から,腕・手首・指の動きのジェスチャを識別することが可能な乾式筋電センサである.MyoWareはAdvancer Technologiesが提供する湿式筋電センサであり,Myoと異なり任意の筋肉の筋電位を計測することができ,Arduino等の外部接続したマイコンで簡単に筋電位を読み取ることが可能である.

    \fig{EMG_device_HV-F122.pdf}{width=1\hsize}{HV-F122(OMRON Corporation)}
    次に,ヒトから仮想翼への情報伝達の触覚提示については,振動での提示機器とEMSを用いた電気刺激での提示機器を用意する.それぞれ,振動での触覚提示はMyoの振動機能,電気刺激での触覚提示は低周波治療器Omron HV-F122(\figref{EMG_device_HV-F122.pdf})\cite{Omron-HV-F122}を用いる.

    \fig{hmd-oculus_quest.jpg}{width=1\hsize}{Meta Quest(Meta Platforms Inc.)}
    \fig{virtualwingborn.png}{width=1\hsize}{Virtual Wings skeleton on one side}
    そして視覚提示装置は,液晶モニターでの視覚提示とHMDを用いた視覚提示の2種類を用紙する.液晶モニターはGW2765HT(BenQ),HMDはMeta社のMeta Quest(\figref{hmd-oculus_quest.jpg})\cite{OculusQuest}を使用する.視覚提示する際に使用する仮想翼を\figref{virtualwingborn.png}に示す.

\section{操作/提示方法の検討}
    \subsection{翼の操作方法の比較}
        まず,翼の操作方法について比較し,主観評価を行う.第3章で述べたように,本研究ではヒトから翼への情報伝達として筋電位を用いる.筋電位を用いた操作方法は,関節動作を伴う動きである「ジェスチャ」と関節動作を伴わない「力み」による操作に分類することが出来る.

        触覚と視覚提示の条件を固定し,ジェスチャと力みによる操作を比較する. 触覚はMyoを用いた振動を前腕に提示,視覚は仮想翼の3人称視点を液晶ディスプレイに描画し提示する.

        \fig{Manipulation_of_VirtualWings_using_Myo.pdf}{width=1\hsize}{Manipulation of virtual wings skeleton using Myo}
        \fig{Manipulation_of_VirtualWings_using_MyoWare.pdf}{width=1\hsize}{Manipulation of virtual wings skeleton using MyoWare}
        仮想翼の操作は,筋電計測装置としてMyo用いる場合は\figref{Manipulation_of_VirtualWings_using_Myo.pdf}のように,手首を内側に曲げると翼も内側に羽ばたき,手首を外側に開くと翼も外側へ開くように設計する.また,触覚提示は翼が内側に羽ばたく際に合わせて前腕に装着したMyoが振動するように行う.
        
        MyoWareを用いる場合は,力むと翼が閉じ,弛緩すると翼が開くように設計する.触覚提示はMyoで筋電計測する場合と同様に,腕に装着したMyoを翼が内側に羽ばたく際に振動させることで提示を行う.また,MyoWareは計測部位として,力み動作が容易な上腕・胸・肩を選択する.

        検証の結果,VRアプリケーションにおいて一般的なジェスチャを用いた操作だけでなく,上腕・胸・肩を問わず力みを用いた仮想翼の操作も問題なく意図通りに動作可能であることを確認した.

    \subsection{触覚提示方法の比較}
        次に,触覚提示方法について比較し,主観評価を行う.第3章で述べたように本研究では,触覚として振動を用いた触覚提示と電気刺激を用いた触覚提示の2つを準備する.

        翼の操作方法と視覚提示の条件を固定し,触覚として振動を用いた提示と電気刺激を用いた提示を比較する.翼の操作方法は上腕・胸・肩の力み,視覚提示は液晶ディスプレイに翼の動きを描画して提示を行う.

        \tabref{exp1_result},\tabref{exp2_result}に,触覚提示として振動と電気刺激を行った際の,実験中の仮想翼からヒトへの情報の主観評価を示す.
        
        % 実験の結果
        \begin{table}[tb]
            \tablabel{exp1_result}
            \begin{center}
                \caption{Results of an experiment using vibration as a haptics presentation\\}
                
                \begin{tabular}{l|c|c|c}
                    \hline
                    Position(Sensing/Haptics) & Arm/Arm & Chest/Arm & Shoulder/Arm\\
                    \hline
                    Sense of having wings & 1 & 1 & 2 \\
                    Sense of meneuvering the wings & 4 & 4 & 4 \\
                    Sense of flying with wings & 1 & 2 & 2 \\
                    Sense of air resistance & 4 & 4 & 4 \\
                    \hline
                \end{tabular}                
            \end{center}
        \end{table}
        
        \begin{table}[t]
            \tablabel{exp2_result}
            \begin{center}
                \caption{Results of experiments using EMS as haptics presentation}
                \begin{tabular}{l|c|c|c}
                    \hline
                    % 筋電取得位置(腕)
                    Position(Sensing/Haptics) & Arm/Arm & Arm/Abs & Arm/Back \\\hline
                    Sense of having wings & 1 & 3 & 5 \\
                    Sense of meneuvering the wings & 3 & 3 & 3\\
                    Sense of flying with wings & 3 & 3 & 4 \\
                    Sense of air resistance & 3 & 4 & 4 \\\hline\hline
    
                    % 筋電取得位置(胸)
                    Position(Sensing/Haptics) & Chest/Arm & Chest/Abs & Chest/Back \\\hline
                    Sense of having wings & 2 & 3 & 5 \\
                    Sense of meneuvering the wings & 3& 3 & 4\\
                    Sense of flying with wings & 3 & 4 & 5 \\                        
                    Sense of air resistance & 3 & 4 & 4 \\\hline\hline
    
                    % 筋電取得位置(腹)
                    Position(Sensing/Haptics) & Shoulder/Arm & Shouler/Abs & Shoulder/Back  \\\hline                        
                    Sense of having wings & 2 & 3 & 5 \\                        
                    Sense of meneuvering the wings & 4 & 4 & 5 \\
                    Sense of flying with wings & 3 & 3 & 5 \\
                    Sense of air resistance & 3 & 4 & 4\\\hline\hline
                \end{tabular}
            \end{center}
        \end{table}

    \subsection{視覚提示方法の比較}
        最後に,視覚提示方法について比較し,主観評価を行う.視覚提示装置として,液晶ディスプレイとHMDを用意する.

        翼の操作方法と触覚提示の条件を固定し,視覚提示装置として液晶ディスプレイとHMDを比較する.翼の操作方法は上腕・胸・肩の力み,触覚提示は前腕・胸・腹・背中にEMS機器による電気刺激を行う.
        % 未

    
\section{操作/提示位置の検討}
    \subsection{操作位置の比較}
    \subsection{提示位置の比較}

\section{未}
    
        MyoとUnity通信\\
        myowareとUnity通信\\
    
    

    力覚提示として振動を用いた実験の環境としては,ヒトから仮想翼の部分(筋電の計測)と仮想翼からヒトへの振動提示の両方を,筋電センサーを搭載したマルチジェスチャーハンドであるMyo(\figref{myo_armband.pdf}(a), Thalmic社)で行った.

    また視覚提示として用いた仮想翼は\figref{virtualwingborn.png}(b)に示すものを用いた.

    

    


    

    実験より主観ではあるが,力覚提示として振動を用いることの有用性,3人称視点での視覚提示の不十分であることを確認した.また,ジェスチャーよる仮想翼の操作は関節動作を伴いことで不要な疲労感を生む.これは飛行体験において翼の操作における障害となり,没入感の妨げになると考えられる.飛行体験において,力みといった関節動作を伴わない筋収縮による仮想翼の操作が有用である.

    \section{力覚提示としてEMSを用いた実験}
    %実験に用いたデバイス
    \fig{MyoWare.pdf}{width=1\hsize}{MyoWare}
    \fig{EMG_device_HV-F122.pdf}{width=1\hsize}{EMG device}
    \fig{VirtualWingsV2.pdf}{width=1\hsize}{Virtual Wings model version 2}

    %実験の様子
    \fig{Movement_of_VirtualWingsV2.pdf}{width=1\hsize}{Virtual presentation of Virtual wings}

    力覚提示としてEMS機器を用いた実験では,筋電計測装置としてMyoWare(\figref{MyoWare.pdf}(a), AdvanceerTechnologies LLC),EMS機器として低周波治療器HV-F122(\figref{EMG_device_HV-F122.pdf}(b), OMRON Corporation)を使用した.視覚提示する仮想翼としては\figref{VirtualWingsV2.pdf}(c)のモデルを\figref{Movement_of_VirtualWingsV2.pdf}のように1人称視点して提示した.

    また実験の際,筋電計測位置を腕と胸,腹,力覚の提示位置を腕,腹,背中の複数個所を別々に計8通り行い,位置ごとの没入感の違いについて確認した.

    翼の操作方法としては,筋電計測箇所の筋肉を力ませると翼が内側へ羽ばたき,弛緩させると翼が外側に開くように設計した.EMS装置による力覚提示は翼が内側へ羽ばたく際に行うものとした.


    \tabref{exp2_result}に,実験中の没入感に関する各項目に対する主観評価を示す.
    
    
    

    \tabref{exp2_result}より,力覚の提示位置が背中に近づくほど没入感に関する評価が高くなっている.
    EMS装置を用いた力覚提示の有効性,没入感において筋電計測位置よりも力覚提示位置の方が重要であることを確認した.
    さらに,本来人間に備わっていない部位である翼の存在を感じ,それを操作している感覚も得られた.

\section{被験者実験}
    \fig{position_of_mesurement.png}{width=1\hsize}{position of mesurement}
    
    % 実験の目的→方法の順番で書くこと
    被験者実験では,筋電計測位置と羽ばたく感覚の提示位置を変化させた場合の没入感の違いについて検証する.
    
    被験者はHMD,筋電計測装置,羽ばたく感覚の力覚提示装置を装着し,仮想翼を操縦する.この際,被験者の筋電のデータを記録する.
    筋電計測装置に関してはMyoWareを使用し体に直接貼り付けて計測を行う.筋電取得位置は関節動作を伴わない静的な筋収縮が容易な部位である胸肩部・腹部・臀部を検討している.
    羽ばたく感覚の力覚提示に関しては,ハプティックスーツ等による振動・押し力,またはEMS機器の筋収縮作用による疑似的な力覚提示によって行う.羽ばたく感覚の提示位置に関しては仮想翼が存在する背中から体の側面を検討している.
    
    その後,操縦中の没入感に関してアンケートを行う.被験者へは筋電取得箇所と力覚提示提示位置ごと没入感の違いについての回答を得る.具体的には以下のような内容を検討している.
    \begin{itemize}
    \item 筋電計測位置別の没入感
    \item 力覚提示位置別の没入感
    \item 振動・押し力提示機器とEMS機器の没入感
    \item 筋電計測位置と力覚提示位置の重要性比較
    \item 一番没入感の高い組み合わせ
    \end{itemize}

    
    % --------以下倫理審査の書類を参考にして作成--------------
    % 倫理審査,コロナ関係は1段落で簡潔に触れる程度とする
    本実験は「東京農工大学 人を対象とする研究に関する倫理審査委員会の倫理審査」を通過しており,実験は被験者の同意を得て行う.
    % 被験者の募集は学内メーリングリスト, 掲示, アルバイト募集用WEBサイトなどを利用して行う. 被験者の選定方針に関しては特に定めない.ただし, 未成年の場合には保護者の承諾を取ることとする. 
    また,被験者に生じるリスクとしては,実験中に発生するVR酔いや新型コロナウイルス感染症への感染がある.これらのリスクは,感染症予防対策を十分に行い,被験者が体調に違和感を感じたらすぐに対応することで対策をする.
    
    % 被験者実験の様子(予定)の図があるともっとわかりやすいかも
    
\section{被験者アンケート}
リッカート尺度,t検定


\section{おわりに}