\chapter[提案手法を用いた身体像拡張の主観評価実験]%
        {提案手法を用いた\\身体像拡張の主観評価実験}

\section{はじめに}
    本章では,前章に提案した身体像拡張の手法を用いて主観評価実験を行う.操作/提示方法の検討,操作/提示位置の検討を行い,それぞれの組み合わせの評価を下す.主観評価実験の結果を踏まえ,被験者実験(客観評価実験)で比較する対象について述べる.

\section{主観評価実験を行う実験環境}
    \fig{Experiment_equipment_system_eng.pdf}{width=1\hsize}{Experiment equipment system}
    % --図の修正--
    % Virtual Wingsではなく,Visual display
    % Force presentation dev -> Haptic display

    主観評価実験を行う実験環境について述べる.本章では,\figref{Experiment_equipment_system_eng.pdf}のような,筋電計測装置で計測した値を,端末上のソフトウェア(Unity\footnote{Unity Technology Inc.が開発したゲームエンジン及びゲームの統合開発環境.2005年に配信され,ポケモンGO(任天堂)やFall Guys(Mediatonic)といった様々なゲームの開発に用いられている.})へ送信し,Unity上から視覚提示装置と触覚提示装置を動作させるシステムを用いる.

    \fig{myo_armband.pdf}{width=1\hsize}{Myo(Thalmic Labs社)}
    \fig{MyoWare.pdf}{width=1\hsize}{MyoWare(Advancer Technologies LLC)}
    まず,筋電計測装置として,ジェスチャと力みによる操作を比較するためにMyo(\figref{myo_armband.pdf})\cite{thalmiclabs}とMyoWare(\figref{MyoWare.pdf})\cite{advancertechnologies}の2つを用意する.MyoはThalmic Labsが開発した,筋電センサを搭載したマルチジェスチャバンドであり,上腕部の筋肉位から,腕・手首・指の動きのジェスチャを識別することが可能な乾式筋電センサである.MyoWareはAdvancer Technologiesが提供する湿式筋電センサであり,Myoと異なり任意の筋肉の筋電位を計測することができ,Arduino等の外部接続したマイコンで簡単に筋電位を読み取ることが可能である.

    \fig{EMG_device_HV-F122.pdf}{width=1\hsize}{HV-F122(OMRON Corporation)}
    次に,ヒトから仮想翼への情報伝達の触覚提示については,振動での提示機器とEMSを用いた電気刺激での提示機器を用意する.それぞれ,振動での触覚提示はMyoの振動機能,電気刺激での触覚提示は低周波治療器Omron HV-F122(\figref{EMG_device_HV-F122.pdf})\cite{Omron-HV-F122}を用いる.

    \fig{hmd-oculus_quest.jpg}{width=1\hsize}{Meta Quest(Meta Platforms Inc.)}
    \fig{virtualwingborn.png}{width=1\hsize}{Virtual Wings skeleton on one side}
    そして視覚提示装置は,液晶モニターでの視覚提示とHMDを用いた視覚提示の2種類を用紙する.液晶モニターはGW2765HT(BenQ),HMDはMeta社のMeta Quest(\figref{hmd-oculus_quest.jpg})\cite{OculusQuest}を使用する.視覚提示する際に使用する仮想翼を\figref{virtualwingborn.png}に示す.

\section{操作/提示方法の検討}
    \subsection{翼の操作方法の比較}
        まず,翼の操作方法について比較し,主観評価を行う.第3章で述べたように,本研究ではヒトから翼への情報伝達として筋電位を用いる.筋電位を用いた操作方法は,関節動作を伴う動きである「ジェスチャ」と関節動作を伴わない「力み」による操作に分類することが出来る.

        触覚と視覚提示の条件を固定し,ジェスチャと力みによる操作を比較する. 触覚はMyoを用いた振動を前腕に提示,視覚は仮想翼の3人称視点を液晶ディスプレイに描画し提示する.

        \fig{Manipulation_of_VirtualWings_using_Myo.pdf}{width=1\hsize}{Manipulation of virtual wings skeleton using Myo}
        \fig{Manipulation_of_VirtualWings_using_MyoWare.pdf}{width=1\hsize}{Manipulation of virtual wings skeleton using MyoWare}
        仮想翼の操作は,筋電計測装置としてMyo用いる場合は\figref{Manipulation_of_VirtualWings_using_Myo.pdf}のように,手首を内側に曲げると翼も内側に羽ばたき,手首を外側に開くと翼も外側へ開くように設計する.また,触覚提示は翼が内側に羽ばたく際に合わせて前腕に装着したMyoが振動するように行う.
        
        MyoWareを用いる場合は,力むと翼が閉じ,弛緩すると翼が開くように設計する.触覚提示はMyoで筋電計測する場合と同様に,腕に装着したMyoを翼が内側に羽ばたく際に振動させることで提示を行う.また,MyoWareは計測部位として,力み動作が容易な上腕・胸・肩を選択する.

        検証の結果,VRアプリケーションにおいて一般的なジェスチャを用いた操作だけでなく,上腕・胸・肩を問わず力みを用いた仮想翼の操作も,操縦者の意図通りに動作させることが可能であることを確認した.また,筋電計測装置を用いたジェスチャの判定にはある程度の力みが必要となり,関節動作を伴わない筋収縮による仮想翼の操作と比べ,疲労感が多くなることが分かった.

    \subsection{触覚提示方法の比較}
        次に,触覚提示方法について比較し,主観評価を行う.第3章で述べたように本研究では,触覚として振動を用いた触覚提示と電気刺激を用いた触覚提示の2つを準備する.

        翼の操作方法と視覚提示の条件を固定し,触覚として振動を用いた提示と電気刺激を用いた提示を比較する.翼の操作方法は上腕・胸・肩の力みにより翼が閉じ,弛緩すると翼が開くように設計する.視覚提示は液晶ディスプレイに3人称視点での翼の動きを描画して提示を行う.

        \tabref{exp1_result},\tabref{exp2_result}に,触覚提示として振動と電気刺激を行った際の,実験中の仮想翼からヒトへの情報の5段階の主観評価を示す.
        
        % 実験の結果
        \begin{table}[tb]
            \tablabel{exp1_result}
            \begin{center}
                \caption{Results of an experiment using vibration as a haptics presentation\\}
                
                \begin{tabular}{l|c|c|c}
                    \hline
                    Position(Sensing/Haptics) & Arm/Arm & Chest/Arm & Shoulder/Arm\\
                    \hline
                    Sense of having wings & 1 & 1 & 2 \\
                    Sense of meneuvering the wings & 4 & 4 & 4 \\
                    Sense of flying with wings & 1 & 2 & 2 \\
                    Sense of air resistance & 4 & 4 & 4 \\
                    \hline
                \end{tabular}                
            \end{center}
        \end{table}
        
        \begin{table}[t]
            \tablabel{exp2_result}
            \begin{center}
                \caption{Results of experiments using EMS as haptics presentation}
                \begin{tabular}{l|c|c|c}
                    \hline
                    % 触覚提示位置(腕)
                    Position(Sensing/Haptics) & Arm/Arm & Chest/Arm & Shoulder/Arm \\
                    \hline
                    Sense of having wings & 1 & 2 & 2 \\
                    Sense of meneuvering the wings & 3 & 3 & 4\\
                    Sense of flying with wings & 2 & 2 & 3 \\
                    Sense of air resistance & 3 & 3 & 3 \\
                    \hline\hline
    
                    % 筋電取得位置(胸)
                    Position(Sensing/Haptics) & Arm/Abs & Chest/Abs & Shoulder/Abs \\
                    \hline
                    Sense of having wings & 3 & 3 & 3 \\
                    Sense of meneuvering the wings & 3 & 3 & 4\\
                    Sense of flying with wings & 3 & 4 & 3 \\                        
                    Sense of air resistance & 4 & 4 & 4 \\
                    \hline\hline
    
                    % 筋電取得位置(腹)
                    Position(Sensing/Haptics) & Arm/Back & Chest/Back & Shoulder/Back  \\
                    \hline                        
                    Sense of having wings & 5 & 5 & 5 \\                        
                    Sense of meneuvering the wings & 3 & 4 & 5 \\
                    Sense of flying with wings & 4 & 5 & 5 \\
                    Sense of air resistance & 4 & 4 & 4\\
                    \hline\hline
                \end{tabular}
            \end{center}
        \end{table}

        \tabref{exp1_result}と\tabref{exp2_result}(1行目:Hapticsの項目がArmの行)より,振動と電気刺激の主観評価に顕著な違いが無いことが分かる.従って,触覚提示として一般的な振動加え,電気刺激による提示も有用であることが言える.

        また\tabref{exp2_result}より,筋電計測位置と触覚提示位置によって主観評価の違いが表れているのが分かる.筋電計測・触覚提示ともに,四肢(腕)よりも胴体のへ行うことで羽ばたいて飛ぶ感覚を強く提示可能と考えられる.
        
        % 電気の有用性,提示位置と計測位置の位置による違い

    \subsection{視覚提示方法の比較}
        最後に,視覚提示方法について比較し,主観評価を行う.視覚提示装置として,液晶ディスプレイとHMDを用意し,それぞれ3人称視点と1人称視点の映像を提示する.1人称視点の映像は,仮想翼を背中に配置し,体をひねって背中側を見ると仮想翼が生えているような映像となっている.

        翼の操作方法と触覚提示の条件を固定し,視覚提示装置として液晶ディスプレイとHMDを比較する.翼の操作方法は上腕・胸・肩の力みにより翼が閉じ,弛緩すると翼が開くように設計する.触覚提示は前腕・胸・腹・背中にEMS機器による電気刺激を行う.

        実験より,3人称視点と比べ1人称視点の映像提示を行った場合の方が,自分の体から翼が生えている様子を強く感じた.3人称視点を提示を行った場合は,自らの翼を操作している感覚よりも,遠隔地の翼を操作している感覚(テレイグジスタンスような感覚)を惹起させた.身体像の拡張においては,3人称視点よりも1人称視点の映像提示の方が有効であることが分かった.
    
\section{操作/提示位置の検討}
    操作/提示方法の検討の結果,操作位置(筋電計測位置)と触覚提示位置によって羽ばたいて飛ぶ感覚の提示に違いが生じる事が分かった.そこで,位置による違いの比較を効率的に行うため,筋電計測・触覚提示位置の候補を挙げ,選定を行う.
    
    \subsection{操作位置の検討}
        \fig{body-muscle.pdf}{width=1\hsize}{Muscle structure\cite{からだと病気のしくみ図鑑}}
        
        関節動作を伴わない筋収縮(静的収縮\cite{thistle1967isokinetic})が容易な部位として,四肢では腕(上腕二頭筋・上腕三頭筋),ふともも
        (大腿直筋),ふくらはぎ(腓腹筋・ヒラメ筋)が挙げられる.胴体部では,胸(大胸筋),腹(腹直筋),肩(僧帽筋),臀部(大殿筋)が挙げられる.筋電計測において,計測点の皮下脂肪が多い場合,筋電位の振幅が減衰し不明瞭となる\cite{白石恵1992筋電位多点計測による体幹背部の神経支配帯の分布}.従って,比較的皮下脂肪が少ない部位を筋電計測位置として選択する必要がある.以上より,筋電計測位置として胸・肩と,ジェスチャと力みによる仮想翼の操作を比較するために,上腕二頭筋の動的収縮の3種類を用いる.

    \subsection{提示位置の検討}
        \tabref{exp2_result}より,触覚提示の位置は四肢よりも胴体部に行った場合の方が評価が高いことが分かった.そこで,胴体の中でどこの部位が触覚提示として一番有効であるかを比較する.ヒトの感覚野の内,手の占める触知覚の割合が大きく\cite{penfield1950cerebral},胴体部の触知覚の割合が少ない\cite{gibson1962observations}\cite{丸本耕次1997触覚表示の認知特性に関する研究}\cite{杉輝夫2005身体部位による触知覚の差}.従って,胴体部の提示部位を細かく分類するのではなく,大きく分類する方がそれぞれの違いを比較できると考える.そこで,胴体を上部・下部,表・裏の4部分に大きく分けて(つまり胸・腹・背中・腰),提示を行う.        

\section{おわりに}
    本章では,前章に提案した身体像拡張の手法を用いて主観評価実験を行った.操作/提示方法の検討,操作/提示位置の検討を行い,それぞれの組み合わせの評価を下した.主観評価実験の結果,操作方法としてジェスチャだけでなく力みを用いた方法も有用であることが分かった.また,仮想翼からの提示として電気刺激を用いた触覚提示も有効であることが分かった.
    そして,操作/提示の位置について候補を挙げ検討を行った.