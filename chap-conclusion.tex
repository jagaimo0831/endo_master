\chapter[結論および今後の展望]%
        {結論および今後の展望}

\section{結論}
    本稿では,翼を動かして飛ぶ感覚を与える研究に注目し,四肢を用いず翼を操作している感覚の提示方法と,VR空間で翼に作用する力をヒトに伝達する手法を提案した.
    実験装置のシステムを作成し,振動とEMS装置による力覚提示についての有用性についての主観評価実験を行った.そして主観評価実験を踏まえ,位置による身体像拡張の差異を評価する被験者実験を行った.
    実験より,触覚提示位置が,絶対的な距離よりも胴体の前面・後面の要素が重要となることが分かった.また,筋電計測位置について,ジェスチャに加え力みによる操作も有効であること,訓練次第でジェスチャよりも力みによる操作の方が評価が高くなる可能性があることを示した.そして,触覚として電気刺激を用いた提示の方が全体的に評価が高くなることが分かった.そして,本研究で提案した手法での仮想翼の身体像拡張が可能であることが分かった.

    
\section{今後の展望}
    今後の展望として,仮想翼からヒトへの提示情報を増やすことが挙げられる.前庭電気刺激による加速度感覚\cite{maeda2005shaking}\cite{青山一真2014前庭電気刺激における逆方向不感電流を用いた加速度感覚の増強}のや,風や音の追加が考えられる.

    また,実験に用いた装置の改良が考えられる.1点で筋電の計測を行っていたが,多点で筋電計測を行うことで,より安定した筋電位を取得することができる\cite{白石恵1992筋電位多点計測による体幹背部の神経支配帯の分布}.触覚提示装置は,提示可能な周波数帯域を広げることで,繊細な触覚提示が可能となる.本稿は筋電計測位置と触覚提示位置をある程度絞って比較を行ったが,これら候補を増やすことで位置による違いをより詳しく知ることができるだろう.

    そして,今回は仮想翼で飛行するだけであったが,飛行しながらの的当てタスクといったことを行い点数を付けることで,四肢から独立した翼での飛行体験を評価することや,身体像の拡張がどの程度成功できているかを客観的に評価できる指標を準備することが求められる.
  
\section{未}
    \begin{itemize}
    \item 長期間使用したときの脳地図の変化(義手をしようすると脳地図に書き込まれる)
    \item 触覚提示の周波数帯域を広げる(より細かい触覚提示)
    \item 触覚提示のデバイスを向上(ハプティックスーツ(振動,電気))
    \item 仮想翼と実翼の比較
    \item 翼の生える位置変更したときの比較
    \item 位置による比較を行った->拡張(翼の根元だけの提示)とリマップ(背中全体への提示)の比較
    \item 客観的に身体像が拡張されたかどうかを確認する手法
    \end{itemize}