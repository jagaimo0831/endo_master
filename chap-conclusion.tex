\chapter[結論および今後の展望]%
        {結論および今後の展望}

\section{結論}
    本稿では,翼を動かして飛ぶ感覚を与える研究に注目し,四肢を用いず翼を操作している感覚の提示方法と,VR空間で翼に作用する力をヒトに伝達する手法を提案した.
    実験装置のシステムを作成し,振動とEMS装置による力覚提示についての有用性についての実験を行った.実験より主観ではあるが,力覚提示として振動とEMS装置を用いることの有用性を確認し,ヒトに本来備わっていない部位である翼の存在を感じ,それを操作している感覚を得た.
    %そして,被験者実験のために倫理審査を行い実験の方法と環境について検討をした.
  
    今後の展望として,ヒトによる没入感を調査するために被験者実験を行う.そして,筋電計測位置・力覚提示位置を変化させた場合の没入感の違いについて検証する.その結果から得られる最も評価が高い筋電計測位置・力覚提示位置より,没入感を向上させる.また,デバイスからヒトへの提示情報として前庭電気刺激による加速度感覚\cite{maeda2005shaking}\cite{青山一真2014前庭電気刺激における逆方向不感電流を用いた加速度感覚の増強}の追加し,さらに飛行体験の没入感を高めることも検討している.
  
\section{今後の展望}