\input{tuatmech-final.tex}
\usepackage{pxjahyper}

\usepackage{jtygm}

\secret{b}{0mm}                 %学外秘/専攻外秘の設定.学部はb(学外秘),修士はm(専攻外秘)にする.
                                %第2引数は位置の調整用.-側に大きくすれば左に寄る.+側に大きくすれば右に寄る.
\newcommand{\FIGDIR}{./fig}	%図を置くディレクトリを指定する
				%Makefileとは連動していないので注意

\usepackage{ikuo}
\pagenum{A-5}%ページ番号 プログラムが確定したら修正を!

\setlength{\headheight}{15pt}

\begin{document}
\twocolumn[%
\title{四肢から独立した翼で羽ばたいて飛ぶ感覚の提示}{Presenting the sensation of flying with flapping virtual wings independent of the limbs}
\author{水内研究室}{遠藤 健}{Ken ENDO}
\begin{abstract}
  % 要旨を英訳すればok
  This paper describes analysis of hoge and the hoge generator.  In previous researches, it was said that hoge is hoge.  In this research, we propose the idea ``hoge is not hoge''.  Based on this idea, we introduced the theory named ``The Second Law of Hoge'' and developed the hoge generator.  We confirmed the effect of the generator by simulations and experiments.  This is a pen.  This is a pen.  This is a pen.  This is a pen.  This is a pen.  This is a pen.  This is a pen.  This is a pen.  This is a pen.  This is a pen.  This is a pen.  This is a pen.  This is a pen.  This is a pen.  This is a pen.  This is a pen.  This is a pen.  This is a pen.  
\end{abstract}
\keyword{Hoge, Piyo, Design Criteria}
]
\begin{small}

\section{緒言}
  \fig{WingMan.png}{width=1\hsize}{Flying with flapping virtual wings independent of the limbs}

  ヒトは古くから空を飛ぶことに憧れを抱いている.
  これまで私たちは,飛行機やハンググライダーといった乗り物を用いることで飛行体験をしてきた.
  また,個人飛行装置\footnote{Portable Parsonal Airmobility System...ジェットパック,動力式ウイングスーツ,動力式パラフォイル(風により展開される柔軟構造を持つ翼.Ex.パラグライダーの翼)}
  のような,ウェアラブルな装置で空を飛ぶ研究も行われている\cite{gravityindustries}.
  % 実際に飛ぶことにはリスクやコストが伴うが,VR装置を使用することで簡単に飛行体験が可能である.
  しかし,実際に空を飛ぶことは墜落などのリスクや燃料といったコスト,機器を操縦するための技術が必要となる.
  VR(Virtual Reality: 仮想現実)システムを使用することで,それらリスクやコストを回避し,乗り物・ウェアラブルな装置を問わず簡単に飛行体験が可能となる.

  \figref{WingMan.png}は,四肢から独立した翼で羽ばたいて飛ぶ様子を示した図である.
  本研究では,VRシステムを用いて\figref{WingMan.png}のように,ヒトの背中から翼が生えた生物になり羽ばたいて飛ぶ感覚の提示手法を提案する.
  
  \subsection{本研究での羽ばたいて飛ぶ感覚の定義}
    本論文では「浮遊感」,「飛ぶ感覚」,「羽ばたいて飛ぶ感覚」を\figref{intro-Classification_of_floating_feeling.pdf}のように位置付ける.
    \fig{intro-Classification_of_floating_feeling.pdf}{width=1\hsize}{Classification of floating feeling}

    \begin{itemize}
            \item 「浮遊感」\\
            ...空中に浮いて漂っている感覚.
            \item 「飛ぶ感覚」\\
            ...「浮遊感」に,空中を移動する感覚を追加した感覚.
            \item 「羽ばたいて飛ぶ感覚」\\
            ...「飛ぶ感覚」に,翼を羽ばたかせる感覚を追加した感覚.
    \end{itemize}

\subsection{研究の背景と目的}
    VR装置を用いた「浮遊感」や「飛ぶ感覚」を与える研究は多く行われてきた.視覚刺激をによって発生する落下感覚に関しての研究\cite{奥川夏輝2017VR空間における視覚刺激によって発生する落下感覚の分析}や身体幇助メカニズムを用いた飛行体験装置の提案\cite{鈴木拓馬2014hmd}等がある.また,飛行しているドローンを上半身のジェスチャーで制御し,ドローンからの映像をHMD(Head Mounted Display: ヘッドマウントディスプレイ)によって与えることで飛ぶ感覚を提示する研究\cite{rognon2018flyjacket}もある.

    \fig{Birdly.jpg}{width=0.7\hsize}{System of presenting the sensation of flying with flapping\cite{rheiner2014birdly}}

    「羽ばたいて飛ぶ感覚」を与える研究について,\figref{Birdly.jpg}のような操縦装置に登場し,飛行中の鳥の体験をすることができる装置の研究が行われている\cite{rheiner2014birdly}\cite{hypersuit}.
    上記装置は,操縦装置にうつ伏せで搭乗し手と腕を用いて翼を動かしながら,鳥視点での景色の映像を提示することで,飛行中の鳥のような体験できる装置である.この方法の場合,大がかりな装置が必要であることや,手足の動きが制限されるといったデメリットが存在する.
    
    また,羽ばたいて飛ぶ感覚を与える研究はまだ知見が少なく,鳥になり飛ぶ感覚を与える研究が大半であり,トビトカゲのような四肢から独立した翼を持つ生物になり,飛ぶ感覚を与える研究は未だ着目されていない.

    本研究では,四肢の動きを用いないで背中から生えた翼を操作し羽ばたく感覚を提示する手法を提案する.四肢の動きを用いないことで,VR飛行体験中に手足を用いた動作,例えば飛びながら物を投げるといった行為,が可能となりVR飛行体験の幅が広がることが期待できる.

\section{身体像の拡張}
  本研究において以下の要素が重要となる.
  \begin{itemize}
      \item ヒトに本来存在しない「翼」を感じさせる(存在)
      \item その翼で「羽ばたいて飛ぶ感覚」を提示する(運動)
      % \item 翼に作用する外力(外力)
  \end{itemize}
  上記の感覚を与えるために,身体像の拡張について注目する.本節では,身体像について説明し,身体像の拡張の仕組みと方法について述べる.

  \subsection{身体像}
    ヒトは身体像(Body image)\cite{head1911sensory}と呼ばれる,自分自身の身体に関するイメージを持っている.
    自身の身体形状を知覚する能力を有している.それにより自己とそれ以外を区別することができる.

    身体像の基盤となる概念に身体図式(Body schema)がある.身体像が意識された身体の形状情報に対し,身体図式は習慣としての身体の表像,つまり無意識下に身体運動を調整している経験であり顕在的な知識があるとは限らない.身体像は身体図式を基盤として構成される顕在的な自己身体に関する知識を指す\cite{nishida-bodyimage}.
    % コトバンクに詳しく書かれている

    身体像や身体図式は,幻肢痛\cite{Ramachandran}の観察により生じた概念である.幻肢痛とは,事故で存在しないはずの失った手や脚に痛みを感じる症状である.幻肢痛が発症する仕組みのとして,脳における各機能の分布(脳内地図\cite{池谷裕二2007進化しすぎた脳})が書き変わり,幻肢を自分の意思で動かせないことが原因として挙げられる.
    % ブロードマンの脳地図とは別?


    また,身体像と密接に関係する概念に自己がある.認知科学では,自己は永続的に存在する自己(Narrative self)と一時的な自己(Minimal self)から構成されていると考えられている\cite{gallagher2000}.Narrative selfは,過去過去の記憶から未来の展望まで含めたアイデンティティとしての自己,一方,Minimal selfは一時的な自己,つまり経験から即時的に形成される身体的な自己である.
    Minimal selfは,さらに自己所有感(または身体所有感)(Sense of self-ownership)と自己主体感(または行為主体感)(Sense of self-agency)に分類が出来る.
    自己所有感 は,自分の身体部位が自分自身の身体の一部に属していると感じる感覚や経験である\cite{感覚・知覚・認知の基礎}.
    自己主体感は,自分自身である行為を行っている,その身体部位制御しているのは自分であるという感覚や経験である\cite{matsumiya2021awareness}.

    このように,身体像が自己のMinimal selfと密接に関係していることが分かる.従って,仮想翼の身体像を得て飛ぶこと,つまり身体像拡張して動作することで,本研究における以下の要素を満たすことが出来ると考える.

    \begin{itemize}
        \item ヒトに本来存在しない「翼」を感じさせる(自己所有感)
        \item その翼で「羽ばたいて飛ぶ感覚」を提示する(自己主体感)
        % \item 翼に作用する外力(外力)
    \end{itemize}

  \subsection{身体像拡張}

    \fig{bodyimage-body_image_expansion.pdf}{width=1\hsize}{Body image expantion}
    % これは道具の身体化についての図
    % リマッピングの図も欲しい(けど図示ができるのか...)

    自己以外の部分に身体像がダイナミックに変化することがある(\figref{bodyimage-body_image_expansion.pdf}).このことを身体像の拡張(Body image expansion)と呼ぶ.身体像拡張の例として,手に持った道具(テニスラケットや野球バット)を,その形状を意識せず自分の体の一部であるかのように球を打ち返すといった事が挙げられる\cite{渡辺貴文2005仮想道具による身体像拡張の評価手法に関する研究}.
    身体像の拡張は,言い換えると感覚の情報処理を神経系から拡張する能力(錯覚)の事である.

    身体像拡張は,大きく分類して2種類存在し,1つはラバーハンド錯覚(RHI:Rubber Hand Illusion)\cite{botvinick1998rubber}のような感覚のリマッピング,もう1つは先に挙げた道具使用時に身体像がダイナミックに拡張すること(道具の身体化)である.

    \subsubsection{ラバーハンド錯覚}

    ラバーハンド錯覚とは, ラバーハンドをあたかも自分の手のように感じる錯覚である..視界から隠れた本物の手と目の前にあるラバーハンドに絵筆等で2分から20分程度同期した触覚刺激を与え続けると,ラバーハンド上に触覚刺激を知覚するという錯覚現象である.RHI系の身体像拡張の特徴として,元の自分の身体部位と,リマッピングされた部位が共存できないという条件がある.

    ラバーハンド錯覚と同様な身体像拡張の例について,視触覚を同期することで,遠隔にあるロボットやアバターへ乗り移ったような感覚を生成する研究がある
    \cite{tachi2015telexistence}\cite{ehrsson2004s}\cite{slater2008towards}\cite{iwasaki2017research}\cite{petkova2008if}.

  \subsubsection{道具の身体化}
    道具の身体化について,道具使用による身体像拡張をニホンザルを用いて神経生理学的に示した研究がある\cite{iriki1996coding}.この研究では,道具使用時の二ホンザルの登頂連合野における手の体性感覚受容やと手近傍の視覚受容野を持つバイモーダルニューロンの活動を観測することにより,サルの身体像が道具先端まで拡張している事を示した.

    道具の身体化について,指や腕を追加する例があり,指の本数を増やす研究\cite{prattichizzo2014sixth},余剰筋力を用いて第3の腕となるロボットアームを操作する研究\cite{iwadare2017thirdarm}\cite{岩垂真哉2016余剰筋力を用いた第三の腕ロボットの操縦}や顔面ベクトルを用いて第3の腕を操作する研究\cite{iwasaki2017research},両足を用いて第3・第4の腕を操作する研究が挙げられる\cite{sasaki2017metalimbs}.


  \subsection{身体像拡張のアプローチ} 
    本研究では,身体像拡張の中でも道具への身体像拡張に注目する.

    ヒトと自由度やダイナミクスが類似した遠隔ロボットやアバタを,身体動作と完全に同期させることで,RHIのような乗り移ったような感覚の生成や,道具への身体像拡張のように身体の一部として認識可能ということが知られてる.
    % しかし,
    \footnote{身体像拡張の研究において身体像をヒトと異なる構造の対象に投射することや,四肢以外から身体像を拡張させる知見はまだ少ないのが現状である.}

    また,ラバーハンド錯覚について,視覚情報と触覚情報といった感覚情報の時間的一致の重要性が高いことが示されている\cite{本間元康2010ラバーハンドイリュージョン}\cite{ehrsson2007experimental}\cite{shimada2009rubber}.従って,道具への身体像拡張においても感覚情報の時間的一致が重要となると考えられる.
    他方で,空間的な情報一致に関しては柔軟だと考えられており,RHIが生じた状態でラバーハンドを叩くと,被験者が自分の手をたたかれたかのような反応を示す例がある\cite{armel2003projecting}.

    \fig{bodyimage-Method_of_body_image_expansion.pdf}{width=1\hsize}{Method of body expansion}

    以上より,複数感覚の統合,提示する感覚情報の時間的一致させることで身体像拡張(道具の身体化)が期待できることを確認した.
    そこで,本研究では\figref{bodyimage-Method_of_body_image_expansion.pdf}のようにして身体像拡張を試みる,
    % ヒトから情報を仮想翼へ送信し,それに対応する情報をヒトへ返信し,身体像拡張を行う方法を提案する.
    ヒトから
    % 仮想
    翼へは,翼を動かす指令を与え,
    % 仮想
    翼からヒトへは,翼が生えている様子・翼を動かして飛んでいる様子・翼へ作用する空気抵抗の感覚を伝え,複数感覚の統合を試みる.上記より,四肢から独立した翼で羽ばたいて飛ぶ感覚を提示する.

%% \begin{thebibliography}{99}
%% \small
%%  \setlength{\kanjiskip}{0.0zw plus.01zw} %
%%  \setlength{\baselineskip}{9pt}        %
%%  \setlength{\itemsep}{0.2pt}             %
%%  \setlength{\lineskip}{0pt}              %
%%  \setlength{\normallineskip}{0.2pt}      %


%% \bibitem{hogege} 川村マサキ,
%% ほげの可能性と適用限界に関する実験的研究,日本ほげ学会ほげ工学部門講演会,(2010).


%% \bibitem{hohoge} 本堂貴敏,
%% ほげの力学,(2006),pp.11--43,ほげ出版.

%% \end{thebibliography}

{
%% \scriptsize %%←どうしても入らない時は,このコメントをはずすと少し小さくなる.
\bibliographystyle{junsrt}
\bibliography{reference}
}

\end{small}
\end{document}
