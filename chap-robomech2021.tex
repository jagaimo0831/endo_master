\chapter[ロボメック2021の内容]%
        {ロボメック2021の内容}
    
\section{緒言}

\fig{WingMan.png}{width=1\hsize}{Flying with flapping virtual wings independent of the limbs}

\fig{How2present_the_feeling_of_flapping_eng.pdf}{width=1\hsize}{Research concept}

ヒトは古くから空を飛ぶことに憧れを抱いている.実際に飛ぶことにはリスクやコストが伴うが,VR装置を使用することで簡単に飛行体験が可能である.
\figref{WingMan.png}は,四肢から独立した翼で羽ばたいて飛ぶ様子を示した物である.
本研究では,
VR装置を用いて\figref{WingMan.png}のように羽ばたいて飛ぶ感覚の提示手法を提案する.

    \subsection{浮遊感・飛ぶ感覚・羽ばたいて飛ぶ感覚の定義}
    本研究では「羽ばたいて飛ぶ感覚」を「浮遊感」,「飛ぶ感覚」を用いて次のように位置づける.
    「浮遊感」とは,空中に浮いて漂っている感覚を指す.「飛ぶ感覚」は浮遊感に空中を移動する感覚を追加したものとする.本稿で注目する「羽ばたいて飛ぶ感覚」は,飛ぶ感覚に加え,翼を羽ばたかせる感覚を追加したものである.

    \subsection{関連研究}    
    浮遊感や飛ぶ感覚を与える研究は多く行われてきた.視覚刺激をによって発生する落下感覚に関しての研究\cite{奥川夏輝2017VR空間における視覚刺激によって発生する落下感覚の分析}や身体幇助メカニズムを用いた飛行体験装置の提案\cite{鈴木拓馬2014hmd}等がある.また,飛行しているドローンを上半身のジェスチャーで制御し,ドローンからの映像をヘッドマウントディスプレイ(以下HMD)によって与えることで飛ぶ感覚を提示する研究\cite{rognon2018flyjacket}もある.

    浮遊感と飛ぶ感覚の研究に対して,鳥のように羽ばたいて飛ぶ感覚を与える研究はまだ少ない.羽ばたいて飛ぶ感覚を与える研究の例としては,飛行中の鳥の体験をすることができる装置であるBirdly\cite{rheiner2014birdly}やHypersuit\cite{hypersuit}がある.操縦装置にうつ伏せで搭乗し手と腕を用いて翼を動かしながら,鳥視点での景色の映像を提示することで,飛行中の鳥のような体験できる装置である.

    \subsection{研究目的}
    従来の羽ばたいて飛ぶ感覚を与える研究は,ヒトが鳥のように腕を動かすことによって羽ばたいて飛ぶ感覚を提示していた.しかし,大がかりな装置が必要であることや,手足の動きが制限されるといったデメリットが存在する.
    また,羽ばたいて飛ぶ感覚を与える研究において,\figref{WingMan.png}のような背中から翼が生えた生物になる感覚を提示する研究はまだ着目されていない.

    そこで四肢から独立した翼を用いて,羽ばたいて飛ぶ感覚を提示する方法を提案する.
    本研究では,
    % 大がかりな装置を使用しないこと,(核ではないので排除)
    四肢を用いずに翼を操作している感覚の提示手法とVR空間で翼に作用する力をヒトに提示する手法を考案する.
    これにより,羽ばたいて飛んでいる状態での投擲や射撃が可能となる.これらのように飛行体験中に手足を使用する新しいアプリケーションが期待できる.

    本研究では,ヒトに翼が生えている感覚を与えるために,身体像の拡張について注目をする.

    
    \subsection{身体像の拡張}
    ヒトは身体像と呼ばれる,自身の身体形状を知覚する能力を有している.それにより自己とそれ以外を区別することができる.しかし,自己以外の部分に身体像が拡張する場合がある.身体像の拡張に関する代表的な研究として,
    % 偽物の手である
    ラバーハンドをあたかも自分の手のように感じるラバーハンド錯覚についての研究がある\cite{botvinick1998rubber}.視界から隠れた本物の手と目の前にあるラバーハンドに絵筆等で2分から20分程度同期した触覚刺激を与え続けると,ラバーハンド上に触覚刺激を知覚するという錯覚現象である.
    このように提示される視覚情報と,触覚情報の位置が一致または近しければ身体像を拡張することが可能となる.
    ラバーハンド錯覚ではヒトは情報を受けとるだけであったが,ヒトから情報を送信し,それに対する返信を受け取ることで身体像の拡張をより円滑にすることができると考える.
    
    身体像の拡張には情報の双方向性が重要であることを踏まえ,本研究では\figref{How2present_the_feeling_of_flapping_eng.pdf}のような形で身体像の拡張を行う.
    ヒトから仮想翼へは,翼を動かす指令を与える.仮想翼からヒトへは,翼が生えている様子,翼を動かして飛んでいる様子,翼へ作用する空気抵抗の感覚を伝える.上記より,四肢から独立した翼で羽ばたいて飛ぶ感覚を提示する.
    
    \subsection{ヒトから仮想翼}
    まず,ヒトから仮想翼へ翼を動かす指令を与える方法について述べる.

    ヒトから仮想翼を操作する方法として,コントローラやジェスチャによる操作や,生体信号を用いることが挙げられる.
    本研究では,四肢以外で動かすことが目的なので,主に手を用いるコントローラや,手足の動きが必要となるジェスチャではなく,生体信号を用いる.また,生体信号の中でも数値の取得が容易な筋電位によって翼を操作する.

    \fig{How2present_force_applied2wings_eng.pdf}{width=1\hsize}{How to present force applied virtual wings}




   
        

