\chapter[関連研究]%
        {関連研究}

\section{はじめに}
        本節では,関連動向調査分析と問題点・課題の提示を行う.
        (※追加で身体像拡張について?)

\section{関連研究}
        浮遊感や飛ぶ感覚を与える研究は多く行われてきた.視覚刺激をによって発生する落下感覚に関しての研究\cite{奥川夏輝2017VR空間における視覚刺激によって発生する落下感覚の分析}や身体幇助メカニズムを用いた飛行体験装置の提案\cite{鈴木拓馬2014hmd}等がある.また,飛行しているドローンを上半身のジェスチャーで制御し,ドローンからの映像をヘッドマウントディスプレイ(以下HMD)によって与えることで飛ぶ感覚を提示する研究\cite{rognon2018flyjacket}もある.

        浮遊感と飛ぶ感覚の研究に対して,鳥のように羽ばたいて飛ぶ感覚を与える研究はまだ少ない.羽ばたいて飛ぶ感覚を与える研究の例としては,飛行中の鳥の体験をすることができる装置であるBirdly\cite{rheiner2014birdly}やHypersuit\cite{hypersuit}がある.操縦装置にうつ伏せで搭乗し手と腕を用いて翼を動かしながら,鳥視点での景色の映像を提示することで,飛行中の鳥のような体験できる装置である.

\section{おわりに}