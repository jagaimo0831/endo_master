\chapter[四肢から独立した翼の提示方法]%
        {四肢から独立した翼の提示方法}

\section{はじめに}
    本章では,四肢から独立した翼の提示方法について述べる.前章の身体像の拡張では,身体像についての概要と身体像拡張の原理と方法について述べた.本章では,前章の内容を踏まえた四肢から独立した翼の提示方法を提案する.


\section{身体像拡張を行う方法}
    \fig{method-Method_of_body_image_expansion.pdf}{width=1\hsize}{Method of body image expansion}
    身体像の拡張には情報の双方向性が重要であることを踏まえ,本研究では\figref{method-Method_of_body_image_expansion.pdf}のような形で身体像の拡張を行う.ヒトから仮想翼へは,翼を動かす指令を与える.仮想翼からヒトへは,翼が生えている様子(自己所有感),翼を動かして飛んでいる様子(自己主体感),翼へ作用する空気抵抗の感覚(外力)を伝える.上記より,四肢から独立した翼で羽ばたいて飛ぶ感覚を提示する.
    
\section{ヒトから仮想翼への情報伝達}
    まず,ヒトから仮想翼へ翼を動かす指令を与える方法について述べる.

    VR空間でのヒトからシステムへの情報提示方法として,コントローラやジェスチャによる操作や,生体信号を用いることが挙げられる.
    本研究では,四肢以外で翼を操作することを目的としているため,主に手を用いるコントローラや,手足の動きが必要となるジェスチャではなく,四肢の動きが伴わずに計測が可能な生体信号を用いる.また,生体信号の中でも数値の取得が容易な筋電位によって仮想翼を操作する.
    
    \subsection{筋電の計測}
        筋電位とは,筋肉が収縮する際に発する微弱な活動電位の事を指す.筋電センサは,筋肉の発する微弱な信号を増幅し計測を行う機器である\cite{alts-myography}.筋電センサには,乾式タイプと湿式タイプが存在し,それぞれの以下の特徴を持つ.本章では乾式と湿式どちらも使用する.
        % 筋肉の動作メカニズムも乗っけておくと良い
        % 乾式湿式の前に,侵襲or非侵襲の違いがある...

        \begin{itemize}
        \item 乾式筋電センサ
            \begin{itemize}
            \item 金属製の電極を皮膚へ接触させて計測を行う
            \item 電極を交換する必要が無い
            \item 激しい動きで,電極がズレてノイズが生じる可能性がある
            \item 汗の影響を受けやすい
            \item Ex. MYO(Thalmic Labs)
            \end{itemize}

        \item 湿式筋電センサ
            \begin{itemize}
            \item ジェル状の電極を皮膚へ貼り付けて計測を行う
            \item 電極が使い捨てなため,費用がかかる
            \item 皮膚との密着性が高く,信号にノイズが生じにくい
            \item Ex. MyoWare(Advancer Technologies)
            \end{itemize}
        \end{itemize}


\section{仮想翼からヒトへの情報伝達}
    次に,仮想翼からヒトへ情報を与える方法について述べる.

    ヒトへ働きかける感覚として
    % ナイーブな表現だが
    五感が挙げられる.
    その中でも,身体像の拡張におけるヒトへ働きかける情報として,視覚と触覚
    % ,聴覚
    が用いられることが多い.
    これは,身体像拡張において視触覚の統合が有効であることを示す.
    % 聴覚に関しては空間的定位,ここでは翼のある場所を認識する場合において,一般的に視覚よりも情報としての重要度が低く\cite{岡嶋克典20182},提示する情報としての優先度が低かったので今回は用いない.    
    % 以上を踏まえて本研究では,五感の中でも視覚と触覚を用いて仮想翼からの情報を提示する.
    本研究でも,視覚と触覚を用いて仮想翼からの情報を提示する.

    \subsection{視覚の提示}
        \fig{hmd_vection.png}{width=1\hsize}{Presentation of visual information by using the virtual environment}
        
        視覚を用いた提示は,\figref{hmd_vection.png}のようにUnityで作成した映像を
        % HMDに
        視覚ディスプレイに出力することで行う.
        % HMDに
        出力される映像は,空中を移動している様子と,背中からはえた翼の一部が見える様子を提示する.

        空中を移動している様子の提示について述べる.
        空中を移動する様子の提示として,ベクション(自己誘導性自己運動感覚)と呼ばれる錯覚を用いる.ベクションとは,視野の大部分に一様な運動刺激を提示すると刺激の運動方向と反対の方向に体が動いているように感じる錯覚である\cite{bhalla1999visual}\cite{妹尾武治2014ベクションとその周辺の近年の動向}.例として,停車中の電車から動き出す他の電車の視覚情報を受け取ると,観測者側の電車が動いているように感じる現象が挙げられる.
        % 浮遊感に関する研究で,ベクションは多く活用されている.例としてベクションによる落下感覚を分析した研究がある\cite{奥川夏輝2017VR空間における視覚刺激によって発生する落下感覚の分析}.このようにVRを用いた飛行体験においてベクションは有用である.

        背中から翼が生えている様子・翼を動かして飛んでいる様子は,後ろを振り返ると使用者の背中から翼が生えている様子を描画されるようにする.また,翼を羽ばたかせる際に,視界に翼を広げたり閉じたりする様子の描画を行う.以上より,背中から翼が生え羽ばたく様子を提示する.

% とりあえずいらない
        % \subsubsection{視覚提示を行うデバイス}
        %     \fig{hmd-vive_pro_eye.jpg}{width=0.7\hsize}{Head Mounted Display\cite{htc-vive}}

        %     視覚の提示には,\figref{hmd-vive_pro_eye.jpg}のようなヘッドマウントディスプレイ(以下HMD)を用いる.HMDは頭部に装着するディスプレイであり,ゴーグル型やメガネ型の種類が存在する.テレビやモニターのような,一般的なディスプレイとは異なり眼前に映像が表示され,視界を占める映像の割合が多くなるため,映像への没入感が高い.

        %     以下に,HMDの大まかな分類と特徴,製品例を示す.
        %     % 特徴について述べていない
        %     % 具体製品例については,箇条書きだとスペース取りすぎで逆に見づらいので1行にまとめる
        %     \begin{itemize}
        %     \item ゴーグル型\\
        %         ...眼の回りを完全に覆うタイプ.スタンドアローン型,PC接続型,ゲーム機型,スマホ型がある.
        %         \begin{itemize}
        %         \item スタンドアローン型\\
        %             ...PCとの接続等が不要で,単体でVR体験が可能な機器.
        %             \begin{itemize}
        %             \item VIVE Flow
        %             \item Meta Quest2
        %             \item HTC Vive Focus 3
        %             \end{itemize}
        %         \item PC接続型
        %             \begin{itemize}
        %             \item HTC VIVE PRO
        %             \item Windows MR
        %             \item Valve Index
        %             \item Pimax
        %             \item Varjo
        %             \end{itemize}
        %         \item ゲーム機型
        %             \begin{itemize}
        %             \item PlayStation VR
        %             \item Nintendo Switch Labo: VR Kit
        %             \end{itemize}
        %         \item スマホ型
        %             \begin{itemize}
        %             \item ハコスコ タタミミ 二眼
        %             \item エレコム/VRG-D02PBK
        %             \item サンワサプライ/MED-VRG1
        %             \end{itemize}
        %         \end{itemize}

        %     \item メガネ型
        %         \begin{itemize}
        %         \item Microsoft HoloLens2
        %         \item Magic Leap1
        %         \end{itemize}
        %     \end{itemize}

    
    \subsection{触覚の提示}
        % \fig{How2present_force_applied2wings_eng.pdf}{width=1\hsize}{How to present haptics applied virtual wings}
        % この図はリマッピングの身体像の拡張のやり方,実際に行ったのは道具への身体像拡張のやり方なので差し替える
        % あと生えている位置が腰ではなく,背中になったのでそれも反映する(図)

        触覚を用いた提示は,
        % \figref{How2present_force_applied2wings_eng.pdf}のように
        翼の根元に触覚を提示することで,翼が生えている様子・翼を動かして飛んでいる様子・翼に作用する空気抵抗の感覚を提示する.
        これは,ある1つの領域への触覚提示による身体像拡張を促すものあり,道具の身体化の手で所持した棒の先端まで身体像が拡張される事象と同様な例である.
        % 上は,身体像拡張(道具拡張,リマップの2種分類)の内,道具拡張の方を採用したよ!ということを述べたい文章.
        
        また本研究では,触覚ディスプレイとして振動を用いた触覚提示と電気刺激を用いた触覚提示の2つを準備する.
        
        触覚提示として一般的な振動を用いた触覚提示は,
        偏心モータによる振動を提示を行う.振動を用いた触覚提示の例として,ゲームのコントローラに振動機能を追加\cite{shim2020fs},携帯電話やスマートフォンのバイブレーション機能などが挙げられる.

        電気刺激を用いた触覚提示は,筋収縮により疑似的に触覚を提示する方法を用いる.この方法はEMS(Electro Myo Stimulation: 神経筋電気刺激療法)と呼ばれる,筋肉や運動神経へ電気刺激を与えることで筋収縮を促し,運動効果を得ることで筋肉の増強や萎縮の予防等をする治療法を用いた手法である.EMSにより筋肉を収縮させることで,疑似的に触覚(重量の知覚)を提示する研究がある\cite{小川剛史2017電気的筋肉刺激が重量知覚に及ぼす影響の分析}.本研究では,EMS機器により筋収縮を起こすことで
        % 疑似的に
        触覚を生じさせ,仮想翼からの情報を提示する.

\section{おわりに}
    本章では,前章の身体像の拡張の内容を踏まえた,四肢から独立した翼の提示方法について述べた.身体像拡張の方法として,ヒトから仮想翼への情報伝達・仮想翼からヒトへの情報伝達の内容と方法について述べた.
