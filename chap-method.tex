\chapter[四肢から独立した翼の提示方法]%
        {四肢から独立した翼の提示方法}

\section{はじめに}
    本章では,四肢から独立した翼の提示方法について述べる.前章の身体像の拡張では,身体像についての概要と身体像拡張の原理と方法について述べた.本章では,前章の内容を踏まえた四肢から独立した翼の提示方法を提案する.


\section{身体像拡張を行う方法}
    \fig{How2present_the_feeling_of_flapping_eng.pdf}{width=1\hsize}{Method of body expansion}
    身体像の拡張には情報の双方向性が重要であることを踏まえ,本研究では\figref{How2present_the_feeling_of_flapping_eng.pdf}のような形で身体像の拡張を行う.ヒトから仮想翼へは,翼を動かす指令を与える.仮想翼からヒトへは,翼が生えている様子,翼を動かして飛んでいる様子,翼へ作用する空気抵抗の感覚を伝える.上記より,四肢から独立した翼で羽ばたいて飛ぶ感覚を提示する.
    
\section{ヒトから仮想翼への情報伝達}
    まず,ヒトから仮想翼へ翼を動かす指令を与える方法について述べる.

    ヒトから仮想翼を操作する方法として,コントローラやジェスチャによる操作や,生体信号を用いることが挙げられる.
    本研究では,四肢以外で動かすことが目的なので,主に手を用いるコントローラや,手足の動きが必要となるジェスチャではなく,四肢の動きが伴わない計測が可能な生体信号を用いる.また,生体信号の中でも数値の取得が容易な筋電によって翼を操作する.
    
    筋電とは,筋肉が収縮する際に発する微弱な電気信号の事を指す.筋電の計測機器(筋電センサ)は,大きく分けて乾式タイプと湿式タイプが存在する.それぞれの特徴を以下に示す.

    \begin{itemize}
    \item 乾式筋電センサ
        \begin{itemize}
        \item 金属製の電極を皮膚へ接触させて計測を行う
        \item 電極を交換する必要が無い
        \item 激しい動きで,電極がズレてノイズが生じる可能性がある
        \item 汗の影響を受けやすい
        \item Ex. MYO(Thalmic Labs)
        \end{itemize}

    \item 湿式筋電センサ
        \begin{itemize}
        \item ジェル状の電極を皮膚へ貼り付けて計測を行う
        \item 電極が使い捨てなため,費用がかかる
        \item 皮膚との密着性が高く,信号にノイズが生じにくい
        \item Ex. MyoWare(Advancer Technologies)
        \end{itemize}
    \end{itemize}

    
\section{仮想翼からヒトへの情報伝達}
    \fig{How2present_force_applied2wings_eng.pdf}{width=1\hsize}{How to present force applied virtual wings}

    \fig{hmd_vection.png}{width=1\hsize}{Virtual presentation by HMD}

    次に,仮想翼からヒトへ情報を与える方法について述べる.

    ヒトへ働きかける感覚として主に五感が挙げられる.
    ヒトへ働きかける情報として,五感の中でも力覚(触覚)と視覚,聴覚が重要と考えた.
    聴覚に関しては空間的定位,ここでは翼のある場所を認識する場合において,一般的に視覚よりも情報としての重要度が低い\cite{岡嶋克典20182}ので今回は不採用とする.
    以上を踏まえて本研究では,五感の中でも力覚(触覚)と視覚を用いて仮想翼からの情報を提示する.

    \subsubsection{視覚の提示}
 
        視覚を用いた提示は,\figref{hmd_vection.png}のようにUnityで作成した映像をHMDに出力することで行う.HMDに出力される映像は,空中を移動している様子と背中から翼が生えている様子である.

        空中を移動している様子の提示について述べる.
        ベクションと呼ばれる,視野の大部分に一様な運動刺激を提示すると刺激の運動方向と反対の方向に体が動いているように感じる錯覚がある\cite{妹尾武治2014ベクションとその周辺の近年の動向}.例として,停車中の電車から動き出す他の電車の視覚情報を受け取ると,観測者側の電車が動いているように感じる現象が挙げられる.
        浮遊感に関する研究で,ベクションによる落下感覚を分析した研究がある\cite{奥川夏輝2017VR空間における視覚刺激によって発生する落下感覚の分析}.
        空中を移動している様子の提示はベクションを用いる.

        背中から翼が生えている様子は,使用者の背中から翼が生えている可のような映像を出力することで再現する.  

        \subsection{視覚提示について}
        ・翼の3Dオブジェクトの準備(選定)

        ・HMDの選定
            - HMDの紹介(psvr, oculus, htc) (PC用,独立型,スマホ用)(情報まとめ)-> 選定理由

        ・環境の選定?(Unity, UnrealEngine, Blender),OS
            - 候補と各ソフトウェアの説明,選定理由

        ・terrain の説明

    
    \subsubsection{触覚の提示}

        力覚を用いた提示は,
        \figref{How2present_force_applied2wings_eng.pdf}
        のように羽ばたく際に翼の場所ごとに作用する力を,ヒトの体に対応させることで,翼が連動的にしなっている様子を伝える.

        力覚提示の種類として2種類について比較した.
        1つ目は,モータによる振動または押す力を活用し力覚を提示する.
        2つ目は,EMS(神経筋電気刺激療法)という筋肉や運動神経へ電気刺激を与えることで筋収縮を促し,筋肉の増強や萎縮の予防等をする治療法を用いたものである.EMSにより筋肉を収縮させることで,疑似的に重量を知覚させる研究がある\cite{小川剛史2017電気的筋肉刺激が重量知覚に及ぼす影響の分析}.本研究では,EMS機器により筋収縮を起こすことで疑似的に力覚を提示する.

    \subsection{触覚提示について}
        振動と電気にした
        (電気の仕組み)
        デバイスの選択

            電気系(もっと調べる)(場所はどこでも対応可)
        EMSの内医療認可を受けたものを低周波治療器と呼ぶ(周波数関係なしに低周波治療器という)
        携帯型(例えばこんなの)は周波数?パターンが既にプログラムされているので制御はダルそう?
        →安定化電源的なやつ(これとか,これ)
        ヒトの抵抗値(1000~3000Ω),30mAで死ねる.→なのになぜ1000V, 100mA出せる?
        制御できるのだろうか
        周波数について(リンク)
        低周波:筋肉運動しやすい
        高周波:皮膚抵抗が減る→インナーマッスルまで届く
        装置候補
        論文:電気的筋肉刺激が重量近くに及ぼす影響の分析
        電気刺激装置:Digitimer社の医療用電気刺激装置マルチパスD185
        トリガ制御:Arudiono MEGA(シリアル通信でPCから刺激タイミング,刺激時間,周波数の調整)
        電極:日本光電社のPALS Electrodes(MODEL 895220)
        装置に関しては「電気刺激装置」と検索すると良い(not EMS)


        トランジスタon/off

        HV-F122,125,127
        
\section{おわりに}
    hogehogeなのでfugafuga.