\chapter[四肢から独立した翼の提示方法]%
        {四肢から独立した翼の提示方法}

\section{はじめに}
    本節では,四肢から独立した翼の提示方法について述べる.



\section{身体像拡張を行う方法}
    まず,ヒトから仮想翼へ翼を動かす指令を与える方法について述べる.

    ヒトから仮想翼を操作する方法として,コントローラやジェスチャによる操作や,生体信号を用いることが挙げられる.
    本研究では,四肢以外で動かすことが目的なので,主に手を用いるコントローラや,手足の動きが必要となるジェスチャではなく,生体信号を用いる.また,生体信号の中でも数値の取得が容易な筋電位によって翼を操作する.

    \fig{How2present_force_applied2wings_eng.pdf}{width=1\hsize}{How to present force applied virtual wings}

    \fig{hmd_vection.png}{width=1\hsize}{Virtual presentation by HMD}

    次に,仮想翼からヒトへ情報を与える方法について述べる.

    ヒトへ働きかける感覚として主に五感が挙げられる.
    ヒトへ働きかける情報として,五感の中でも力覚(触覚)と視覚,聴覚が重要と考えた.
    聴覚に関しては空間的定位,ここでは翼のある場所を認識する場合において,一般的に視覚よりも情報としての重要度が低い\cite{岡嶋克典20182}ので今回は不採用とする.
    以上を踏まえて本研究では,五感の中でも力覚(触覚)と視覚を用いて仮想翼からの情報を提示する.

    \subsubsection{力覚を用いた仮想翼からヒトへの情報提示}

        力覚を用いた提示は,
        \figref{How2present_force_applied2wings_eng.pdf}
        のように羽ばたく際に翼の場所ごとに作用する力を,ヒトの体に対応させることで,翼が連動的にしなっている様子を伝える.

        力覚提示の種類として2種類について比較した.
        1つ目は,モータによる振動または押す力を活用し力覚を提示する.
        2つ目は,EMS(神経筋電気刺激療法)という筋肉や運動神経へ電気刺激を与えることで筋収縮を促し,筋肉の増強や萎縮の予防等をする治療法を用いたものである.EMSにより筋肉を収縮させることで,疑似的に重量を知覚させる研究がある\cite{小川剛史2017電気的筋肉刺激が重量知覚に及ぼす影響の分析}.本研究では,EMS機器により筋収縮を起こすことで疑似的に力覚を提示する.
    
    \subsubsection{視覚を用いた仮想翼からヒトへの情報提示}
 
        視覚を用いた提示は,\figref{hmd_vection.png}のようにUnityで作成した映像をHMDに出力することで行う.HMDに出力される映像は,空中を移動している様子と背中から翼が生えている様子である.

        空中を移動している様子の提示について述べる.
        ベクションと呼ばれる,視野の大部分に一様な運動刺激を提示すると刺激の運動方向と反対の方向に体が動いているように感じる錯覚がある\cite{妹尾武治2014ベクションとその周辺の近年の動向}.例として,停車中の電車から動き出す他の電車の視覚情報を受け取ると,観測者側の電車が動いているように感じる現象が挙げられる.
        浮遊感に関する研究で,ベクションによる落下感覚を分析した研究がある\cite{奥川夏輝2017VR空間における視覚刺激によって発生する落下感覚の分析}.
        空中を移動している様子の提示はベクションを用いる.

        背中から翼が生えている様子は,使用者の背中から翼が生えている可のような映像を出力することで再現する.  


\section{筋電計測について}
    生体信号,筋電計測
    ジェスチャ,コントローラ,外骨格
    ・乾式,湿式
    MYO(使い方等?構成,ソフトウェア)

    自作筋電計測装置
    https://invbrain.neuroinf.jp/static/moth/EMG-tool.pdf
    (12/15)





\section{触覚提示について}
    振動と電気にした
    (電気の仕組み)
    デバイスの選択

    電気系(もっと調べる)(場所はどこでも対応可)
EMSの内医療認可を受けたものを低周波治療器と呼ぶ(周波数関係なしに低周波治療器という)
携帯型(例えばこんなの)は周波数?パターンが既にプログラムされているので制御はダルそう?
→安定化電源的なやつ(これとか,これ)
ヒトの抵抗値(1000~3000Ω),30mAで死ねる.→なのになぜ1000V, 100mA出せる?
制御できるのだろうか
周波数について(リンク)
低周波:筋肉運動しやすい
高周波:皮膚抵抗が減る→インナーマッスルまで届く
装置候補
論文:電気的筋肉刺激が重量近くに及ぼす影響の分析
電気刺激装置:Digitimer社の医療用電気刺激装置マルチパスD185
トリガ制御:Arudiono MEGA(シリアル通信でPCから刺激タイミング,刺激時間,周波数の調整)
電極:日本光電社のPALS Electrodes(MODEL 895220)
装置に関しては「電気刺激装置」と検索すると良い(not EMS)


    トランジスタon/off

    HV-F122,125,127



    


\section{視覚提示について}
    ・翼の3Dオブジェクトの準備(選定)

    ・HMDの選定
        - HMDの紹介(psvr, oculus, htc) (PC用,独立型,スマホ用)(情報まとめ)-> 選定理由

    ・環境の選定?(Unity, UnrealEngine, Blender),OS
        - 候補と各ソフトウェアの説明,選定理由

    ・terrain の説明



\section{流体シミュレータについて} 
%流体シミュレータは結局使わなかったけど折角調べてあったのでとりあえず書いとく...
    空気から受ける力をシミュレーションし,その力をヒトへ与えることで翼で羽ばたいて飛ぶ感覚を提示する.空気から受ける力をシミュレートするのに流体シミュレータを用いる.使用する流体シミュレータの候補として以下のソフトウェアが挙げられる.

    
        \begin{itemize}
        \item \href{http://flowsquare.com/jp/}{Flowsquare}
            \begin{itemize}
            \item 開発: Nora Scientific(2009年)
            \item 特徴: 2次元非定常,非反応/反応性,完全圧縮性/非圧縮性流体のシミュレーションソフト 
            \item 対応OS: Windows
            \item 料金: 無料
            % \item 無料(典型的な流体シミュレーションソフトは1ライセンスあたり数10万くらい(参考:\href{https://icfd.co.jp/product/price.html}{株式会社流体力学研究所})
            % \item 専門知識(プログラミング・CAD・メッシュ生成・前処理(初期場ほ生成)・後処理etc)を必要としない.\\
            % -\textgreater ペイントソフトを用いて解析対象の絵を書く,解析したい条件(流体速度)をテキストファイルに入力
            \end{itemize}
            
        \item \href{https://fsp.norasci.com/}{Flowsquare+}
            \begin{itemize}
            \item 開発: Nora Scientific
            \item 特徴:
                \begin{itemize}
                \item Flowsqureの新バージョン.
                \item 3次元の解析に対応
                \item CFD(Computational Fluid Dynamics:数値流体力学)搭載
                \end{itemize}
            \item 対応OS: Windows
            \item 料金: 無料
            % \textless\textless 通常100万以上のコスト
            % \item 以前と同様に専門知識不要
            \end{itemize}
            
        \item \href{https://fastar.chofu.jaxa.jp/}{FaSTAR}
            \begin{itemize}
            \item 開発: JAXA (宇宙航空研究開発機構)
            \item 特徴: 
                \begin{itemize}
                \item Fast Unstructuired CFD Code
                \item 高速非構造格子(任意の形状のメッシュ)に対応した圧縮性流体解析ソルバー
                \item 航空機や宇宙器などの空力解析に適する
                \end{itemize}
            \item 料金: 授業等の教育目的に限り無償で提供
            \end{itemize}
            
        \item \href{https://altairhyperworks.jp/product/ultrafluidx}{ultraFluidX}
            \begin{itemize}
            \item GPUが必要 (というかサーバーが1基必要...)
            \end{itemize}
        
        \item \href{https://www.openfoam.com/}{OpenFOAM}
        
        \item \href{http://www.ciss.iis.u-tokyo.ac.jp/dl/}{FrontFlow/blue}
            \begin{itemize}
            \item 国産
            \item blue: 乱流音場用,  red: 乱流燃焼用
            \end{itemize}
        
        \item \href{http://www.cenav.org/kdb/?page_id=328}{FrontFlow/violet Cartesian}
            \begin{itemize}
            \item 直交格子を用いた実用複雑系流体解析プログラム
            \end{itemize}
        
        \item \href{http://www.cenav.org/kdb/?page_id=334}{FrontWorkBench}
            \begin{itemize}
            \item 流体・構造・音響錬成解析の自動設定
            \end{itemize}
        
        \item \href{https://www.blender.org/download/}{Blender}
            \begin{itemize}
            \item コンピュータグラフィックスソフトで有名
            \item Unityでも流体解析はできる
            \end{itemize}

        \item \href{https://fenicsproject.org/}{FEniCS}
            \begin{itemize}
            \item pythonやC++で開発可能
            \item 英語
            \end{itemize}
        
        \end{itemize}

        手持ちのノートPCのスペックで使用可能(コロナで在宅な為),無償,3次元シミュレーションが出来る,という観点から今回はFlowSqure+を使用する.((美術)解剖学的には人間の形を保ったまま,背中から生えた翼でバランスよく飛翔することは困難であるので,現実的にはあまり意味はない解析である(\href{https://genkosha.pictures/illustration/18103116710}{小田隆 PICTURES 美しい美術解剖図 第2回 人体に翼を生やすことは可能か?キューピッドを美術解剖図で考察する})).
        
\section{おわりに}