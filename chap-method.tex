\chapter[四肢から独立した翼の提示方法]%
        {四肢から独立した翼の提示方法}

\section{はじめに}
    本節では,四肢から独立した翼の提示方法について述べる.
    
\section{身体像拡張について}
    ヒトは身体像と呼ばれる,自身の身体形状を知覚する能力を有している.それにより自己とそれ以外を区別することができる.しかし,自己以外の部分に身体像が拡張する場合がある.身体像の拡張に関する代表的な研究として,
    % 偽物の手である
    ラバーハンドをあたかも自分の手のように感じるラバーハンド錯覚についての研究がある\cite{botvinick1998rubber}.視界から隠れた本物の手と目の前にあるラバーハンドに絵筆等で2分から20分程度同期した触覚刺激を与え続けると,ラバーハンド上に触覚刺激を知覚するという錯覚現象である.
    このように提示される視覚情報と,触覚情報の位置が一致または近しければ身体像を拡張することが可能となる.
    ラバーハンド錯覚ではヒトは情報を受けとるだけであったが,ヒトから情報を送信し,それに対する返信を受け取ることで身体像の拡張をより円滑にすることができると考える.

    身体像の拡張には情報の双方向性が重要であることを踏まえ,本研究では\figref{How2present_the_feeling_of_flapping_eng.pdf}のような形で身体像の拡張を行う.
    ヒトから仮想翼へは,翼を動かす指令を与える.仮想翼からヒトへは,翼が生えている様子,翼を動かして飛んでいる様子,翼へ作用する空気抵抗の感覚を伝える.上記より,四肢から独立した翼で羽ばたいて飛ぶ感覚を提示する.

\section{身体像拡張を行う方法}
    まず,ヒトから仮想翼へ翼を動かす指令を与える方法について述べる.

    ヒトから仮想翼を操作する方法として,コントローラやジェスチャによる操作や,生体信号を用いることが挙げられる.
    本研究では,四肢以外で動かすことが目的なので,主に手を用いるコントローラや,手足の動きが必要となるジェスチャではなく,生体信号を用いる.また,生体信号の中でも数値の取得が容易な筋電位によって翼を操作する.

    \fig{How2present_force_applied2wings_eng.pdf}{width=1\hsize}{How to present force applied virtual wings}

    \fig{hmd_vection.png}{width=1\hsize}{Virtual presentation by HMD}

    次に,仮想翼からヒトへ情報を与える方法について述べる.

    ヒトへ働きかける感覚として主に五感が挙げられる.
    ヒトへ働きかける情報として,五感の中でも力覚(触覚)と視覚,聴覚が重要と考えた.
    聴覚に関しては空間的定位,ここでは翼のある場所を認識する場合において,一般的に視覚よりも情報としての重要度が低い\cite{岡嶋克典20182}ので今回は不採用とする.
    以上を踏まえて本研究では,五感の中でも力覚(触覚)と視覚を用いて仮想翼からの情報を提示する.

    \subsubsection{力覚を用いた仮想翼からヒトへの情報提示}

        力覚を用いた提示は,
        \figref{How2present_force_applied2wings_eng.pdf}
        のように羽ばたく際に翼の場所ごとに作用する力を,ヒトの体に対応させることで,翼が連動的にしなっている様子を伝える.

        力覚提示の種類として2種類について比較した.
        1つ目は,モータによる振動または押す力を活用し力覚を提示する.
        2つ目は,EMS(神経筋電気刺激療法)という筋肉や運動神経へ電気刺激を与えることで筋収縮を促し,筋肉の増強や萎縮の予防等をする治療法を用いたものである.EMSにより筋肉を収縮させることで,疑似的に重量を知覚させる研究がある\cite{小川剛史2017電気的筋肉刺激が重量知覚に及ぼす影響の分析}.本研究では,EMS機器により筋収縮を起こすことで疑似的に力覚を提示する.
    
    \subsubsection{視覚を用いた仮想翼からヒトへの情報提示}
 
        視覚を用いた提示は,\figref{hmd_vection.png}のようにUnityで作成した映像をHMDに出力することで行う.HMDに出力される映像は,空中を移動している様子と背中から翼が生えている様子である.

        空中を移動している様子の提示について述べる.
        ベクションと呼ばれる,視野の大部分に一様な運動刺激を提示すると刺激の運動方向と反対の方向に体が動いているように感じる錯覚がある\cite{妹尾武治2014ベクションとその周辺の近年の動向}.例として,停車中の電車から動き出す他の電車の視覚情報を受け取ると,観測者側の電車が動いているように感じる現象が挙げられる.
        浮遊感に関する研究で,ベクションによる落下感覚を分析した研究がある\cite{奥川夏輝2017VR空間における視覚刺激によって発生する落下感覚の分析}.
        空中を移動している様子の提示はベクションを用いる.

        背中から翼が生えている様子は,使用者の背中から翼が生えている可のような映像を出力することで再現する.  





\section{おわりに}