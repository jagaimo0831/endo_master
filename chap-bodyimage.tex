\chapter[身体像の拡張]%
        {身体像の拡張}

\section{はじめに}
    本研究において以下の要素が重要となる.
    \begin{itemize}
        \item ヒトに本来存在しない「翼」を感じさせる(存在)
        \item その翼で「羽ばたいて飛ぶ感覚」を提示する(運動)
    \end{itemize}
    上記の感覚を与えるために,身体像の拡張について注目する.本章では,身体像について説明し,身体像の拡張の仕組みと方法について述べる.

\section{身体像}
    ヒトは身体像(Body image)\cite{head1911sensory}と呼ばれる,自分自身の身体に関するイメージを持っている.
    自身の身体形状を知覚する能力を有している.それにより自己とそれ以外を区別することができる.
    身体像の基盤となる概念に身体図式(Body schema)がある.身体像が意識された身体の形状情報に対し,身体図式は習慣としての身体の表像,つまり無意識下に身体運動を調整している主体であり顕在的な知識があるとは限らない.身体像は身体図式を基盤として構成される顕在的な自己身体に関する知識を指す\cite{nishida-bodyimage}.
    % コトバンクに詳しく書かれている
    身体像や身体図式は,幻肢痛\cite{Ramachandran}の観察により生じた概念である.幻肢痛とは,事故で存在しないはずの失った手や脚に痛みを感じる症状である.幻肢痛が発症する仕組みのとして,脳における各機能の分布(脳地図\cite{池谷裕二2007進化しすぎた脳})が書き変わり,幻肢を自分の意思で動かせないことが原因として挙げられる.
    % 脳地図(ブロードマン脳地図)

\section{身体像拡張}
    自己以外の部分に身体像がダイナミックに変化することがある.このことを身体像の拡張(Body image expansion)と呼ぶ.身体像拡張の例として,手に持った道具(テニスラケットや野球バット)を,その形状を意識せず自分の体の一部であるかのように球を打ち返すといった事が挙げられる\cite{渡辺貴文2005仮想道具による身体像拡張の評価手法に関する研究}.
    身体像の拡張は,言い換えると感覚の情報処理を神経系空拡張する能力の事である.

    身体像拡張は,大きく分類して2種類存在し,1つはラバーハンド錯覚\cite{botvinick1998rubber}のような感覚のリマッピング,もう1つは先に挙げた道具使用時に身体像がダイナミックに拡張することである.

    ラバーハンド錯覚とは, ラバーハンドをあたかも自分の手のように感じる錯覚である..視界から隠れた本物の手と目の前にあるラバーハンドに絵筆等で2分から20分程度同期した触覚刺激を与え続けると,ラバーハンド上に触覚刺激を知覚するという錯覚現象である.

    道具への身体像拡張のとして,二ホンザルを用いた道具への身体像拡張を神経生理学的に示した研究がある\cite{iriki1996coding}.この研究では,道具使用時の二ホンザルの登頂連合野における手の体性感覚受容やと手近傍の視覚受容野を持つバイモーダルニューロンの活動を観測することにより,サルの身体像が道具先端まで拡張している事を示した.

    ラバーハンド錯覚と同様な身体像拡張の研究に関しては,視触覚を同期することで,遠隔にあるロボットやアバターへ乗り移ったような感覚が生成可能ということが知られてる
    \cite{tachi2015telexistence}\cite{iwasaki2017research}.
    道具への身体像拡張に関しては,手先から道具への身体像において,余剰筋力を用いて第3の腕となるロボットアームを操作する研究\cite{iwadare2017thirdarm}\cite{岩垂真哉2016余剰筋力を用いた第三の腕ロボットの操縦}や顔面ベクトルを用いて第3の腕を操作する研究\cite{iwasaki2017research},両足を用いて第3・第4の腕を操作する研究がある\cite{sasaki2017metalimbs}.
    
    このように,ヒトと自由度やダイナミクスが類似した遠隔ロボットやアバタを,身体動作と完全に同期させることで,乗り移ったような感覚が生成可能ということが知られてる.しかし,身体像拡張の研究において身体像をヒトと異なる構造の対象に投射することや,四肢以外から身体像を拡張させる知見はまだ少ないのが現状である.


    本研究では,身体像拡張の中でも道具への身体像拡張に注目する.ラバーハンド錯覚と道具への身体像拡張のどちらにおいても,視覚情報と触覚情報の位置が一致または近しければ身体像を拡張することが可能と考える.また,ヒトから情報を送信し,それに対する返信を受け取ることで身体像の拡張をより円滑にすることができると考える.
    このように,身体像の拡張には情報の双方向性が重要であることを踏まえ,本研究では\figref{How2present_the_feeling_of_flapping_eng.pdf}のような形で身体像の拡張を行う.
    ヒトから仮想翼へは,翼を動かす指令を与える.仮想翼からヒトへは,翼が生えている様子,翼を動かして飛んでいる様子,翼へ作用する空気抵抗の感覚を伝える.上記より,四肢から独立した翼で羽ばたいて飛ぶ感覚を提示する.\\


\section{おわりに}
    本章では,身体像の概要と身体像拡張の例と方法について述べた.そして身体像の拡張に着目しヒトに本来無い「翼」を感じさせる方法,その翼で羽ばたいて飛ぶ感覚の提示方法について述べた.
    
