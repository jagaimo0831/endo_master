\chapter[身体像の拡張]%
        {身体像の拡張}

\section{はじめに}
    本研究において以下の要素が重要となる.
    \begin{itemize}
        \item ヒトに本来存在しない「翼」を感じさせる(存在)
        \item その翼で「羽ばたいて飛ぶ感覚」を提示する(運動)
    \end{itemize}
    上記の感覚を与えるために,身体像の拡張について注目する.本章では,身体像について説明し,身体像の拡張の仕組みと方法について述べる.

\section{身体像}
    ヒトは身体像と呼ばれる,自分自身の身体に関するイメージを持っている.
    自身の身体形状を知覚する能力を有している.それにより自己とそれ以外を区別することができる.
\section{身体像拡張}
    自己以外の部分に身体像が拡張する場合がある.身体像の拡張に関する代表的な研究として,

% 道具の身体像拡張の例
    ニホンザル道具使用の    

% これはリマッピングの例
    % 偽物の手である
    ラバーハンドをあたかも自分の手のように感じるラバーハンド錯覚についての研究がある\cite{botvinick1998rubber}.視界から隠れた本物の手と目の前にあるラバーハンドに絵筆等で2分から20分程度同期した触覚刺激を与え続けると,ラバーハンド上に触覚刺激を知覚するという錯覚現象である.
    このように提示される視覚情報と,触覚情報の位置が一致または近しければ身体像を拡張することが可能となる.
    ラバーハンド錯覚ではヒトは情報を受けとるだけであったが,ヒトから情報を送信し,それに対する返信を受け取ることで身体像の拡張をより円滑にすることができると考える.

    身体像の拡張には情報の双方向性が重要であることを踏まえ,本研究では\figref{How2present_the_feeling_of_flapping_eng.pdf}のような形で身体像の拡張を行う.
    ヒトから仮想翼へは,翼を動かす指令を与える.仮想翼からヒトへは,翼が生えている様子,翼を動かして飛んでいる様子,翼へ作用する空気抵抗の感覚を伝える.上記より,四肢から独立した翼で羽ばたいて飛ぶ感覚を提示する.\\



・感覚\\
・用語\\



    \begin{itemize}
    \item 幻肢痛
    \item 脳(まずは池谷先生の本)
    \end{itemize}

\section{おわりに}
    本章では,身体像の拡張に着目しヒトに本来無い「翼」を感じさせる方法,その翼で羽ばたいて飛ぶ感覚の提示方法について述べた.
    
