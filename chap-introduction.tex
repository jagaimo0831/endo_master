\chapter[序論]%
        {序論}
\section{研究の背景}
        \fig{WingMan.png}{width=1\hsize}{Flying with flapping virtual wings independent of the limbs}

        \fig{How2present_the_feeling_of_flapping_eng.pdf}{width=1\hsize}{Research concept}

        ヒトは古くから空を飛ぶことに憧れを抱いている.
        これまで私たちは,飛行機やハンググライダーといった乗り物を用いることで飛行体験をしてきた.
        また,個人飛行装置\footnote{Portable Parsonal Airmobility System.}(Ex. ジェットパック,動力式ウイングスーツ,動力式パラフォイル\footnote{風により展開される柔軟構造を持つ翼.Ex.パラグライダーの翼.}
        )のような,ウェアラブルな装置で空を飛ぶ装置の研究も行われてい\cite{gravityindustries}.
        % 実際に飛ぶことにはリスクやコストが伴うが,VR装置を使用することで簡単に飛行体験が可能である.
        しかし,実際に空を飛ぶことは墜落などのリスクや燃料といったコスト,機器を操縦するための技術が必要となる.
        VR装置を使用することでリスクやコストを回避し,乗り物・ウェアラブルな装置に関わらず簡単に飛行体験が可能となる.

        \figref{WingMan.png}は,四肢から独立した翼で羽ばたいて飛ぶ様子を示した物である.
        本研究では,VR装置を用いて\figref{WingMan.png}のように背中から生えた翼で羽ばたいて飛ぶ感覚の提示手法を提案する.

\section{本論文での飛ぶ感覚の位置付け}
        本論文では,「浮遊感」,「飛ぶ感覚」,「羽ばたいて飛ぶ感覚」を\figref{Classification_of_floating_feeling.png}のように位置付ける.
        \fig{Classification_of_floating_feeling.png}{width=1\hsize}{Classification of floating feeling}

        \begin{itemize}
                \item 浮遊感\\
                ...空中に浮いて漂っている感覚.
                \item 「飛ぶ感覚」\\
                ...「浮遊感」に空中を移動する感覚を追加した感覚
                \item 「羽ばたいて飛ぶ感覚」\\
                ...「飛ぶ感覚」に翼を羽ばたかせる感覚を追加した感覚
        \end{itemize}

\section{研究の目的}
% ここら辺をもっと充実させるのが良い(文章を増やすというよりは,引用の件数を増やす感じ,種類が多くなる)

        VR装置を用いた「浮遊感」や「飛ぶ感覚」を与える研究は多く行われてきた.視覚刺激による落下感覚についての研究\cite{奥川夏輝2017VR空間における視覚刺激によって発生する落下感覚の分析}や,体験者の腕のポーズを認識してハンモック上のネットを動かすことで飛行体験を提供する研究\cite{鈴木拓馬2014hmd}がある.

        \fig{Birdly.jpg}{width=1\hsize}{System of presenting the sensation of flying with flapping \cite{rheiner2014birdly}}

        しかし,「羽ばたいて飛ぶ感覚」を与える研究に関してはまだ知見が少なく,\figref{Birdly.jpg}のように大がかりな装置を用いたものがほとんどである\cite{rheiner2014birdly}\cite{hypersuit}.
        また,従来の「羽ばたいて飛ぶ感覚」を与える研究では,装置にうつ伏せになり実際に腕を動かすことによって,翼で羽ばたいて飛ぶ感覚を提示している.この方法の場合,体の動きが制限されることにより,飛行体験中の動きが限定されてしまう.
        
        % 大きく体を動かすことによる疲労感が生まれ,VR体験に影響を与える.(※疲労感がVR体験に影響を与えるかどうかのソースがあると説得力が上がる)
        % 体の動きが制限されることによるデメリット等が言えると更に良い.

        本研究では,四肢の動きを用いないで翼を操作し羽ばたく感覚を提示する手法を提案する.四肢の動きを用いないことで,VR飛行体験の質(※)が向上する.また,VR飛行体験中に手足を用いた動作(Ex. 飛びながら物を投げるといった行為)が可能となりVR飛行体験の幅が広がる.
        (※質とは?)

        被験者実験を行い,提案した手法の有効性を検証した.

\section{研究の目的}
        従来の羽ばたいて飛ぶ感覚を与える研究は,ヒトが鳥のように腕を動かすことによって羽ばたいて飛ぶ感覚を提示していた.しかし,大がかりな装置が必要であることや,手足の動きが制限されるといったデメリットが存在する.
        また,羽ばたいて飛ぶ感覚を与える研究において,\figref{WingMan.png}のような背中から翼が生えた生物になる感覚を提示する研究はまだ着目されていない.

        そこで四肢から独立した翼を用いて,羽ばたいて飛ぶ感覚を提示する方法を提案する.
        本研究では,
        % 大がかりな装置を使用しないこと,(核ではないので排除)
        四肢を用いずに翼を操作している感覚の提示手法とVR空間で翼に作用する力をヒトに提示する手法を考案する.
        これにより,羽ばたいて飛んでいる状態での投擲や射撃が可能となる.これらのように飛行体験中に手足を使用する新しいアプリケーションが期待できる.

        本研究では,ヒトに翼が生えている感覚を与えるために,身体像の拡張について注目をする.


\section{本論文の構成}
% 章の数とそれぞれの概要を書く

        本論文は全x章で構成される.以下に拡張の概要を述べる.

        \begin{itemize}
                \item 第1章(本章)では,本研究の背景と関連研究,目的について述べた.
                \item  第2章hogehogeでは,
                \item 第3章
                \item 第4章
                \item 第5章
                \item 第6章
        \end{itemize}
        


        
