\chapter[序論]%
        {序論}

        \fig{WingMan.png}{width=1\hsize}{Flying with flapping virtual wings independent of the limbs}

        \fig{How2present_the_feeling_of_flapping_eng.pdf}{width=1\hsize}{Research concept}

        ヒトは古くから空を飛ぶことに憧れを抱いている.
        これまで私たちは,飛行機やハンググライダーといった乗り物を用いることで飛行体験をしてきた.
        また,個人飛行装置\footnote{Portable Parsonal Airmobility System.}(Ex. ジェットパック,動力式ウイングスーツ,動力式パラフォイル\footnote{風により展開される柔軟構造を持つ翼.Ex.パラグライダーの翼.}
        )のような,ウェアラブルな装置で空を飛ぶ装置の研究も行われている\cite{gravityindustries}.
        % 実際に飛ぶことにはリスクやコストが伴うが,VR装置を使用することで簡単に飛行体験が可能である.
        しかし,実際に空を飛ぶことは墜落などのリスクや燃料といったコスト,機器を操縦するための技術が必要となる.
        VR装置を使用することでリスクやコストを回避し,乗り物・ウェアラブルな装置に関わらず簡単に飛行体験が可能となる.
        \figref{WingMan.png}は,四肢から独立した翼で羽ばたいて飛ぶ様子を示した物である.
        本研究では,VR装置を用いて\figref{WingMan.png}のように背中から翼が生えた生物になり羽ばたいて飛ぶ感覚の提示手法を提案する.

\section{本論文での飛ぶ感覚の定義}
        本論文では,「浮遊感」,「飛ぶ感覚」,「羽ばたいて飛ぶ感覚」を\figref{Classification_of_floating_feeling.png}のように位置付ける.
        \fig{Classification_of_floating_feeling.png}{width=1\hsize}{Classification of floating feeling}

        \begin{itemize}
                \item 浮遊感\\
                ...空中に浮いて漂っている感覚.
                \item 「飛ぶ感覚」\\
                ...「浮遊感」に空中を移動する感覚を追加した感覚
                \item 「羽ばたいて飛ぶ感覚」\\
                ...「飛ぶ感覚」に翼を羽ばたかせる感覚を追加した感覚
        \end{itemize}

\section{研究の背景と目的}
% ここら辺をもっと充実させるのが良い(文章を増やすというよりは,引用の件数を増やす感じ,種類が多くなる)(ここには飛ぶ研究についてのciteを書く,身体像拡張については後),ここは軽く説明する程度?

        VR装置を用いた「浮遊感」や「飛ぶ感覚」を与える研究は多く行われてきた.視覚刺激をによって発生する落下感覚に関しての研究\cite{奥川夏輝2017VR空間における視覚刺激によって発生する落下感覚の分析}や身体幇助メカニズムを用いた飛行体験装置の提案\cite{鈴木拓馬2014hmd}等がある.また,飛行しているドローンを上半身のジェスチャーで制御し,ドローンからの映像をヘッドマウントディスプレイ(以下HMD)によって与えることで飛ぶ感覚を提示する研究\cite{rognon2018flyjacket}もある.

        \fig{Birdly.jpg}{width=1\hsize}{System of presenting the sensation of flying with flapping \cite{rheiner2014birdly}}

        「羽ばたいて飛ぶ感覚」を与える研究に関してはまだ知見が少ない.例として\figref{Birdly.jpg}のような,飛行中の鳥の体験をすることができる装置がほとんどである\cite{rheiner2014birdly}\cite{hypersuit}.
        上記装置は,操縦装置にうつ伏せで搭乗し手と腕を用いて翼を動かしながら,鳥視点での景色の映像を提示することで,飛行中の鳥のような体験できる装置である.この方法の場合,大がかりな装置が必要であることや,手足の動きが制限されるといったデメリットが存在する.
        また,「羽ばたいて飛ぶ感覚」を与える研究において,鳥になり飛ぶ感覚を与える研究が大半であり,トビトカゲのような四肢から独立した翼を持つ生物になり,飛ぶ感覚を与える研究は未だ着目されていない.
% 大きく体を動かすことによる疲労感が生まれ,VR体験に影響を与える.(※疲労感がVR体験に影響を与えるかどうかのソースがあると説得力が上がる)
% 体の動きが制限されることによるデメリット等が言えると更に良い.


        本研究では,四肢の動きを用いないで背中から生えた翼を操作し羽ばたく感覚を提示する手法を提案する.四肢の動きを用いないことで,VR飛行体験中に手足を用いた動作(Ex. 飛びながら物を投げるといった行為)が可能となりVR飛行体験の幅が広がることが期待できる.
% 四肢の動きを用いないことで,VR飛行体験の質(※)が向上する.(※質とは?)その根拠となる論文は?

% 身体像拡張についてはここで触れる?,とりあえず他を書いてから出良い

\section{本論文の構成}
% 章の数とそれぞれの概要を書く

        本論文は全x章で構成される.以下に拡張の概要を述べる.

        \begin{itemize}
                \item 第1章(本章)では,本研究の背景と目的について述べた.
                \item 第2章「hogehoge」では,
                \item 第3章「」では,
                \item 第4章「」では,
                \item 第5章「」では,
                \item 第6章「」では,
        \end{itemize}
        


        
