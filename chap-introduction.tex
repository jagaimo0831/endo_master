\chapter[序論]%
        {序論}

        \fig{WingMan.png}{width=1\hsize}{Flying with flapping virtual wings independent of the limbs}

        ヒトは古くから空を飛ぶことに憧れを抱いている.
        これまで私たちは,飛行機やハンググライダーといった乗り物を用いることで飛行体験をしてきた.
        また,個人飛行装置\footnote{Portable Parsonal Airmobility System...ジェットパック,動力式ウイングスーツ,動力式パラフォイル(風により展開される柔軟構造を持つ翼.Ex.パラグライダーの翼)}
        のような,ウェアラブルな装置で空を飛ぶ研究も行われている\cite{gravityindustries}.
        % 実際に飛ぶことにはリスクやコストが伴うが,VR装置を使用することで簡単に飛行体験が可能である.
        しかし,実際に空を飛ぶことは墜落などのリスクや燃料といったコスト,機器を操縦するための技術が必要となる.
        VR(Virtual Reality: 仮想現実)システムを使用することで,それらリスクやコストを回避し,乗り物・ウェアラブルな装置を問わず簡単に飛行体験が可能となる.

        \figref{WingMan.png}は,四肢から独立した翼で羽ばたいて飛ぶ様子を示した図である.
        本研究では,VRシステムを用いて\figref{WingMan.png}のように,ヒトの背中から翼が生えた生物になり羽ばたいて飛ぶ感覚の提示手法を提案する.

\section{本研究での羽ばたいて飛ぶ感覚の定義}
        本論文では「浮遊感」,「飛ぶ感覚」,「羽ばたいて飛ぶ感覚」を\figref{intro-Classification_of_floating_feeling.pdf}のように位置付ける.
        \fig{intro-Classification_of_floating_feeling.pdf}{width=1\hsize}{Classification of floating feeling}

        \begin{itemize}
                \item 「浮遊感」\\
                ...空中に浮いて漂っている感覚.
                \item 「飛ぶ感覚」\\
                ...「浮遊感」に,空中を移動する感覚を追加した感覚.
                \item 「羽ばたいて飛ぶ感覚」\\
                ...「飛ぶ感覚」に,翼を羽ばたかせる感覚を追加した感覚.
        \end{itemize}

\section{研究の背景と目的}
% VRの飛ぶ研究についてもっと充実させる(分類して一気にciteする感じ)
% 身体像拡張は次章で引用

        VR装置を用いた「浮遊感」や「飛ぶ感覚」を与える研究は多く行われてきた.視覚刺激によって発生する落下感覚に関しての研究\cite{bubka2008expanding}\cite{奥川夏輝2017VR空間における視覚刺激によって発生する落下感覚の分析}や身体幇助メカニズムを用いた飛行体験装置の提案\cite{鈴木拓馬2014hmd}等がある.また,飛行しているドローンを上半身のジェスチャーで制御し,ドローンからの映像をHMD(Head Mounted Display: ヘッドマウントディスプレイ)によって与えることで飛ぶ感覚を提示する研究\cite{piumsomboon2018superman}\cite{higuchi2013flying}\cite{rognon2018flyjacket}もある.

        \fig{Birdly.jpg}{width=0.7\hsize}{System of presenting the sensation of flying with flapping wings\cite{rheiner2014birdly}}

        「羽ばたいて飛ぶ感覚」を与える研究について,\figref{Birdly.jpg}のような操縦装置に搭乗し,飛行中の鳥の体験をすることができる装置の研究が行われている\cite{rheiner2014birdly}\cite{hypersuit}.
        上記装置は,操縦装置にうつ伏せで搭乗し手と腕を用いて翼を動かしながら,鳥視点での景色の映像を提示することで,飛行中の鳥のような体験できる装置である.この方法の場合,大がかりな装置が必要であることや,手足の動きが制限されるといったデメリットが存在する.
% 四肢の動きを用いないことで,VR飛行体験の質(※)が向上する.(※質とは?)その根拠となる論文は?
% 大きく体を動かすことによる疲労感が生まれ,VR体験に影響を与える.(※疲労感がVR体験に影響を与えるかどうかのソースがあると説得力が上がる)
% 体の動きが制限されることによるデメリット等が言えると更に良い.
        また,羽ばたいて飛ぶ感覚を与える研究はまだ知見が少なく,鳥になり飛ぶ感覚を与える研究が大半であり,トビトカゲのような四肢から独立した翼を持つ生物になり,飛ぶ感覚を与える研究は未だ着目されていない.

        本研究では,四肢の動きを用いずに背中から生えた翼を操作し羽ばたく感覚を提示する手法を提案する.四肢の動きを用いないことで,VR飛行体験中に手足を用いた動作,例えば飛びながら物を投げるといった行為,が可能となりVR飛行体験の幅が広がることが期待できる.

\section{本論文の構成}
        
        本論文は全6章で構成される.以下に各章の概要を述べる.

        \begin{itemize}
                \item 第1章「序論」では,羽ばたいて飛ぶ感覚の定義,本研究の背景と目的について述べた.
                
                \item 第2章「身体像の拡張」では,身体像について説明し,身体像の拡張の仕組みと方法について示し,本研究での身体像拡張のアプローチについて述べる.
                
                \item 第3章「四肢から独立した翼の提示方法」では,四肢から独立した翼の提示方法について述べる.
                
                \item 第4章「提案手法を用いた身体像拡張の主観評価実験」では,提案した身体像拡張の手法を用いて主観評価実験を行う.操作・提示方法の検討,操作・提示位置の検討を行い,それぞれの組み合わせの評価を下す.主観評価実験の結果を踏まえ,被験者実験で比較する対象について述べる.
                
                \item 第5章「主観評価実験を踏まえた位置による身体像拡張の差異を評価する被験者実験」では,主観評価実験を踏まえた位置による身体像拡張の差異を評価する被験者実験を行う.筋電計測位置と羽ばたく感覚の提示位置を変化させた場合の,羽ばたいて飛ぶ感覚の感じ方の違い,触覚提示として振動と電気刺激を用いた装置を比較し検証を行う.
                
                \item 第6章「結論および今後の展望」では,本研究の結論と今後の展望について述べる.
        \end{itemize}
        
        


        
