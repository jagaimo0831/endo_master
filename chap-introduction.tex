\chapter[序論]%
        {序論}
\section{研究の背景}
        \fig{WingMan.png}{width=1\hsize}{Flying with flapping virtual wings independent of the limbs}

        \fig{How2present_the_feeling_of_flapping_eng.pdf}{width=1\hsize}{Research concept}

        ヒトは古くから空を飛ぶことに憧れを抱いている.実際に飛ぶことにはリスクやコストが伴うが,VR装置を使用することで簡単に飛行体験が可能である.

        \figref{WingMan.png}は,四肢から独立した翼で羽ばたいて飛ぶ様子を示した物である.
        本研究では,
        VR装置を用いて\figref{WingMan.png}のように羽ばたいて飛ぶ感覚の提示手法を提案する.

        従来の研究では,実際に腕を動かすことによって翼で羽ばたいて飛ぶ感覚を提示している.この方法の場合,大きく体を動かすことによる疲労感が生まれ,VR体験に影響を与える.
        (※疲労感がVR体験に影響を与えるかどうかのソースがあると説得力が上がる)

        本研究では,四肢の動きを用いないで翼を操作し羽ばたく感覚を提示する手法を提案する.四肢の動きを用いないことで,VR飛行体験の質(※)が向上する.また,VR飛行体験中に手足を用いた動作(Ex. 飛びながら物を投げるといった行為)が可能となりVR飛行体験の幅が広がる.
        (※質とは?)

        被験者実験を行い,提案した手法の有効性を検証した.

\section{浮遊感・飛ぶ感覚・羽ばたいて飛ぶ感覚の定義}
        本研究では「羽ばたいて飛ぶ感覚」を「浮遊感」,「飛ぶ感覚」を用いて次のように位置づける.
        「浮遊感」とは,空中に浮いて漂っている感覚を指す.「飛ぶ感覚」は浮遊感に空中を移動する感覚を追加したものとする.本稿で注目する「羽ばたいて飛ぶ感覚」は,飛ぶ感覚に加え,翼を羽ばたかせる感覚を追加したものである.

\section{研究の目的}
        従来の羽ばたいて飛ぶ感覚を与える研究は,ヒトが鳥のように腕を動かすことによって羽ばたいて飛ぶ感覚を提示していた.しかし,大がかりな装置が必要であることや,手足の動きが制限されるといったデメリットが存在する.
        また,羽ばたいて飛ぶ感覚を与える研究において,\figref{WingMan.png}のような背中から翼が生えた生物になる感覚を提示する研究はまだ着目されていない.

        そこで四肢から独立した翼を用いて,羽ばたいて飛ぶ感覚を提示する方法を提案する.
        本研究では,
        % 大がかりな装置を使用しないこと,(核ではないので排除)
        四肢を用いずに翼を操作している感覚の提示手法とVR空間で翼に作用する力をヒトに提示する手法を考案する.
        これにより,羽ばたいて飛んでいる状態での投擲や射撃が可能となる.これらのように飛行体験中に手足を使用する新しいアプリケーションが期待できる.

        本研究では,ヒトに翼が生えている感覚を与えるために,身体像の拡張について注目をする.


\section{本論文の構成}
        本論文は以下のような構成なっている.

        背景\\
        目的\\
        関連研究\\
        (理論?身体像について, いままで調べてきたことは全てこれに繋がる(VR, 触覚, etc))\\
        方法\\
        結果\\
        考察\\
        おわり\\