\chapter[その他]%
    {その他}

\section{鳥の飛ぶ仕組み}
    \subsection{羽の仕組み}
        \fig{bird-how_to_fly.png}{width=1\hsize}{Wing Quill Mechanism}
        \fig{bird-wing_types.png}{width=1\hsize}{Wing Types}
        
        鳥が空を飛べるのは,翼に「風切羽」という飛ぶための羽が付いているからである.風切羽は翼を持ち上げるときには空気を通すように縦になり(疎になる),下すときには空気を通さないように横に倒れる(密になる)(\figref{bird-how_to_fly.png}).これにより,翼を下すときにのみ力が発生し空を飛ぶことが出来る.

        鳥の羽には種類(\figref{bird-wing_types.png})があり,それぞれ役割が違う.以下に代表的な鳥の羽の種類と役割を示す.

        \begin{itemize}
        \item 初列風切羽\\
            \quad ...進むための羽.羽軸が進行方向にカーブし,左右の幅が違う.これにより羽ばたいた際に,進行方向逆側へ風が生まれ前に進むことができる.

        \item 次列風切羽\\
            \quad 浮かぶための羽.初列風切よりも短く,太く,羽軸がカーブしており左右の幅はほぼ等しい.羽ばたくと下へ向かって風が生まれ,上昇することができる.

        \item 三列風切羽\\
            \quad ...翼と体の間を埋める羽.他の風切羽よりも短い. 翼をたたむと風切羽は重なり合い,小さく折りたたまれる.
            
        \item 尾羽\\
            \quad ...空中でのブレーキや方向転換を行うための羽.次列風切羽と似た形状で,比較的羽軸が真っすぐな羽が多い.
        \end{itemize}

        他にも飛ぶための仕組みとして,発達した胸筋(鳩胸),軽量化のため骨が空洞,短い腸(食事は直ぐ消化し水分と一緒に分とつぃて放出),顎が無いといったことが挙げられる.


        % 参考文献
        % \href{https://global.canon/ja/environment/bird-branch/bird-column/kids2/}{Canon Global 鳥はなぜ飛べるの}
        % \href{https://k-tac.jp/about_feather/}{Kamatac 羽の専門店}

    \subsection{飛び方}
        \begin{itemize}
        \item 直線飛行
        \item 波状飛行
        \item ホバリング
        \item 滑空
        \item はん翔...翼を広げた滑空の姿勢のまま,上昇気流に乗って飛び上がる方法.
        \end{itemize}

        本研究では上記のうち,hogehogeを対象とした飛び方を行っている.
