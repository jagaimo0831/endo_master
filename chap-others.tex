\chapter[その他]%
    {その他}

\section{鳥の飛ぶ仕組み}
    \subsection{羽の仕組み}
        \fig{bird-how_to_fly.png}{width=1\hsize}{Wing Quill Mechanism}
        \fig{bird-wing_types.png}{width=1\hsize}{Wing Types}
        
        鳥が空を飛べるのは,翼に「風切羽」という飛ぶための羽が付いているからである.風切羽は翼を持ち上げるときには空気を通すように縦になり(疎になる),下すときには空気を通さないように横に倒れる(密になる)(\figref{bird-how_to_fly.png}).これにより,翼を下すときにのみ力が発生し空を飛ぶことが出来る.

        鳥の羽には種類(\figref{bird-wing_types.png})があり,それぞれ役割が違う.以下に代表的な鳥の羽の種類と役割を示す.

        \begin{itemize}
        \item 初列風切羽\\
            \quad ...進むための羽.羽軸が進行方向にカーブし,左右の幅が違う.これにより羽ばたいた際に,進行方向逆側へ風が生まれ前に進むことができる.

        \item 次列風切羽\\
            \quad 浮かぶための羽.初列風切よりも短く,太く,羽軸がカーブしており左右の幅はほぼ等しい.羽ばたくと下へ向かって風が生まれ,上昇することができる.

        \item 三列風切羽\\
            \quad ...翼と体の間を埋める羽.他の風切羽よりも短い. 翼をたたむと風切羽は重なり合い,小さく折りたたまれる.
            
        \item 尾羽\\
            \quad ...空中でのブレーキや方向転換を行うための羽.次列風切羽と似た形状で,比較的羽軸が真っすぐな羽が多い.
        \end{itemize}

        他にも飛ぶための仕組みとして,発達した胸筋(鳩胸),軽量化のため骨が空洞,短い腸(食事は直ぐ消化し水分と一緒に分とつぃて放出),顎が無いといったことが挙げられる.


        % 参考文献
        % \href{https://global.canon/ja/environment/bird-branch/bird-column/kids2/}{Canon Global 鳥はなぜ飛べるの}
        % \href{https://k-tac.jp/about_feather/}{Kamatac 羽の専門店}

    \subsection{飛び方}
        \begin{itemize}
        \item 直線飛行
        \item 波状飛行
        \item ホバリング
        \item 滑空
        \item はん翔...翼を広げた滑空の姿勢のまま,上昇気流に乗って飛び上がる方法.
        \end{itemize}

        本研究では上記のうち,hogehogeを対象とした飛び方を行っている.


        \section{流体シミュレータについて} 
        %流体シミュレータは結局使わなかったけど折角調べてあったのでとりあえず書いとく...
            空気から受ける力をシミュレーションし,その力をヒトへ与えることで翼で羽ばたいて飛ぶ感覚を提示する.空気から受ける力をシミュレートするのに流体シミュレータを用いる.使用する流体シミュレータの候補として以下のソフトウェアが挙げられる.
        
            
                \begin{itemize}
                \item \href{http://flowsquare.com/jp/}{Flowsquare}
                    \begin{itemize}
                    \item 開発: Nora Scientific(2009年)
                    \item 特徴: 2次元非定常,非反応/反応性,完全圧縮性/非圧縮性流体のシミュレーションソフト 
                    \item 対応OS: Windows
                    \item 料金: 無料
                    % \item 無料(典型的な流体シミュレーションソフトは1ライセンスあたり数10万くらい(参考:\href{https://icfd.co.jp/product/price.html}{株式会社流体力学研究所})
                    % \item 専門知識(プログラミング・CAD・メッシュ生成・前処理(初期場ほ生成)・後処理etc)を必要としない.\\
                    % -\textgreater ペイントソフトを用いて解析対象の絵を書く,解析したい条件(流体速度)をテキストファイルに入力
                    \end{itemize}
                    
                \item \href{https://fsp.norasci.com/}{Flowsquare+}
                    \begin{itemize}
                    \item 開発: Nora Scientific
                    \item 特徴:
                        \begin{itemize}
                        \item Flowsqureの新バージョン.
                        \item 3次元の解析に対応
                        \item CFD(Computational Fluid Dynamics:数値流体力学)搭載
                        \end{itemize}
                    \item 対応OS: Windows
                    \item 料金: 無料
                    % \textless\textless 通常100万以上のコスト
                    % \item 以前と同様に専門知識不要
                    \end{itemize}
                    
                \item \href{https://fastar.chofu.jaxa.jp/}{FaSTAR}
                    \begin{itemize}
                    \item 開発: JAXA (宇宙航空研究開発機構)
                    \item 特徴: 
                        \begin{itemize}
                        \item Fast Unstructuired CFD Code
                        \item 高速非構造格子(任意の形状のメッシュ)に対応した圧縮性流体解析ソルバー
                        \item 航空機や宇宙器などの空力解析に適する
                        \end{itemize}
                    \item 料金: 授業等の教育目的に限り無償で提供
                    \end{itemize}
                    
                \item \href{https://altairhyperworks.jp/product/ultrafluidx}{ultraFluidX}
                    \begin{itemize}
                    \item GPUが必要 (というかサーバーが1基必要...)
                    \end{itemize}
                
                \item \href{https://www.openfoam.com/}{OpenFOAM}
                
                \item \href{http://www.ciss.iis.u-tokyo.ac.jp/dl/}{FrontFlow/blue}
                    \begin{itemize}
                    \item 国産
                    \item blue: 乱流音場用,  red: 乱流燃焼用
                    \end{itemize}
                
                \item \href{http://www.cenav.org/kdb/?page_id=328}{FrontFlow/violet Cartesian}
                    \begin{itemize}
                    \item 直交格子を用いた実用複雑系流体解析プログラム
                    \end{itemize}
                
                \item \href{http://www.cenav.org/kdb/?page_id=334}{FrontWorkBench}
                    \begin{itemize}
                    \item 流体・構造・音響錬成解析の自動設定
                    \end{itemize}
                
                \item \href{https://www.blender.org/download/}{Blender}
                    \begin{itemize}
                    \item コンピュータグラフィックスソフトで有名
                    \item Unityでも流体解析はできる
                    \end{itemize}
        
                \item \href{https://fenicsproject.org/}{FEniCS}
                    \begin{itemize}
                    \item pythonやC++で開発可能
                    \item 英語
                    \end{itemize}
                
                \end{itemize}
        
                手持ちのノートPCのスペックで使用可能(コロナで在宅な為),無償,3次元シミュレーションが出来る,という観点から今回はFlowSqure+を使用する.((美術)解剖学的には人間の形を保ったまま,背中から生えた翼でバランスよく飛翔することは困難であるので,現実的にはあまり意味はない解析である(\href{https://genkosha.pictures/illustration/18103116710}{小田隆 PICTURES 美しい美術解剖図 第2回 人体に翼を生やすことは可能か?キューピッドを美術解剖図で考察する})).
